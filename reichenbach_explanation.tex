% !TEX encoding = UTF-8 Unicode
\documentclass[final,12pt]{article}
\usepackage{appendix}
\usepackage{els}
\usepackage{caption}
\captionsetup{font=scriptsize}
\usepackage{notate}
\usepackage{enumitem}
\newlist{enumerate*}{enumerate}{1}
\usepackage{mat2}
\newcommand{\Mink}{Minkowski\xspace}
\renewcommand{\rev}[1]{Review of \cite{#1}}

\newcommand{\tras}[1]{tr.~as \cite{#1}}
\newcommand{\trin}[1]{tr.~as \cite{#1}}
\newcommand{\PRZL}{\citetitle{Reichenbach1928}\xspace}
%\newcommand[\rzla]{\citep[#1]{Reichenbach1928}; tr.~\cite[041-2101, #2]{HR}}
\newcommand{\crc}[2]{(\cite[#1]{Reichenbach1925}/#2)}
%; tr.~\cite*[#2]{Reichenbach2006}
%\cite[#3]{Reichenbach1926a}; tr.~\cite*[#4]{Reichenbach2006}}
%\cite[#3]{Reichenbach1924}; tr.~\cite*[#4]{Reichenbach1969}
\newcommand{\crw}[2]{(\cite[#1]{Reichenbach1926a}/#2)}



\DeclareSourcemap{
  \maps[datatype=bibtex]{
    \map{
      \step[fieldsource=entrykey, match=Einstein1925d, fieldset=options,  fieldvalue={skipbib=true}]
    }
  }
}



\usepackage{etoolbox}

\pretocmd{\everypar}{\citereset}{}{}

%; tr.~\cite*[#2]{Reichenbach2006}

\RenewDocumentCommand\qa{mmoo}{\blockquote[{\cite[#3]{Reichenbach1928}; tr.~\cite[041-2101, #4]{HR}}]{#1} \orig{#2}\xspace}
\RenewDocumentCommand\qah{mmoo}{\blockquote[{\cite[#3]{Reichenbach1928}; tr.~\cite[041-2101, #4]{HR}}]{#1} \hide{#2}\xspace}

\newcommand{\crz}[2]{(\cite[#1]{Reichenbach1928}/#2)\xspace}
\renewcommand{\rzl}[2]{(\cite[#1]{Reichenbach1928}/#2)\xspace}
\renewcommand{\rzlp}[2]{\cite[#1]{Reichenbach1928}/#2\xspace}
%; tr.~\cite*[#2]{Reichenbach1958}

%\NewDocumentCommand\qrz{mmoo}{\blockquote[{\cite[#3]{Reichenbach1928}; tr.~\cite*[#4]{Reichenbach1958}}]{#1} \orig{#2}\xspace}
%\NewDocumentCommand\qrzh{mmoo}{\blockquote[{\cite[#3]{Reichenbach1928}; tr.~\cite*[#4]{Reichenbach1958}}]{#1} \hide{#2}\xspace}
\NewDocumentCommand\qrw{mmoo}{\blockquote[{\cite[#3]{Reichenbach2006}; tr.~\cite*[#4]{Reichenbach1958}}]{#1} \orig{#2}\xspace}

\NewDocumentCommand\qrwh{mmoo}{\blockquote[{\cite[#3]{Reichenbach1926a}; tr.~\cite*[#4]{Reichenbach2006}}]{#1} \hide{#2}\xspace}

\NewDocumentCommand\qrc{mmoo}{\blockquote[({\cite[#3]{Reichenbach1925}; tr.~\cite*[#4]{Reichenbach2006}})]{#1} \orig{#2}\xspace}
\NewDocumentCommand\qrch{mmoo}{\blockquote[{\cite[#3]{Reichenbach1925}; tr.~\cite*[#4]{Reichenbach2006}}]{#1} \hide{#2}\xspace}
\NewDocumentCommand\qax{mmoo}{\blockquote[{\cite[#3]{Reichenbach1925}; tr.~\cite*[#4]{Reichenbach1969}}]{#1} \orig{#2}\xspace}
\NewDocumentCommand\qaxh{mmoo}{\blockquote[{\cite[#3]{Reichenbach1924}; tr.~\cite*[#4]{Reichenbach1969}}]{#1} \hide{#2}\xspace}

\NewDocumentCommand\lK{mm}{\ensuremath{l^{K{#1}}_{K{#2}}}}
\NewDocumentCommand\LK{mm}{\ensuremath{L^{K{#1}}_{K{#2}}}}


\newcommand{\MME}{Michelson–Morley experiment\xspace}
\newcommand{\ED}{ether-drift\xspace}
\renewcommand{\qrw}{\qrwh}
\renewcommand{\qa}{\qah}
\renewcommand{\qrc}{\qrch}
\renewcommand{\qrz}{\qrzh}
\renewcommand{\qax}{\qaxh}
%\newcommand{\SN}{\cite{SN}}
%\newcommand{\Ap}{Appendix\xspace}
\renewcommand{\hr}[1]{HR 041-2101, #1\xspace}

\newcommand{\rep}[2]{(\cite[#1]{Reichenbach1920a}/#2)\xspace}

\renewcommand{\rzla}[2]{(\cite[#1]{Reichenbach1928}; tr.~\cite*[041-2101, #2]{HR})\xspace}
\newcommand{\cax}[2]{(\cite[#1]{Reichenbach1924}; tr.~\cite*[#2]{Reichenbach1969})\xspace}
\newcommand{\ra}[2]{(\cite[#1]{Reichenbach1924}/#2)\xspace}
\newcommand{\rc}[2]{(\cite[#1]{Reichenbach1925}/#2)\xspace}
\newcommand{\rw}[2]{(\cite[#1]{Reichenbach1926a}/#2)\xspace}
\newcommand{\gs}[2]{(\cite[#1]{Reichenbach1922a}/#2)\xspace}


%\renewcommand{\rzl}[2]{(\cite[#1]{Reichenbach1928}; tr.~\cite*[#2]{Reichenbach1958})\xspace}
\renewcommand{\theequation}{\roman{equation}}

\newcommand{\rhp}[2]{(\cite[#1]{Reichenbach1920a}; tr.\ \citeyear{Reichenbach1969} #2)\xspace}

\newcommand{\wpo}{worldpoint\xspace}

\begin{document}

\title{Reichenbach and the Prehistory of the Dynamical Approach to Special Relativity}
\maketitle

\begin{abstract}
The paper aims to revisit Reichenbach’s interpretation of special relativity, making two different but interrelated claims: \begin{inparaenum}[(I)] \item Reichenbach's interpretation is best characterized not as a variant of the conventionalist interpretation, but rather an early form of the dynamical interpretation; \item Reichenbach offers a more robust version of the dynamical interpretation than contemporary accounts. \end{inparaenum} On this basis, the paper argues that Reichenbach’s approach provides the conceptual resources to \begin{inparaenum}[(I)] \item strengthen the dynamical approach against common criticisms from defenders of the geometrical approach, \item expose on its true weak point of both approaches. \end{inparaenum} Unlike the dynamical approach, special relativity does not require a \emph{specific} theory of matter to explain ether drift experiments; rather, it demands that \emph{any} such theory be Lorentz invariant. Unlike the geometrical approach, \Mink’s formalism helps test this requirement but lacks explanatory power. The paper concludes that, following Lange, \sr provides an \s{explanation by constraint}.
\end{abstract}



%\begin{keywords}
%Hans Reichenbach \sep Lenght Contraction \sep Special Relativity \sep Dynamical Relativity \sep explanation
%\end{keywords}

\intro

%Dürr, P, and J Read. 2024. "An invitation to conventionalism: a philosophy for modern (space-)times." Synthese 204 (1): 1-55. https://doi.org/https://doi.org/10.1007/s11229-024-04605-z.



Harvey Brown's \citeyear{Brown2005} book \citetitle{Brown2005} is widely credited with reshaping the debate on the foundations of space-time theories. Brown argues that special relativity as it stands is incomplete: phenomena like length contraction must ultimately receive a \textit{dynamical explanation} in a fundamental theory of the material structure of rods. Defenders of the traditional view, such as Michel \citet{Janssen2009}, object that special relativity was already completed by \Mink, who provided a theory of the mathematical structure of \spti: length contraction receives a \textit{geometrical explanation} from the fact that the world tube of the rod is intersected differently by the hyperplane of simultaneity. Whatever one’s view, following this debate, the problem of the role of \rac in relativity theory has been reframed \citep[see][for a balanced account]{Acuna2016}: from the logical empiricist concern with \emph{confirmation} (conventional vs. empirical) to a focus on \emph{explanation} (geometrical vs. dynamical).

This paper aims to draw attention to a forgotten chapter in the pre-history of the dynamical approach. It argues that it was the logical empiricist Hans Reichenbach who, already in the 1920s, first drew attention to the problem of \s{explanation} in \spti theories. Reichenbach denied explanatory power to \Mink \spti and, like modern dynamicists, insisted that \sr requires the behavior of \rac to be explained by a \emph{specific, though still unknown, theory of matter}. This paper seeks to draw renewed attention to this aspect of Reichenbach's interpretation of relativity, which has slipped through the cracks of scholarship. In particular, it advances two distinct but interrelated claims:

\begin{enumerate}[label=(a)] 
\item \emph{historical claim}: Reichenbach's axiomatization of special relativity should not be regarded as a variant of Moritz~\citets{Schlick1915}’s conventionalist interpretation, as it is often portrayed, but as an early precursor of the dynamical interpretation;

\item \emph{systematic claim}: Through an analysis of both the notion of \s{contraction} and the notion of \s{explanation}, Reichenbach outlined, in many respects, a more robust version of the dynamical interpretation of \sr than contemporary ones.
\end{enumerate} 
%
Building on these two claims, the paper argues that Reichenbach's work provides the conceptual tools \begin{inparaenum}[(I)] \item to steel-man the dynamical approach against its most common objections; \item to expose the cracks in its strongest armor. \end{inparaenum} The advantage of Einstein's special relativity over Lorentz's ether theory is precisely that the former does not need to explain the failure of ether drift experiments by \emph{finding} a \emph{specific} Lorentz-invariant theory of matter; rather, relativity explains such negative results by \emph{requiring} that \emph{any} possible theory of matter \emph{must} be Lorentz-invariant. In Einstein's view, \Mink's mathematical apparatus facilitates establishing whether available laws comply with this requirement but, contrary to the geometrical approach, does not provide any further explanatory contribution. The paper concludes by arguing that, if one wants to cast the contribution of \sr relativity in explanatory terms, it is better to speak of an \emph{explanation by constraint}, as suggested by Marc \citet{Lange2016}.
%\lipsum[2]

\section{The Schlick–Reichenbach Correspondence and the Emergence of Reichenbach's Axiomatization}
\label{schlickreichenbach}

Reichenbach's \citeyear{Reichenbach1920a} habilitation thesis, \citetitle{Reichenbach1920a}, appeared in print during the famous 86th meeting of the \german{Gesellschaft deutscher Naturforscher und Ärzte} (GDN\"A) in Bad Nauheim (19–25 September), marking the beginning of a politically charged backlash against relativity in Germany \citep{Weyl1920}. Schlick, who did not attend the meeting, received the booklet around that time \letterhrp{Schlick}{Reichenbach}{25}{9}{1920}[015-63-23]. Writing to Einstein, he praised it but complained about Reichenbach's \s{neo-Kantian} critique of conventionalism \lettercpaep{Schlick}{Einstein}{23}{9}{1920}[10][171s]. Schlick articulated his stance in correspondence with Reichenbach over the ensuing months \citep{Padovani2009}.

Schlick famously objected that, upon closer inspection, Reichenbach's \apr coordinating principles were nothing but arbitrary \scare{conventions} in the sense of Poincaré \letterhrp{Schlick}{Reichenbach}{26}{11}{1920}[015-63-22]. Reichenbach initially offered some resistance. If the coordinating principles were fully arbitrary, he feared, they would be empirically meaningless. In Poincaré's conventionalism, Reichenbach missed a restriction in \q{the arbitrariness of the principles \textelp{}, if the principles are combined}; therefore, he concluded, \qt{I cannot accept the term \s{convention}{}}{daß die Willkürlichkeit der Prinzipien eingeschränkt ist, sowie man Prinzipien KOMBINIERT. Darum kann ich den Namen \s{Konvention} nicht annehmen} \letterhrp{Reichenbach}{Schlick}{26}{11}{1920}[015-63-22]. Schlick replied that it would, of course, be unfair to assume he was unaware of this fact \letterhrp{Schlick}{Reichenbach}{11}{12}{1920}[015-63-22]. Indeed, in January, \citet[110\psq]{Schlick1921}, while mentioning Reichenbach's book favorably in an article for the \jt{Kant-Studien}, reiterated in public the objections he had raised in private correspondence.

Reichenbach did not seem to have been fully turned to the conventionalist side at this point. \posscite[126]{Einstein1921} apparent endorsement of Poincaré in his famous January 1921 lecture on \scare{geometry and experience} may have played a role in tipping the scale in favor of conventionalism. In fact, by September 1921, Reichenbach wrote to Schlick that he considered their difference of opinion resolved \letterp{Reichenbach}{Schlick}{10}{9}{1921}[][SN]. He hoped to finally meet Schlick in person in a few days at the Jena \textit{Physikertag}, where he was going to present a report on his project for an axiomatization of \sr (\letter{Reichenbach}{Schlick}{10}{9}{1921}[][SN]). When he sent Schlick the published version of the report, \citet{Reichenbach1921d} emphasized that his axiomatization \qt{obviously provides a confirmation of conventionalism}{Sie liefert natürlich eine Bestätigung des Conventionalismus}; however, he also qualified this remark by insisting that it \q{reveals those facts that even conventionalism cannot interpret} \letterp{Reichenbach}{Schlick}{18}{1}{1922}[][SN]. Schlick was impressed  though he wished to examine the technical details more closely \letterp{Schlick}{Reichenbach}{27}{1}{1922}[][SN]

Reichenbach clarified his position in more accessible terms in a long review paper on the philosophical interpretations of relativity, completed around March 1922 \lettercpaep{Einstein}{Reichenbach}{27}{3}{1922}[13][119]. By outlining his axiomatization of \sr, Reichenbach emphasizes once again that he prefers to \emph{avoid} the term \s{conventionalism}. The problem, he insists, is \qt{not only to detect the \emph{arbitrary} principles of knowledge, but also to determine the totality of admissible combinations}{nicht nur die w i 11kür1 ich e n Prinzipien der Erkenntnis aufzudecken, sondern die Gesamtheit der zulässigen Kombinationen zu bestimmen} \gs{362}{2:39}\footnote{In the remainder of this paper, page numbers after the slash indicate the corresponding passages in the translation}. Reichenbach presents his axiomatization as the solution to this problem. In particular, he emphasizes that the latter was organized around two pairs of distinctions \gs{362f.}{2:39f.}:

\begin{itemize}
\item \emph{axioms} vs.\ \emph{definitions}. In this context, \s{axiom} refers not to a mathematical postulate but to an observable fact, either experimentally confirmed or provisionally assumed. In contrast, \s{definitions} are rules coordinating empirical realities with mathematical concepts. Einstein showed that judging whether two distant events are simultaneous involves a convention about the ratio $\epsilon$ of one-way light speeds in a round trip. Since we cannot measure one-way speed without presupposing simultaneity, Einstein's convention $\epsilon=\sfrac{1}{2}$ is not \s{more correct} than others. The distinction between \s{conventional} elements (definitions) and empirical ones (axioms) limits the wiggle room for Poincaréan adjustments of theory to data. For example, it is a \s{fact} that simultaneity \s{conventions} must be \s{univocal} (independent of prehistory) \citep{Reichenbach1922b}.

\item \emph{light axioms} vs.\ \emph{matter axioms}. Light axioms state only the properties of electromagnetic signals, and matter axioms those of rigid rods and natural clocks. Reichenbach claimes that a \emph{light geometry} alone could ground space and time measurement. While in Einstein’s theory light served only to define simultaneity, in Reichenbach's axiomatization it can be used for all time and distance measurements (via light travel time). The benefit of light geometry is that it avoids defining the \s{metric} using material entities like rods and clocks—complex atomic systems whose behavior depends on physical laws. In Reichenbach's system, \emph{matter geometry} based on \rac appears only after light geometry, as a series of \s{matter axioms}. 
\end{itemize}
%
The separation between these two conceptual pairs is central to Reichenbach's axiomatization: light axioms express facts accepted by classical optics and are thus confirmed independently of special relativity. The difference between classical and relativistic light geometry is a \emph{matter of convention}, depending on the definition of simultaneity. Whether matter geometry agrees with classical or relativistic light geometry is a \emph{matter of fact}, which is empirically testable in principle.  Thus, only the matter axioms encode the physical content specific to relativity: \qt{Einsteinian kinematics rests on the hypothesis that  \textins{relativistic} light geometry is identical with the geometry of rigid rods and natural clocks}{Als grundsätzliche Hypothese der Einsteinsehen Kinematik läßt sich dann der Satz aufstellen, daß die Lichtgeometrie identisch ist mit der Geometrie der starren Maßstäbe und natürlichen Uhren} \gs{365}{2:41}. 

As \citet{Reichenbach1922} points out in a popular French article submitted to the \jt{Revue philosophique} in May, his goal was not merely to \emph{endorse} conventionalism but to \emph{constrain} its scope. Whether Lorentz or Galilei transformations apply to light signals depends, it is true, on arbitrary definition; whether Lorentz transformations are \emph{also} the transformations for \rac, however, is an objective empirical issue. Nevertheless, at this point, Schlick, who had moved to Vienna in the winter term of 1922–1923, seems to consider Reichenbach a reliable conventionalist ally, whose interpretation of relativity was \qt{not only factually unassailable, but also brilliantly expressed}{nicht blos sachlich unanfechtbar, sondern auch glänzend ausgedrückt} \letterp{Schlick}{Reichenbach}{15}{8}{1922}. In the meantime, Reichenbach was already working on expanding the 1921 report into a book. In October, Weyl informed him of a significant flaw in his attempt to derive the Lorentz transformations for the light axioms alone \letterhrp{Weyl}{Reichenbach}{5}{10}{1922}[015-68-02]. Reichenbach managed to somewhat patch the problem in the last draft of the book finished by the of 1923 \citep{Rynasiewicz2005}. 

\citetitle{Reichenbach1924} was published in May–June 1924. In November, \citet{Weyl1924} wrote a scathing review for the \citejournal{Weyl1924}, complaining about the cumbersomeness of Reichenbach's presentation. Weyl's attack—one of the leading relativists of his time—dealt a significant blow to Reichenbach, who, even a decade later, would recall the episode with lingering resentment \letterp{Reichenbach}{Einstein}{12}{4}{1936}[10-107][EA]. Reichenbach’s urge to defend his work is therefore hardly surprising. On \datef{28}{7}{1925}, the journal \citejournal{Reichenbach1925} received a paper in which Reichenbach firmly took a stance. In particular, he complained that Weyl misunderstood his work as a \s{mathematical investigation}, whereas his axiomatization was intended as an \q{epistemological clarification} of the theory of relativity \crc{37}{136}.
 
\section{The Physical Consequences of Reichenbach's Axiomatization}
\label{michelsonmiller}

%As Reichenbach notes, his approach to axiomatization contrasts with the standard form of \scare{deductive axiomatization} associated with figures like David Hilbert. In that tradition, an axiom is typically a general abstract principle, such as a variational principle (see \cite[2]{Reichenbach1924}). Reichenbach instead proposed a \scare{constructive axiomatization}, in which axioms are empirically grounded statements subject to experimental test, distinguishing them from purely conventional definitions.

As Reichenbach explains, his axiomatization differs from the typical \scare{deductive axiomatization} associated with David Hilbert. In that tradition, one sets an abstract general principle as an axiom, such as a variational principle (see \cite[2]{Reichenbach1924}). Reichenbach, in contrast, put forward a \scare{constructive axiomatization}, in which axioms are empirically grounded statements subject to experimental verification, to be distinguished from purely conventional definitions. His axiomatization \qt{has the great advantage for physics that the \emph{implications of each experimental result} can be immediately recognized}{Diese Form der Axiomatik hat fiir die Physik den groBen Vorzug, dab sle die Tragweite jedes experimentellen ResuTtat} \crc{32}{172}. Some statements of the theory are purely definitional and not empirically testable. Among those that are, not all depend on every axiom; thus, one can immediately tell whether an assertion rests on confirmed or uncertain axioms \crc{32}{172}.



As we have seen, light \emph{axioms} (\rom{1}–\rom{5}) capture \emph{facts} established by pre-relativistic optics, along with the limiting nature of the speed of light. The difference between classical and relativistic light geometry, Galilei and Lorentz transformations, is conventional, based on an arbitrary \emph{definition} of simultaneity. In Reichenbach's axiomatization, only the matter axioms (\rom{6}–\rom{10}) contain the empirically testable content of relativity: the claim that the Lorentz transformations are also the transformations for \rac. Axioms \rom{6} and \rom{7} are shared with classical theory; the others are specific to relativity. Axiom \rom{8} encodes the \MME, and Axioms \rom{9} and \rom{10} combined the transverse Doppler effect. While the latter still lacked confirmation, the \MME was widely accepted as a settled matter. However, shortly before Reichenbach’s article appeared, the experimentalists Dayton C.~\textcites{Miller1925}{Miller1925a} started to raise doubt by presenting the results of a series of Michelson-type experiments conducted at Mount Wilson, in Southern California. Reichenbach used the controversy to challenge Weyl and show the value of his axiomatization in addressing the question: \emph{what would happen to relativity if Axiom \rom{8} were empirically refuted?}


\subsection{Michelson's and Miller's Ether-Drift Experiments}

\textcites[53]{Reichenbach1924}[43]{Reichenbach1925}{Reichenbach1926a}[226,299]{Reichenbach1928} repeatedly offer a familiar account of the \MME. Still, it deserves brief attention, since Reichenbach's goal is precisely to \emph{challenge} the standard account. As is well known, the essential aim of the experiment was to determine whether the speed of light varies with direction due to the Earth's motion through the hypothesized ether. In the setup schematically depicted in \cref{mv}, a beam of light is split at point $O$ by a semi-transparent mirror, sending two coherent beams along two perpendicular arms of equal length, each ending at mirrors $S_1$ and $S_2$. The beams reflect back and recombine at $O$, creating an interference pattern—a series of alternating bright and dark fringes. If they return at the same time, their wave crests and troughs align, and the pattern remains stable. If one beam takes slightly longer, the waves arrive out of step, causing a shift in the fringes \rw{325}{195f.}.

According to ether theory, the two beams return to point $O$ simultaneously only if the apparatus is at rest relative to the ether. Since the Earth was assumed to move through the ether, classical optics predicted a \s{deviation}: the beam traveling along the arm $S_2$, aligned with the Earth's motion, would take slightly longer to return, resulting in a shift in the interference fringes. Thus, the experiment was supposed to reveal the Earth's absolute motion through the ether by detecting such a shift. In the 1880s, Albert A.~\citet{Michelson1881}, later with the assistance of Edward W.~Morley (\cite*{Michelson1887}), showed that \qt{in spite of the extreme precision of the measurement, there is no difference in the time to traverse either arm of the apparatus}{trotz allergr\"oßter Meßgenauigkeit zeigte sich keine Differenz f\"{u}r die Durchlaufungszeiten auf beiden Armen der Apparatur} \crw{326}{195}. At the turn of the century, \qt{Morley and Miller [\cite*{Morley1905,Morley1905a}] replicated this negative result in spite of the renewed increase in precision}{Eine 1904/05 von Morley und Miller unternommene Wiederholung hatte denselben negativen Ausfall, trotz abermaliger Steigerung der Meßgenauigkeit} \crw{326}{195\tm}. How can this negative result be explained? The subsequent unfolding, Reichenbach argues, was usually considered uncontroversial:

%We find ourselves in an aether \s{storm}, yet the laws of nature are so precisely arranged that we are entirely unable to perceive it. 

\begin{itemize}
\item At the turn of the century, \q{Lorentz [\citeyear{Lorentz1895}] in Leyden presented his explanation} of this equality \qt{that assumed that all rigid bodies moving in opposition to the ether undergo a contraction}{H. A. Lorentz in Leyden eine Erkl\"arung gab, die eine Verk\"{u}rzung aller starren K\"orper bei Bewegung gegen den Aether annahm} \crw{325}{197}. The arm of the apparatus aligned with the direction of motion is \emph{contracted} by the amount \kappafactor when it moves relative to the ether. The theoretical asymmetry between the ether frame and frames moving with respect to it is hidden from observation by introducing a sort of universal conspiracy of nature.

\item In \citeyear{Einstein1905}, \qt{a more basic explanation was proposed by A.~Einstein in which these contractions occur as a result of a universal principle, the principle of relativity}{Noch tiefer ging die 1905 von A. Einstein aufgestellte Relativit\"atstheorie, welche diese Verk\"{u}rzung als Ausfluß eines universellen Prinzips betrachtet, des Relativit\"atsprinzips;} \crw{326}{197}, and in particular of the relativity of simultaneity. Einstein, on the contrary, declared both arms \emph{equally long} when measured in their own rest frame, but one arm would appear contracted by the factor \kappafactor when observed from a relatively moving system. In this way, the theoretical symmetry between rest and moving systems is reestablished, and the ether could be expunged from the theory.
\end{itemize}
%
Lorentz's and Einstein's theories \emph{agree} on the same empirical fact, stated in axiom \rom{8}: \s{Two space intervals which are equal when measured by rigid rods, are also light-geometrically equal}, meaning they are equal when measured by the round-trip time of a light signal. The null result of the \MME confirmed this equality. 

Further support came recently from Rudolf~\citet{Tomaschek1924}, a critic of relativity, who repeated the interferometer experiment using starlight instead of terrestrial light sources \rc{39}{180}. However, Reichenbach noted that \qt{\textins{r}ecently, doubts have been raised by Dayton C.~Miller, who obtained a positive result on Mount Wilson}{Neuerdings sind dagegen Zweifel erhoben worden durch Dayton C. Miller), der auf dem Mount Wilson einen positiven Effekt erhgl} \crc{39}{180}. If confirmed, Miller’s experiment would imply unequal round-trip times along the two arms ($\overline{OS_1O} \neq \overline{OS_2O}$). The material-rod-based equality of distances would then conflict with the electromagnetic-signal-based one, disproving Axiom \rom{8}. Though Miller’s results were still debated, Reichenbach quickly used the controversy to highlight the value of his axiomatization: \qt{In this context, the axiomatization is proved to be extremely useful because it shows what particular role the Michelson experiment plays in the theory, what follows from it, and what is independent of it}{In diesem Zusammenhang erweist sich die Axiomatik als zuerst n\"{u}tzlich, well sle erkennen lgl]t, welehe Rolls der Michelsonversueh ia der Theorie iiberhaupt spielS, was aus ibm gefolgert wird und was yon ibm unabhgngig ist} \crc{39}{180}.


\begin{figure} \hide{{r}{0.5\textwidth}} \begin{center} \includegraphics[scale=0.2, trim = 0mm 0mm 0mm 0mm, clip]{graph/mvorig} \caption{Reichenbach's diagram of the Michelson-Morley apparatus \citep[from][325]{Reichenbach1926a} \label{mv}} \end{center}
 \end{figure}



\subsection{Two Kinds of Contraction: Einstein Contraction vs.\ Lorentz Contraction} 
\label{twocontractions}

The question, then, is this: what would become of special relativity if the null result of the Michelson–Morley experiment were refuted—that is, if Axiom \rom{8} turned out to be incorrect? Before addressing this question, Reichenbach cautions his readers against uncritically accepting the standard interpretation of the Michelson experiment that we have just presented. Lorentz's dynamical contraction of one arm of the apparatus is usually considered an \emph{ad hoc} hypothesis\footnote{On this issue, see \cite[see also][]{Janssen2002}}. On the contrary, Einstein's hypothesis of the contraction resulting from the relativity of simultaneity appeared to be less contrived since it is a consequence of a universal principle \citep{Schlick1915}. According to Reichenbach, both of these claims are incorrect.

Einstein's theory, like Lorentz's, agrees that the behavior of rigid rods deviates from classical predictions and conforms instead to relativistic light geometry; therefore, the contraction hypothesis is not \emph{ad hoc}. However, contrary to common belief, the relativity of simultaneity is unrelated to the contraction involved in the \MME. The contraction of the apparatus’s arm occurs in the frame relative to which the apparatus is at rest (the Earth), not in the frame relative to which the apparatus is moving (the Sun):

\qt{\textins{W}e should examine a particular error that has crept into the understanding of the theory of relativity. It concerns the problem of Lorentz contraction and thereby leads us to the Michelson experiment. One frequently hears the opinion expressed that in the Lorentzian explanation of the Michelson experiment the contraction of the arms of the apparatus is an \scare{\latin{ad hoc} hypothesis}, whereas Einstein explains it in a most natural way, namely, as a result of the relativization of the concept of simultaneity. But this is false. \myemph{The relativity of simultaneity has nothing to do with length contraction in the Michelson experiment}. That this opinion is false already follows from the fact that the contraction of one of the arms of the apparatus occurs precisely in the system in which the apparatus is at rest}{Man hiirt oft die Meinung uusgesproehen, in der \ls{Lorentz} sehen Erklartmg des Michelsonversuehs sei die Kontraktion des eiaea Appuraturines eine ,,ad hoe ersonnens Hypothese", wahrend sie bei Einstein uuf die nutiirliehste Weise erklart sei, namlieh uls Folge der Relativlerung des Gleiehzeitigkeltsbegriffs. Abet dies ist fulseh. Die Re]ativitat der Gleiehzeitigl~eit hut mit der Stabkontraktion des Miehelsonversuehs niehts zu tun, and die E in s t e i n sehe Theorie gibt hierfiir ebensowenig eine Erklarung wie die Lorentzsehe}[\crc{43}{187\me}] 
%
Einstein contraction has often been called \s{apparent}, for it is not the arm of the interferometer that contracts, but its \s{projection} onto a system at rest\footnote{The \emph{projection} of a moving bar $AB$ in $K$ is the segment $A'B'$ marked on the $x$-axis at the simultaneous positions of $A$ and $B$ in the frame $K$}.  In contrast, Lorentz contraction is usually regarded as \s{real}—a true dynamical phenomenon caused by the action of the ether on molecular forces, similar to the contraction of a metal due to a drop in temperature. Reichenbach, however, explicitly \emph{rejects} this widespread interpretation. In particular, Reichenbach denies that Einstein contraction has anything to do with the null result of the \MME:

\qt{That this opinion is false already follows from the fact that the contraction of one of the arms of the apparatus occurs precisely in the system in which the apparatus is at rest. The \scare{Einstein contraction} only explains that the arm is \myemph{shortened if it is measured from a different system}. But that does not explain the Michelson experiment. \textins{The latter} proves that the rod lying in the direction of motion \myemph{is shorter when measured in the rest system than it should be according to the classical theory}. \textelp The Einsteinian theory, as well as Lorentz's, differs from the classical theory in asserting a measurably different effect on rigid rods that \myemph{has nothing to do with the definition of simultaneity}}{Dal die genunnte Meinung falsch ist, erhellt sehon daraus, dal3 die Kontraktion des einen Apparatarmes gerade fiir das mitbewegte System eintritt, in dem der Apparat ruht. Die \scare{Einsteinsche Kontraktion} wiirde nut erklaren, dab der Arm verkiirzt wird, wenn er yon einem anderen System gemessen wird. Aber das wiirde zur Erklarung des M i c h e 1 s o n versuchs nicht geniigen. Denn dieser bewelst, dai] der in der Lingsrichtung der Bewegung ]iegende Stab, im R.Mtsystem gemessen, kiirzer ist, als er nach der klassisehen Theorie sein sollte. Wiirde es ein ausgezelchnetes Inertialsystem J geben, und hgtte man hierin zwei gleich lange s~arre Stgbe, yon denen der eine sigh naeh der klassischen Theorie, der andere nach der E i n s t e i n schen rlehten wiirde, so waren diese beiden Stabe, in ein Inertialsystem S gebracht, nlcht mehr gleieh lang, weun sie dort in der Lgsrichtung der Bewegung liegen; der Einsteinsehe Stab wgre kiirzer. Und zwar wiirde dleser Unterschied sowohl in S a]s Unterschied der ,Ruhlange", als aueh yon ~edem anderen Inertialsystem aus als Un~ersehied in der ,,Lgnge der beweg en Stgbe" gemessen werden. Es wird also in der Einstelnschen Theorie, genau so wie in der Lorentzschen, ein reel]bar anderes Verhalten der starren Stgbe als in der klassisehen Theorie behanpet, das mit der Glelchzeitigkeitsdefinition gar nichts zu tun hat}[\crc{43--44}{187--188\tm\me}]
%
Einstein contraction depends on the relativity of simultaneity and \qt{is related to the comparison of \emph{different magnitudes} within the \emph{same theory}}{zweier verschiedener Gr\"oßen, die derselben Theorie angeh\"oren} \crc{44}{188}. A comparable case is annual parallax, the apparent displacement of a star observed from opposite sides of Earth's orbit. Lorentz contraction is related to \qt{the behavior of the \emph{same magnitudes} according to \emph{different theories}}{Diese vergleicht die Verhaltensweisen derselben Gr\"oße, wie sie sich nach verschiedenen Theorien ergeben} \crc{45}{188} using the round-trip time of light as \emph{tertium comparationis}\footnote{See also \rzlp{228}{196f}. This distinction is also used by \textcites{Gruenbaum1955}[sec.~12F]{Gruenbaum1963a} without citing Reichenbach. See also \citet{Giannoni1971}}. An analogous case is the difference in gravitational light deflection, where in \gr the deflection is twice as large as it would be according to Newtonian theory.

According to Reichenbach, only Lorentz contraction is at stake in the Michelson experiment, not Einstein contraction: \qt{It just happens that both contractions depend upon the same factor [\kappafactor], and this is probably the reason why they are always confused with one another}{Diese beiden Verkürzungen haben zufällig denselben Faktor und dies ist wohl der Grund, warum man sie immer verwechselt hat} \crc{46}{189}. The fact that Lorentz contraction and Einstein contraction amount to the same factor is, in Reichenbach's assessment, a mathematical fortuity that follows from the linearity of the Lorentz transformations. In order to provide a proof of this claim, Reichenbach resorts to the somewhat idiosyncratic notation introduced in his \citeyear{Reichenbach1924} monograph.

%However, one should not overlook the deep \emph{conceptual} difference hidden behind the \emph{numerical} equality of the two contractions \rc{45f.}{189f.}.


Reichenbach labels $l$ the length of a rod with Lorentz–Einstein behavior, and $L$ the length of a rod with classical behavior. $K$ and $K'$ denote, respectively, the stationary and moving frames. He then uses index notation where the upper index $^{K}$ marks the frame in which the rod is measured, and the lower index $_{K}$ marks the frame in which the rod is at rest. In both the classical and Lorentz–Einstein theories, the lengths of unit rods in the rest frame $K$ are equal, i.e., $\lK{}{} = \LK{}{} = 1$. The difference emerges when one considers the length of the rod in the moving system:

\begin{itemize}
\item \emph{classical theory}: the rod has a unique length regardless of motion:

\begin{equation}\label{eq:CT}
\frac{\LK{}{}}{\LK{'}{'}} = \frac{1}{1} \quad \text{no contraction;}
\end{equation}

\item \emph{Lorentz theory}: the length of the moving rod measured in $K'$ is contracted relative to the classical length in the same frame $K'$:

\begin{equation}\label{eq:LT}
\frac{\lK{'}{'}}{\LK{'}{'}} = \frac{\kappafactor}{1} \quad \text{Lorentz contraction;}
\end{equation}

\item \emph{Einstein theory}: the length of the moving rod measured in $K$ is contracted relative to the length measured in $K'$:

\begin{equation}\label{eq:ET}
\frac{\lK{}{'}}{\lK{'}{'}} = \frac{\kappafactor}{1} \quad \text{Einstein contraction.}
\end{equation}
\end{itemize}
%
Reichenbach aims to prove that the numerical equality of the Lorentz and Einstein contraction factors \(\kappafactor\) follows solely from the linearity of the Lorentz transformations. His proof runs as follows \rzl{230\psq}{199}. According to classical theory, $\LK{}{'} = \LK{}{}$; since $\lK{}{} = \LK{}{}$, then $\LK{}{'} = \lK{}{}$. Substituting in \cref{eq:ET}, it follows that:

\begin{equation}\label{eq:ETCT}
\frac{\lK{}{'}}{\LK{}{'}} = \frac{\kappafactor}{1}
\end{equation}
%
Because of the transformation's linearity, this ratio depends only on the relative velocity $v$ between frames, not on which frame the length is measured. Thus, \cref{eq:ETCT} is the same ratio as \cref{eq:LT}. Since \cref{eq:ETCT} is also the same as \cref{eq:ET}, it follows that \cref{eq:LT} and \cref{eq:ET} express the same ratio, QED. However, Reichenbach insists, there is a deep \emph{conceptual difference} between the two contractions despite their accidental \emph{numerical equality} \rc{45f.}{189f.}.


\subsection{Two Kinds of Explanation: Deviation vs.\  Adjustment}
\label{twoexplanations}
%
As we have seen, according to Reichenbach, both Lorentz's and Einstein's theories assert an agreement between relativistic electromagnetic signal-based geometry and rod-based geometry \cax{36}{176}. The difference between Lorentz's and Einstein's theories must be sought in the way they account for this set of physical facts. According to Reichenbach (1), Lorentz \emph{explains} this empirical fact by claiming that the moving rods, because of their motion through the ether, \emph{must have} a \emph{shorter length} than the rest \scare{ether} rod by a factor \kappafactor. Einstein \emph{stipulates} that the moving and rest rods have the \emph{same length}, despite the fact that the moving rod \emph{as measured in the rest system} is shorter by a factor \kappafactor than the rest rod.

%In particular, the \MME confirms Axiom VI2: \s{Two space intervals which are equal when measured by rigid rods, are also light-geometrically equal} \ra{69}{89}. 

\emph{Both} theories, although empirically equivalent, are, according to Reichenbach, ultimately explanatorily unsatisfying: \begin{inparaenum}[(1)]
\item Lorentz's theory is unsatisfying since it provides a \emph{bad explanation}, as there is no reason to assume that the behavior of rigid rods must be classical. \item Einstein's theory is unsatisfying because he provided \emph{no explanation} at all and simply declared by convention that relativistic rods are rigid.
\end{inparaenum}  The superiority of Einstein's approach is that it frees us from the prejudice that classical light geometry is more \s{natural}, so \rac that departing from the latter must mean \s{distortion}. Einstein shows that we could just as well redefine rigid rods as those rods that conform to relativistic light geometry. However, once this prejudice is overcome, the task is far from complete.

The realization that what counts as a rigid rod is a matter of \emph{convention} should not lead us to sidestep the problem of \emph{explanation}, but rather help reframe it. Unlike Lorentz, one need not explain why rods \emph{disagree} with classical light geometry. But unlike Einstein, one must still explain why they \emph{agree} precisely with relativistic light geometry. Reichenbach suggests that the term \emph{adjustment} \origg{Einstellung}, introduced by his nemesis Weyl \label{adjustment}, aptly captures this peculiar form of \s{explanation}, distinct from the traditional notion of \emph{deflection} \origg{Abweichung} used implicitly by Lorentz. Reichenbach already made this proposal in his 1924 book, though it played a minor role \ra{70--71}{90-91}; in the 1925 paper, the argument took center stage.

As Reichenbach explains, \citet{Weyl1920a} introduced the term to account for the surprising Riemannian behavior of all physical ideal rods of whatever material: they always have the same length when compared side by side after traveling different paths \citep[366]{Reichenbach1922b}. This regularity suggests that rods \scare{adjust} each time to an equilibrium value, rather than \scare{preserve} it\footnote{Following \citet[650]{Weyl1920a}, Reichenbach used this analogy: \begin{inparaenum}[(1)] \item a spinning top maintains its vertical orientation by perseverance \origg{Beharrung} due to angular momentum conservation, but easily loses it if \emph{disturbed}; \item a magnetic needle maintains its northward orientation by adjustment \origg{Einstellung} to the Earth's magnetic field, and thus always \emph{returns} to the it unless \emph{hindered} externally \end{inparaenum}}. The length of a macroscopic rod, for example, may be proportional to some fundamental constant of nature. Thus, in Reichenbach's reading, the \emph{axiom} of Riemannian geometry—the path independence of rod length—is \emph{explained} by a theory of matter implying, say, the fixity of that constant.This explanation does not account for the \emph{deviation} from a standard behavior, but rather for the \emph{convergence} of rods of all kinds to a non-trivial one \citep[see][sec.~424]{Ryckman2005}.


The analogy with \sr appears to be this. As we have seen, the empirical content of Lorentz–Einstein theory \qt{can be formulated as meaning that \emph{light geometry and matter geometry are identical}}{Der Gedanke dei \emph{Einsteins} l\"aßt sich dann dahin formulieren, \emph{daß Lichtgeometrie und K\"orpergeometrie identisch werden}} \cax{11}{14}. It is a striking \emph{coincidence} that any physical ideal rod—whether made of steel, wood\etc —always yields measurements equal in light-geometric terms. As Reichenbach notes, \qt{[l]ight is a much simpler physical object than a material rod, and, when searching for a relation between the two, it should be initially supposed that it would not correspond to so ideal a scheme as the posited matter axioms}{Licht ist ein physikalisch sehr viel elnfacheres Gebilde als ela materieller Stab, und wenn man einen Zusammenhang zwischen beiden sucht~ sollte man zunichst annehmen, dab er nicht einem so idealen Schema entspricht, wie es die K\"orperaxiome behaupten} \crc{47-48}{95}. This coincidence begs for an \emph{explanation}. Yet it is not about explaining the \emph{divergence} from the obvious classical behavior, but rather the \emph{convergence} toward a non-trivial relativistic behavior:

\qt{The word adjustment, first used in this way by Weyl, is a very good characterization of the problem. \textelp All metrical relations between material objects, including the observed fact of the Michelson experiment, must therefore be explained in terms of the particular way in which rigid rods adjust to the movement of light. Of course, the answer can only arise from a detailed theory of matter about which we have not the least idea \textelp. The word \scare{adjustment} here thus only means a problem without providing an answer; the relevant fact is strictly formulated in the matter axioms without using the word \scare{adjustment}. Once we have this theory of matter, we can explain the metrical behavior of material objects; but at present the explanation from Einstein's theory is as poor as Lorentz's or the classical terminology}{Das Wort Einstellung, yon Weyl zum erstenmal in diesem Zusammenhang gebraucht, charakteris;ert das Problem sehr gut \textelp Alle metrischen Beziehungen zwisehen materiellen Gebilden mtissen so erklart werden, also anch der im Miehelsonversuch beobaehtete Tatbestand, wonach sich starre Stabe in bestimmter Weise auf die Lichtbewegung einstellen. Die Antwort kann natiirlich nur eine ausge~iihrte Theorie der Materle geben, yon der wir noch nicht die leiseste Vorstellung besitzen; Das Wort Einstellung deutet bier also nur auf eine Aufgabe bin, ohne selbst eine Antwort zu sein; der vorliegende Tatbestand ist ohne Benutzung des Wortes Einstellung in den K\"orperaxiomen streng formuliert. Wenn wlr dlese Theorie der Materie einmal besitzen, k5nnen wir das metrische Verhalten der materiellen GebJlde erkl~ren; vorerst aber kann yon einer Erklarung in der Einsteinschen Theorie so wenig die Rede sein wie in der Lorentzschen oder der klassisehen}[\crc{46--47}{191}]
%
According to Reichenbach, the difference between Lorentz’s and Einstein’s theories lies not in their differing empirical content, but in their different \emph{explanatory strategies}. 

Lorentz's theory assumes the classical behavior of rods as \s{natural}, so that any deviation from that standard \emph{requires} an explanation. Einstein's theory \emph{dispenses} with any explanation by defining the behavior of rods as \s{natural} if it agrees with the \MME: \qt{The superiority of Einstein's theory lies in the recognition of the epistemological legitimacy of this procedure}{in dem Bewußtsein des erkenntnistheoretischen Rechtes hierzu liegt ihre \"{u}berlegenheit.} \crz{233}{202}. Certainly, Einstein's conventionalism removes the prejudice that the classical behavior of \rac is \apr correct. However, Einstein's agnosticism is unsatisfying, since it does not explain why \rac happen to behave relativistically. Without a suitable \emph{theory of matter} describing those physical systems we happen to use as rods and clocks, \q{Einstein's theory provides just as little an explanation \origins{Erkl\"arung}} of the metrical behavior of material objects \qt{as Lorentz's}{Einsteinsehe Theorie gibt hierfiir ebensowenig eine Erklarung wie die Lorentzsehe} \crc{43}{87}.

Once he has clarified the meaning of explanation in the relativistic context, Reichenbach is in a position to explore the implications of a possible positive result from a Michelson-type experiment. As previously noted, this was far from a purely theoretical concern at the time. \citets{Miller1925} new findings quickly ignited significant discussion within the physics community \citep{Lalli2012-04}, and Reichenbach became the first \scare{philosopher} to attempt to engage in the debate:

\qt{Now we can also address the question what would change in the theory of relativity if Miller's experiment were held to prove that the hitherto negative result of the Michelson experiment is in principle wrong. \myemph{Nothing would change} in Einstein's theory of time as it has nothing to do with the Michelson experiment. Also nothing would change with the light geometry; it remains in any case a possible definition for the space-time metric and probably a much better and more accurate one than the geometry of rigid rods and natural clocks. \myemph{But what would change is our knowledge about the adjustments of material things to the light geometry}. With respect to the matter axioms, as far as they differ from the classical theory, the Michelson experiment is the only one that has been confirmed. If this should be refuted, one has to develop a more complex view of the relationship between material objects and the light geometry}{Jetzt kSnnen wir aueh die Frage beautworten, was sigh in der Relativit\"ats\"aheorie andern wiirde, wenn die Versuehe ~[ill e r s als Beweis angesehen werden mii~ten, da~ der bisherige negative Ausfall des ~[iehelsonversuehs nicht prinzipieH Iestgehalten werden darf. Nicht andern wfirde sieh die Einsteinsche Zeitlehre, sie hat mlt dem Miehelsonversuch gar nichts zu tun. Ni e h t ~ndern wiirde sieh auch die Lichtgeometrie; sle bleibt auf ieden Fall eine m~gllche Definition der raumzeitlichen Metrlk, und wahrseheinllch eine viel bessere und genauere als die Geometrie der starren St~be und natiirllchen Uhren. ~ndern aber wiirde sich unser Wissen fiber die Einstellung tier materiellen Gebilde auf die Lichtgeometrie. Von den K~rperaxiomen, sower sie sich yon denen der klasslschen Theorie unterseheiden, ist tier Michelsonversuch bisher als einziges best~tigt. F~llt dieses, so wird man sich ~ber den Zusammenhang der materiellen Gebilde mit der Liehtgeometrie eine verwlekeltere Auffassung bilden mfissen}[\crc{47}{192\me}]
%
As we have seen, if Miller's results were not spurious, only this axiom \rom{8} would need revision. The principle of the \emph{constancy} of the speed of light could still hold; by changing the conventional definition of simultaneity, one could still claim that light propagates as a spherical wave in any uniformly moving frame. The assertion of light as a \emph{limiting} velocity also remains valid, as confirmed by measurements on fast electrons ($\beta$ rays), whose kinetic energy approaches infinity as their speed nears that of light \ra{72}{92}. What would be refuted is the claim that light always has a \emph{numerical value} $c$ when measured by \rac. A refutation of \MME would mean that rods do not adjust to the relativistic light geometry as Einstein expected. This would simply show that \qt{rigid rods do not after all possess the preferred properties that Einstein still attributes to them}{daß die starren K\"orper doch nicht jene einfachen Vorzugseigenschaften besitzen, die Einstein ihnen immer loch l\"aßt} \crc{328}{203}.


\section{Schlick–Einstein Correspondence on Reichenbach's Axiomatization} 
\label{schlickeinstein} 

It is worth noting that Einstein drew very different conclusions from Miller's experiment. Einstein's view at the time is clearly conveyed in a letter sent to Robert A.~Millikan, the \latin{de facto} head of Caltech: if Miller's result were to be confirmed, Einstein wrote, then \qt{the whole theory of relativity would \myemph{collapse like a house of cards}}{f\"allt die ganze Relativit\"atstheorie zusammen wie ein Kartenhaus} \lettercpaep{Einstein}{Millikan}{13}{7}{1925}[15][20\me]. The analogy seems to allude to the \s{theoretical rigidity} that Einstein demanded from a good theory: if any one of its conclusions proves to be false, the theory must be completely abandoned, since it is constructed in such a way that any modification would lead to its complete breakdown \citep[see][]{Einstein1919}. A few days later, an equally resolute statement by Einstein was published on \datem{8}{8}{1925} in \jt{Science News-Letter} \citep{Einstein1925g}. 

Reichenbach's markedly different attitude toward Miller's results appeared puzzling. By the end of the year, Schlick wrote to Einstein, expressing his bewilderment about Reichenbach's \citeyear{Reichenbach1925} paper, which \qt{quite clearly shows the limit of the axiomatic method}{weil sie ziemlich deutlich die Grenzen der axiomatischen Methode zu zeigen scheint} \lettercpaep{Schlick}{Einstein}{27}{12}{1925}[15][140]. In particular, Schlick was disconcerted by Reichenbach's claim that the Lorentz contraction is not \latin{ad hoc}. After all, this had been Schlick's \s{conventionalist} reading of \sr\ for at least a decade \citepp{Schlick1915}[60\psq]{Schlick1923}. He likely saw Reichenbach's remark as a barely concealed jab. In Schlick’s conventionalist reading, Lorentz and Einstein theories are empirically equivalent. The choice between them is thus one of simplicity, not truth \citep{Schlick1915}. Einstein's theory is preferable precisely because it avoids \latin{ad hoc} assumptions.

Initially, Schlick and his circle saw Reichenbach’s axiomatization as a refined version of this view \citep{Zilsel1925a}. However, Schlick now admitted to Einstein---clearly with some disappointment---that Reichenbach was pursuing a fundamentally different line of thought:

\qt{The remarks on p. 43[/187] \textelp{} seem to me to show only that his axiomatic system cannot find any difference between special relativity and Lorentz’s theory (with the contraction hypothesis), which seems self-evident to me, since the equations are the same in both. The real difference between the two theories—which is a philosophical one and thus not accessible through the purely logical method of axiomatization—is, I believe, well captured by the mode of speech that Reichenbach rejects, namely, that Lorentz's hypothesis was an \latin{ad hoc} invention. For even if, logically speaking, special relativity must make just as many basic assumptions as Lorentz’s theory, in the former they naturally fit into the framework of the relativity principle, and the contraction hypothesis is not, psychologically speaking, an \latin{ad hoc} invention—whereas in Lorentz–Fitzgerald’s version, it appears as an \latin{ad hoc} addition}{Die Ausf\"{u}hrungen S.~43 \textelp{} zeigen aber m.~E.\ nur, dass die seine Axiomatik zwischen der spez. Reltheorie und der Lorentzschen Theorie (mit der Kontraktionshypothese) \"{u}berhaupt keinen Unterschied finden kann, was mir selbstverst\"andlich erscheint, da die Gleichungen ja in beiden dieselben sind. Der wirkliche Unterschied zwischen beiden Theorien, der eben ein philosophischer und auf dem rein logischen Wege der Axiomatik nicht fassbar ist, wird wohl gerade durch die von Reichenbach verworfene Sprechweise, es handle sich bei Lorentz um eine \latin{ad hoc} ersonnene Hypothese, recht treffend angedeutet. Denn wenn auch, logisch gesprochen, die spez. Rel-theorie ebenso viele Grundannahmen machen muss, wie die Lorentzsche, so f\"{u}gen sie sich doch bei der ersteren ganz von selbst in den Rahmen des Relativit\"atsgedankens ein und die Kontraktionshypothese ist psychologisch tats\"achlich nicht \latin{ad hoc} ersonnen, w\"ahrend sie bei Lorentz-Fitzgerald als ein ad hoc angef\"{u}gtes St\"{u}ck auftritt}[\lettercpaep{Schlick}{Einstein}{27}{12}{1925}[15][140]]
%
Schlick’s criticism misconstrues the issue at stake. Reichenbach’s axiomatization does not fail to capture the epistemological difference between the two theories; rather, it situates it at a different juncture. For Schlick, the choice between Lorentz and Einstein theory is ultimately arbitrary—an appeal to the \emph{simplest convention}; for Reichenbach, it is guided by an inference to the \emph{best explanation}. While Lorentz’s theory offers a \emph{bad} explanation, Einstein’s theory clears the way for a \emph{good} one.

Still, Schlick was right in noting that the contrast between their interpretations becomes especially clear in Reichenbach’s response to Miller’s experiment. If Miller’s findings had been confirmed, Schlick contended, the presumed natural \s{conspiracy} concealing the ether would have been exposed, and we would have had to revert to the ether theory—essentially Lorentz’s framework without the contraction hypothesis. Reichenbach, however, denied the necessity of such a reversal. In this sense, Schlick concluded, Reichenbach’s axiomatic approach also lacks any \s{physical consequences}—in direct contradiction with the title of his paper:

\qt{The final remarks of the essay—on the possible interpretation of Miller’s experiments—also seem to me to miss \label{core}\myemph{the philosophical core of the issue}. If those experiments had indeed proven (which is certainly not the case) that a particular direction (that of the \scare{aether wind}) were privileged, then relativistic physics would surely be abandoned. And even if it were possible to preserve relativity by assuming certain \scare{matter axioms}, this path would not be taken. Yet the axiomatic approach remains indifferent to this. It seems to me, therefore, that in a strict sense, one cannot really speak of physical consequences arising from axiomatization. These questions still strike me as philosophically important, and I would be most grateful if you could let me know with a line whether I am right}{Auch die letzten Ausf\"{u}hrungen des Aufsatzes---\"{u}ber die m\"ogliche Interpretation der Millerschen Versuche---scheinen mir den philosophischen Kern der Sache nicht zu treffen. Wenn durch jene Versuche wirklich bewiesen w\"are (was ja gewiss nicht der Fall ist), dass eine bestimmte Richtung (die des \scare{Aetherwindes}) ausgezeichnet w\"are, so w\"{u}rde man gewiss die relativistische Physik aufgeben, und wenn es auch m\"oglich sein sollte, die Relativit\"at durch Annahme bestimmte \scare{K\"orperaxiome} aufrecht zu erhalten, so w\"{u}rde man doch diesen Weg nicht einschlagen. Aber hiergegen verh\"alt sich eben die axiomatische Betrachtung indifferent. Es scheint mir dabei, dass man daher in ganz strengem Sinne von physikalischer Konsequenzen der Axiomatik eigentlich doch nicht sprechen kann. Die Fragen scheinen mir philosophisch doch wichtig, und ich w\"are Ihnen von ganzen Herzen dankbar, wenn Sie mit einer Zeile mir sagen wollten, ob ich recht habe}[\lettercpaep{Schlick}{Einstein}{27}{12}{1925}[15][140\me]]
%
A reply by Einstein is non-extant. In earlier correspondence, he had acknowledged that labeling Lorentz contraction as \emph{ad hoc} was misleading, in a way that resonates with Reichenbach’s stance \lettercpaep{Einstein}{Lorentz}{23}{1}{1915}[8][47]. Still, with respect to Miller’s experiment, Einstein’s stance aligned more closely with Schlick's than with Reichenbach's. On \datef{19}{1}{1926}, a concise yet unambiguous statement by Einstein appeared in what was then Germany’s most influential newspaper, the \citejournal{Einstein1926a}: \q{If the results of Miller's experiments should indeed be confirmed, the relativity theory \myemph{could not be upheld}} \citep[emphasis mine]{Einstein1926a}.

\citet{Reichenbach1926a} submitted a popular article on Miller’s experiment, which was published in the weekly magazine \citejournal{Reichenbach1926a} on \datem{24}{4}{1926}. By then, he has become aware of what \qt{Einstein himself has recently said in the newspapers,}{Einstein selbst hat kürzlich in den Tageszeitungen ausgesprochen}; nevertheless, he saw no reason to revise his \qt{less radical opinion}{eine weniger radikale Ansicht entwickelt} \crw{327}{202}, namely that \qt{\myemph{Miller’s result in no way affects the philosophical consequences of the theory of relativity}}{Auf keinen Fall werden jedoch die philosophischen Konsequenzen der Relativitätstheorie von den Millerschen Versuchen betroffen} \crw{328}{203\me}. For Reichenbach, the outcome would merely require a change in our understanding of the physical processes that govern rods and clocks. Special relativity would turn out to be \qt{a first-order approximation in the same way that the ideal gas law cannot be maintained if the accuracy is increased}{nur die Geltung einer ersten Ann\"aherung, etwa wie die idealen Gasgesetze, die sich bei gr\"oßerer Genauigkeit auch nicht aufrecht halten lassen} \crc{48}{192}. \label{mc}

%\citep[308]{Hentschel1982}.

Reichenbach expresses some doubts about the correctness of Miller's experiment \citep[326\psq]{Reichenbach1926a}. However, the philosophical point lay elsewhere: \qt{\lse{What then does the theory of relativity have to infer from Miller's experiment?}}{\s{Was hat nun die Relativit\"atstheorie aus dem Millerschen Versuch zu schließen?}} \crw{327}{202}. On this matter, Reichenbach does not hesitate to voice a view that sharply diverged from that of Einstein. Reichenbach concurs that \qt{\textins{t}he Michelson experiment, of course, played a crucial role in the \myemph{historical development} of the theory}{Zwar hat der Michelson-Versuch in der historischen Entwicklung der Theorie eine entscheidende Rolle gespielt} \crw{327}{202\me}; however, according to Reichenbach, \qt{it does not occupy this same significant place in the relativistic theory's \myemph{logical structure}}{Aber im logischen System der Relativit\"atstheorie kommt ihm nicht dieselbe f\"{u}hrende Stellung zu} \crw{327}{202\me}. The logical structure of the theory was, of course, captured by his own axiomatic formulation:

\qt{Under the ten axioms of the theory of relativity as I have laid them out, \ie, its ten most basic empirical propositions, there is only one that entails the Michelson result; it is only this axiom then that is thereby threatened. The principle of the constancy of the speed of light could be maintained in a more limited form even if the Michelson experiment's negative result were overturned. One could construct a \scare{light geometry} using light signals but employing no rigid rods to maintain a metrical understanding of the world and allow the previous formulation of all physical laws. From this perspective, the Michelson experiment serves only as a bridge between the light geometry and the geometry of rigid rods. Should this connection be lost, this would only mean that rigid rods do not after all possess the preferred properties that Einstein still attributes to them. This would not mean a return to the old aether theory, but rather a step towards the renunciation of a preferred system of measurement in nature}{Unter den vom Verfasser aufgestellten zehn Axiomen der Relativit\"atstheorie, d. h. ihren zehn obersten Erfahrungss\"atzen, enth\"alt nur ein einziges die Behauptung des Michelson-Versuches; nur dieses Axiom w\"are also ersch\"uttert. Das Prinzip der Konstanz der Lichtgeschwindigkeit l\"aßt sich in eingeschr\"ankter Form auch noch festhalten, wenn der Michelson-Versuch nicht negativ ausf\"allt. Man kann, indem man ohne Benutzung starrer Maßst\"abe nur mit Lichtstrahlen arbeitet, eine "L chtgeometrie" konstruieren, di,e eine metrische Erfassung der Welt leistet und die Formulierung aller physi kalis ehen Gesetze bereits gestattet. Von diesem Gesichtspunkt gesehen kommt dem MichelsonVersuch nur die Rolle eines Verbindungsgliedes zwischen Lichtgeometrie und Geometrie der star}[\crw{327}{203}]
%
Of course, Reichenbach did not dispute that Miller’s result would impact the \s{physical theory of relativity}{}—the empirical assertion that ideal \rac follow relativistic light geometry could no longer be upheld. His claim, rather, is that it would not undermine the \s{philosophical theory of relativity}, that is, Einstein’s insight that simultaneity is not an empirical fact but a matter of definition. For Reichenbach, Einstein’s realization that time is not absolute \q{is independent of specific, physical observations} \crw{203}{328}. While this insight arose \q{from a particular physical theory}, namely \sr, it has \q{given rise to philosophical insights which no longer belong to the realm of physics but rather to the philosophy of nature} \crw{204}{328}.

\section{Dynamical vs.\  Geometrical Explanation in the \citetitle{Reichenbach1928}}
\label{dynamicalfeometrical}
%
Whether Reichenbach was aware of Schlick’s criticism is unclear. If he did, he was not swayed. Indeed, Reichenbach’s line of reasoning reappears in Reichenbach's monograph \citetitle{Reichenbach1928} \citep{Reichenbach1928}, which had already been completed by the end of 1926 (\letter{Reichenbach}{Schlick}{6}{12}{1926}[][SN]). By the time the book's galley proofs were finalized in late 1927, the relevance of Miller’s experiment was already waning, particularly in Germany. Reichenbach could now state that the \qt{Michelson experiment has been confirmed to a very high degree}{Da nun der Michelsonversuch i n hohem Maße gesichert ist,} and regarded \qt{this matter closed}{diese Angelegenheit \textelp{} erledigt} \rzl{225}{195}. Nonetheless, Reichenbach still found it necessary to address the \qt{erroneous interpretations in the usual discussions on relativity}{ein eigentümlicher Irrtum auf seiten der Relativitätstheoretiker angeschlossen hätte} that had surfaced in the discussion over Miller’s result \crz{225}{195}.


%Einstein had expressed a similar view the year before \citep{Einstein1927d}. 

\begin{figure}
\begin{center}
\includegraphics[scale=0.6, trim=0mm 0mm 0mm 0mm, clip]{graph/mr}
\caption{\Mink diagram from \cite[215]{Reichenbach1928}}
\label{mr}
\end{center}
\end{figure}



In \S31, Reichenbach restates the position he had advanced in his \citeyear{Reichenbach1925} paper, with little or no revision. However, he introduces a novel element presenting the difference between Lorentz and Einstein contraction in geometrical terms—that is, by resorting to the \Mink diagram in \cref{mr}. The name of \Mink is, surprisingly, never mentioned in his \citeyear{Reichenbach1924} axiomatization. This is ultimately not simply an oversight. One might perhaps expect that Reichenbach saw in \Mink's work a \s{geometrical explanation} replacing the \s{dynamical explanation} that Einstein had failed to provide. However, this is not the case. Reichenbach seems to believe that \Mink's geometrical presentation of Einstein's kinematics does not provide any explanatory contribution. In Reichenbach's view, \Mink's \qt{geometrical interpretation of the Lorentz transformation}{geometrische Deu-tung der Lorentz-Transformation} \rzl{206}{177} was nothing but a \scare{\emph{graphical representation}} of the agreement between light and matter geometry: the Lorentz transformation, which differs from the Galilean transformation in terms of light geometry only by definition, is also the transformation for \rac \rzl{175}{205}.

Reichenbach borrows the expression \scare{\emph{graphical representation}} \origg{graphische Darstellung} from Arthur Stanley \citet[295]{Eddington1925a}. However, in section \S15 of the book, he takes care to provide a more precise definition. Reichenbach characterizes \scare{graphical representations} as structural analogies between different physical systems: The same \emph{mathematical system} $A$ can be \emph{physically realized} by different systems $a, b, c, \ldots$; thus, one can use, say, the system $b$ to \emph{graphically represent} the system $a$ \rzl{125f.}{103}. In other terms, while the physical realization is a \emph{vertical coordination} of the same mathematical system with different physical systems, the \emph{graphical representation} is a horizontal coordination among different physical systems that realize the same mathematical structure.



Reichenbach gives several examples \rzl{126}{103}, but for the sake of brevity, let's apply this definition to our case. \Mink's achievement was to have presented analytically the dependence of the measurement of space upon simultaneity, in the combining of space and time into a unique mathematical structure $A$, a four-dimensional manifold \xxxx endowed with an indefinite metric: $ds^{2} = dx_{1}^{2} + dx_{2}^{2} + dx_{3}^{2} - dx_{4}^{2}$. The Lorentz transformations can be derived from the invariance of this expression. This \emph{mathematical structure} $A$ can be \emph{physically realized} in different ways: ($a$) \rac, where $ds > 0$ is realized by rods, $ds < 0$ by clocks, and $ds = 0$ by light rays; (b) by lines on a Minkowski diagram as depicted in \cref{mr}, where $ds > 0$ is realized by the horizontal axis $OS$, $ds < 0$ by the vertical axis $OQ$, and $ds = 0$ by a dotted line tilted at $45^{\circ}$. Since (a) and (b) are both realizations of $A$, (a) can be \emph{graphically represented} by (b).

Of course, the mathematical content of \sr is not altered simply because we have visualized the relativistic behavior of \rac using a \Mink diagram: \qt{we only give a graphical representation, which means that the logical structure \origins{Beziehungsgef\"{u}ge}}{so wird sie nur graphisch dargestellt; es wird damit behauptet, dass das Beziehungsgef\"{u}ge, welches r\"aumliche St\"abe von der Art des vorangehenden \S~28 enthalten, zugleich auch; von der Raum-Zeit-Mangfaltigkeit realisiert wird.} \rzl{220}{190} exhibited by rods, clocks, and light rays is also presented through the relations among the lines in the \Mink diagram. Reichenbach concludes with a passing, yet significant remark: if, after \Mink's work, we \qt{speak of a \myemph{geometrization} of physical events, this phrase should not be understood in some mysterious sense; it refers to the identity of types of \emph{structure} and not to the \emph{identity of the coordinated physical elements}}{Wenn man von einer Geometrisierung des Weltgeschehens gesprochen hat, so darf dies auf keinen Fall in irgendeinem geheimnisvollen Sinn aufgefaßt werden; es besagt nur die Identitat von Strukturtypen, nicht der zugeordneten dinglichen Elemente.} \rzl{220}{190\me}.  \Mink's approach serves as a \emph{geometrical illustration} of the relativistic behavior of \rac, not as a \emph{geometrical explanation}.

An example of such \s{geometrical rapresentation} relevant for this paper is, of course, Reichenbach's use of a \Mink diagram (\cref{mr}) to expound the difference between Lorentz and Einstein contraction:

\begin{itemize}
\item \emph{Lorentz contraction}: the rest-length $OS'=\lK{\prime}{\prime}$ of the moving rod as measured in the moving frame $K'$ is shorter than the rest-length $OS^{\prime}_2=\LK{\prime}{\prime}$ of the classical theory measured in the same frame $K'$. This is an \emph{objective difference} on which both Einstein and Lorentz agree. It is represented graphically by the fact that, in the relativistic theory, the world strip of a moving rod is bounded on the right by the solid line parallel to $OQ^{\prime}$, tangent to the hyperbola at $S^{\prime}$; in the classical theory, by the dotted line passing through $S$. That is, the classical strip is wider than the relativistic strip.

\item \emph{Einstein contraction}: the length of the moving rod $OS'$, as measured in the rest frame ($OS_{1} = \lK{}{\prime}$), is shorter than its rest length in the co-moving frame ($OS' = \lK{\prime}{\prime}$). The length of the rest rod $OS$, as measured in the moving frame ($OS'_{3} = \lK{\prime}{}$), is shorter than its rest length in the rest frame ($OS = \lK{}{}$). This \emph{perspectival difference} depends on the chosen frame. Graphically, it is represented by the fact that the projection $OS_{1}$ is shorter than $OS'$, and the projection $OS^{\prime}_{3}$ shorter than $OS$. Indeed, in $K'$, the points $O$, $S^{\prime}$, and $S^{\prime}_{3}$ are simultaneous; $O$, $S_{1}$, and $S$ are simultaneous in $K$\footnoteh{In general, the same rod $OS^{\prime}$ projects differently onto tilted space axes, producing different contractions in different frames}.
\end{itemize}
%
Lorentz's philosophical ingenuousness lay in considering the classical behavior \s{natural}, namely $OS'_{2} = OS$. Thus, from the fact that $OS' < OS'_{2}$ (Lorentz contraction) when measured by the round-trip time of light in $K'$, he concluded that $OS' < OS$ \emph{must} hold in $K$\footnote{\citet{Giannoni1971} calls it the Lorentz-Fitzgerald contraction}. Einstein's philosophical insight was to assert that one \emph{may} stipulate $OS' = OS$ to be the \s{natural} behavior, provided one concedes that $OS_{1} < OS$ in $K$ and $OS'_{3} < OS'$ in $K'$ (Einstein contraction). However, Reichenbach insists, it is often forgotten that the Lorentz contraction also holds in Einstein's theory since $OS' < OS'_{2}$ in $K'$ \rzl{229}{199}.

The two contractions are, nevertheless, independent \rzl{229\psq}{199f.}.  Reichenbach shows that one can alos construct hypothetical cases in which there is no Lorentz contraction ($OS'_{2} = OS$), but Einstein contraction occurs by a suitable definition of simultaneity: \begin{inparaenum}[(a)] \item $\epsilon = \sfrac{1}{2}$ in $K'$: $OS$ appears contracted with respect to $OS^{\prime}_{2}$, if simultaneously measured in $K'$, whose line of simultaneity is tilted: $OS'_{3} < OS'_{2}$;\footnote{By the square of the standard contraction factor, $\LK{'}{} = 1 - \tfrac{v^2}{c^2} \LK{'}{'}$, since the contraction is applied twice in the same direction and at the same speed: $OS'_{3}<OS'<OS'_{2}$} \item $\epsilon > \sfrac{1}{2}$ in $K$: $OS$ is contracted, if simultaneously measured in  $K$, $OS_{\epsilon}<OS$, since the line of simultaneity would be tilted in $K$:\end{inparaenum}

\qt{The example \textelp{} makes particularly clear that the \myemph{Einstein contraction} is a \myemph{metrogenic} phenomenon. In the geometrical representation this means that we may choose as the length of the rod differently directed sections through the world-strip of the rod. On the other hand, the geometrical representation of [\cref{mr}] shows very clearly that through the difference in the width of the strip, the \textit{Lorentz contraction} indicates a difference in the \myemph{actual behavior} of the rod. These considerations also explain how it is possible to compare rods $l$ and $L$, although only one of them is physically realized. $OS$ is the same in both theories; the classical theory claims that the right-hand boundary of the strip parallel to $OQ'$ must be drawn through $S$, whereas the new theory places the boundary along the tangent to the hyperbola which passes through $S'$}{Gerade dieses Beispiel \textelp macht es besonders deutlich, daß die EinsteinVerk\"{u}rzung eine metrogene Erscheinung ist; es kommt in der geometrischen Darstellung darauf' hinaus, daß man als L\"ange des Stabes verschieden gerichtete Schnitte durch den Weltstreifen des Stabes ausw\"ahlt. Andrerseits zeigt die geometrische Darstellung der Fig. 32 (S. 215) deutlieh, daß die LorentzVerk\"{u}rzung mit dem Unterschied der Streifenbreiten einen Unterschied des realen Verhaltens betrifft. Auch erkennt man hier, wie es \"{u}berhaupt m?glich ist, die St\"abe $l$ und $L$ zu vergleichen, obgleich nur der eine von ihnen realisiert ist: die Strecke OS ist f\"{u}r beide Theorien dieselbe; die alte Theorie behauptet, daß die rechte, zu $OQ'$ parallele Begrenzung des Streifens durch S gezogen werden muß, w\"ahrend die neue Theorie behauptet, daß diese Begrenzung als Tangente an die durch $S$ gehende Hyperbel gezogen werden muß}[\rzl{232}{200\me}]
%
The term \s{contraction} is, in both cases, somewhat misleading: Einstein's \s{contraction} does not imply any physical change, whereas Lorentz's \s{contraction} is only a counterfactual change \rzl{227}{196f.}.

It is precisely this unfortunate terminological choice that has led to \q{a mistaken application of the principle of causality} \rzl{232}{201}. According to Reichenbach, the search for a causal explanation within \sr is not only legitimate, but necessary; only it has \q{to be posed in a different form} \rzl{232}{201}. Reichenbach once more invokes Weyl's expression \emph{adjustment} to describe the nature of the sought-for explanation (see above, \cref{adjustment}). However, in this way, the task \origg{Aufgabe} is merely stated, without offering a solution: \qt{The answer can of course be given only by a \myemph{detailed \origins{ausgef\"{u}hrte} theory of matter}}{Die Antwort kann nat\"{u}rlich nur eine ausgef\"{u}hrte Theorie der Materie geben, von der wir noch nicht die leiseste Vorstellung besitzen} \crz{233}{201\me}. If this theory of matter were \q{\emph{exactly formulated}, we would be able to \myemph{explain} the metrical behavior of physical structures}; because no such theory of matter is presently at hand, \q{an explanation by Einstein's theory is as little possible as we can speak of an explanation by Lorentz's theory or by the classical theory} \rzl{233}{201\me}.



%\section{Dynamica vs.\  Geometrical Explanation in General Relativity}

%% !TEX root = reichenbach_explanation.tex

\subsection{Reichenbach Dynamical Interpretation of General Relativity}

That Reichenabch was ultaly, amrginal. Hwove,r that consdierht ethe case. To this purpuse some analitycal elemth should be tinodee. It is important to emphasize that this is not a marginal aspect of Reichenbach's philosophy. According to Reichenbach the very same problem emerges in \gr when the non-Euclidean nature of the continuum is taken into account. In this case, too, he resorts to the expression \scare{adjustment} and he refers his readers to the very same \S31 of the book. The insistence on the problem of explanation must bot be taken as. Indeed, it becomes the same role of \gr. 


As we have seen, because, Einstein redefined the length of simultaneity that it became advantageous to combine space-time into a single geometrical structure, that is a four-dimensional manifold, that \Mink called the \scare{World} or \st (\S24). This simply means that it takes four numbers, the so-called \emph{coordinates}, $x_\nu$ (where $\nu=1,2,3,4$) to identify a \wpo, namely three numbers $x_1,x_2,x_3$ for its spatial location and one for time $x_4$. The set of points whose coordinates are defined by $x_\nu(s)$ where $s$ is an arbitrary parameter is called a \wl{}. At this stage, coordinates $x_\nu$ are nothing but identification numbers, that is, in Reichenbach's parlance, they have only a \scare{topological} function; they determine the order of the \emph{between} relation. In order to define distances between points, one needs to introduce an expression that assigns a number $ds$ to the coordinate differences $dx_\nu$ between two close \wpo{}s, the so-called \scare{fundamental metrical form}. In the case of \Mink \spti, it is always possible to choose the coordinate numbers \xn so that any distance satisfies the relation.
%
That the behavior of small \rac and satisfy the following relations:
%
\begin{equation}\label{eq:mink}
\diff s^{2} = \diff x_{1}^{2}+\diff x_{2}^{2}+\diff x_{3}^{2}-\diff x_{4}^{2}\,.
\end{equation}
%%
This metrical (that is \scare{measurement}) formula allows to calculate the distance $ds$ between two nearby points from their coordinate differences $dx_\nu$. A metric that has positive as well as negative signs is called \scare{indefinite} \rzlp{**}{188--189}. The physical realization of the negative $\diff s^2$ is a physical object that satisfies the relations of congruence defined by the hyperbolas of quadrants \rom{1} and \rom{2}. The realization of the positive $\diff s^2$ is a physical object that satisfies the relations of congruence defined by the hyperbolas of quadrants \rom{3} and \rom{4}. The first is called a time-like interval $\diff s^2 = -1$ and is realized by the proper time of a clock. $\diff s^2 = 1$ is the space-like interval and is realized by the proper length of a rod. Light rays realize $\diff s^2 = 0$, the limiting velocity, which cannot be reached but only approached arbitrarily closely. Otherwise rods and clocks behave by following the hyperbolic contour lines.

As is well-konw pecial realtivity, that Lorentsz recatug coordiantes, invariance of \cref{eq:mink} that  itself in the existence of a group of transformations which leave invariant the four-dimensional distance or interval between two points. In \Mink, space-time one can of also introduce a non-rectangular coordinate system, for example polar, cylindrical coordinates\etc, or arbitrary curvilinear coordinates. If the topological relations remain unchanged, the information about the reciprocal distances among any pair of points is lost. In order to recover the distance between two points in the new coordinate system, one needs to know the \scare{generalized fundamental metrical form}. In Einstein's notation (where summation over repeated indices is implied), it reads:

\begin{equation}\label{eq:lineelement}
ds^2=\gmn dx_\mu dx_\nu\,.
\end{equation}
%
In Reichenbach's interpretation \cref{eq:lineelement} leaves the \scare{topological} function of numbering to the coordinate system, and assigns the \scare{metrical} function of measuring spatial lengths and time intervals to the metrical coefficients \gmn \rzlp{**}{**}. The \gmn are numbers by which coordinate differences $dx_\nu$ have to be multiplied so that, over larger region of \st, identical rods measure $ds=+ 1$ in every position and in every orientation, identical clocks $ds=-1$, and light rays $ds =0$. For example, a unit rod laid along the $x_1$-axis ($dx_2=0,dx_3=0,dx_4=0$) will measure $-1=\sqrt{g_{11}}dx_1$; a unit clock at rest will measure $1=\sqrt{g_{44}}dx_4$. By transporting our \rac, we can determine $g_{11}= ds^2/dx_1^2$, $g_{44}= ds^2/dx_4^2$ and in general all values of the \gmn in a given coordinate system. 

This measurement procedure is meaningful under the condition that the ratio $n_1/n_2$ is fixed once and for all, so that identical rods and clocks always measure the same $ds^2$ wherever they are placed. This empirical fact. However, which is a convetion which to be rigid is  The geometry of space-time is \Mink if it is possible so that $\gmn$ are constant, the diagonal $\gmnbar=1,1,1,-1$. However, in the general case, this is not possible, as $\gmn$ can be functions of the coordinates, $\gmn \neq \gmnbar$, and space-time is non-\Mink{}ian over larger regions. This indeed, the case in the case of the presence of a real gravitational field. The \gmn are the potentials of the gravitational field. The statement of Reichenbach's famous conventionalism is in this formalism. In principle, Reichenbach pointed out, one can always reintroduce such a difference, by a universal force, that rod is not a unit rod and clock is not unit clock, by they shortened and by a field $\gamma\mn$.

In other terms one can set $\gmn=\gmnbar+\gamma_{\mu \nu}$, where $\gmnbar$ are the normal orthogonal values of the $\gmn$, and refer to the $\gmnbar$ as the \Mink geometry measured by \rac and light rays, and only to the $\gamma_{\mu \nu}$ as the some gravitational potential field determined by the path of free-falling test particles \rzlp{**}{237}. If this separation into the geometry and gravitational field is introduced, then \rac and light rays measure the $\gmnbar$, but free-falling particles follow the geodesics of the $\gamma_{\mu \nu}$ \rzlp{**}{237}. Free-falling particles will determine a \scare{measurable but distorted} geometry, which would differ from the \scare{true but hidden} \Mink geometry. However, since the $\gmnbar$ are in principle inaccessible to measurement, the splitting of $\gmnbar$ and $\gamma_{\mu \nu}$ is \q{hardly appropriate} \rzlp{**}{237}. In \gr, the gravitational field affects equally rods and clocks, light rays, and the behavior of test particles; as a consequence, all these instruments agree on the \emph{same} geometry, that is, on the same Riemannian geometry with the same curvature.

Once again this agreement is the reason \gr is to have established a new link between physical geometry and gravitation. Traditionally, \spti geometry has been associated with the behavior of ideal measuring instruments rods (geometry) and clocks (chronometry) and \scare{free} test-particles (inertial structure). In \pgrc theories, including \sr, non-accelerating \rac, and light rays determine the geometry of space and time (that is flat \Mink geometry of special relativity) $\gmnbar$, whereas the motion of test particles determine field properties. For example, charged test particles deviate from their straight \wl in \Mink \st under the influence of the electromagnetic field. The equivalence principle makes this separation impossible in the case of the electromagnetic field. Gravitation is a \emph{universal force} that affects all bodies in the same manner. \q{This is the physical significance of the equality of gravitational and inertial mass} \rzlp{**}{257}: \q{If gravitational and inertial mass were not equal, we would not be able to look upon the paths of freely falling mass points as (four-dimensional) geodesics \textelp{}, \emph{since different geometries would result for the various materials of the mass points}} \rzlp{**}{257}. It be bacuse that are equal that that  \rac and light rays and free-falling particles define a \emph{single geometry} \rzlp{**}{256}. The distinction between the fixed geometrical background and the changing gravitatianl field disappears.

As Reichenbach points out, because of this non-separability, \q{it has occasionally been said that this conception deprives gravitation of its physical character and \emph{that gravitation, therefore, becomes geometry}} \rzlp{**}{256}. However, this conclusion, in Reichenbach's view, was not justified. Two aspects have to be distinguished:

\begin{itemize}
\item The universal effect of gravitation on all kinds of measuring instruments defines a \emph{single geometry}. \q{In this respect we may say that gravitation is \emph{geometrized}} \rzlp{**}{256}. We do not speak of deformation of our measuring instruments \q{produced by the gravitational field}, but we regard \q{the measuring instruments as \scare{free from deforming forces} in spite of the gravitational effects} \rzlp{**}{256}. what theu disagree, from \Mink space-time.

\item Even if \q{we do not introduce a force to explain the \emph{deviation} of a measuring instrument from some normal geometry}, we must still invoke a force as a \emph{cause} for the fact that \q{there is a general correspondence of all measuring instruments} \rzlp{**}{256}. One may still consider the gravitational field as the \emph{cause} of the fact that all measuring instruments happen to \textit{agree} on the same geometry.
\end{itemize}

Thus, Reichenbach points out that there two different philosophical issues at stake that should not be confused:

\begin{itemize}
\item Riemann, Helmholtz, and Poincar\'e introduced the problem of the \emph{coordinative definition} in the philosophy of geometry \rzlp{**}{257}. \Gr continued in this tradition. Once one defines $ds\pm =1$, $ds=0$ in terms of \rac and light rays, the geometry of \st can be ascertained empirically; it is a branch of physics that can be true or false.

%Conceptual definition of $ds\pm =1$, $ds=0$ in integrated by by light rays, there is a coordinative definition that \rac.

\item Einstein introduced the problem of a \emph{scientific explanation} of physical geometry, which finds its mathematical solution in the field equations. The gravitational field has an effect on \rac and light rays, that is comparable to that of any other field of force, if not for the fact that it is a universal effect \rzlp{**}{256}.

\end{itemize}
%
The issue of \s{coordination} is well-known to Reichenbach's scholars, the second one has been hardly emphasized in the literature. The conventionalist move is a stepping stone. Once, again it serve to breake the crypto-kantian prejudice that natural geoemty is Euclidean. However, once the is broken Reichenbach once again to reastiblich of geomery. Indeed, the removal of serve that only one convetional choice is legitimate, and the geometry is empirical task. 

%Still However, it has not been appreciated that 

%Reichenbach insisted on the importance of the problem of \scare{explanation} on several occasions, including in his writings on \sr. 

%Lorentz provided a \emph{dynamical explanation} for the fact that \rac do not behave according to Newtonian \spti, but they are distorted by their motion through the ether. By contrast, Einstein-\Mink provided a \emph{geometrical illustration} of the same behavior by claiming that \rac behave as they do because \spti is \Mink{}an.  In \gr, an analogous distinction can be made. However, no explation is proved, of wy theu do conferom to this geomety and not another. In \gr, an analogous distinction can be made. 

Rods and clocks, light rays, and free-falling particles all agree on the same non-Euclidean non-flat \spti geometry. Two types of explanations of this empirical fact are possible: \begin{inparaenum}[(1)] \item a \emph{dynamical explanation}: there is force, gravitation, that distorts all our measuring instruments, \rac, light rays, and free-falling particles in the same way \item a \emph{geometrical explanation}: free-falling particles light rays, \rac behave as they do the geometry of \st is non-Euclidean, and \end{inparaenum}. Like in the case of \sr, Reichenbach considered the first dynamical explanation unsuitable since it introduced an otherwise inaccessible background geometry. However, he considered the geometrical explanation, simply the renunciation of any explanation. It is simply the restament of the agrement between, light particle  and matter goemtry, happen to agree.

%It simply the repression of a physical of fact in geometrical terms, that gravitational field \textit{does not} affect the behavior of \rac. 

Thus, Reichenbach insists that also in \gr, it was necessary to provide a dynamical explanation of the observed behavior of \rac, although a dynamical explanation of a new kind. As we have seen, traditionally, we introduce a dynamical explanation to account for the fact that \rac, light rays, and test particles \emph{diverge} from an alleged standard behavior defined by a flat background geometry. However, according to Reichenbach, one needs to explain why \rac, light rays, and test particles \emph{converge} towards a \scare{single geometry}, which is general non-flat and depends on the matter distribution. In order to characterize this kind of explanation, Reichenbach resorts once again to Weyl's notion of \scare{\emph{adjustment}} (\german{Einstellung}) to contrast to that of \emph{deviation} \origg{Abweichung}: \q{The word \scare{adjustment,}, which was first used by Weyl in this connection, characterizes the problem very well} \rzlp{**}{201}. \Gr requires the introduction of a similar explanation. It cannot be an accident that, say, around the sun, measuring instruments of all kinds, light rays, free falling ecc. all agree on the same non-flat \spti geometry.

%Weyl had used the expressions to account for the non-trivial Riemannian behavior or \rac. It cannot be an accident that two measuring rods that have the same length at one place always have the same length when brought to a different place along different paths. \q{It must be explained as an adjustment to the field in which the measuring rods are embedded like test-bodies} \rzlp{**}{201}. \Gr requires the introduction of a similar explanation. It cannot be an accident that, say, around the sun, measuring instruments of all kinds, light rays, free falling ecc. all agree on the same non-flat \spti geometry. The geometrical is only statemnet of this fact, that that this fact must be explaiedn: In Reichenbach's view, \q{the word \scare{adjustment} \textelp{} poses a problem rather than supplies a solution}. As Reichenbach quite surprisingly pointed out, \q{\textins{t}he answer can, of course, be given only by a detailed theory of matter, which has not yet been elaborated} \rzlp{**}{201}.

The combination of gravitation and geometry, therefore, does not force us to forgo \emph{dynamical explanations} but teaches us that this kind of explanation is applicable even to those cases in which it was customary to resort to a \emph{geometrical explanation}.  Indeed, according to Einstein's theory, \gr teaches us that we may consider the \q{effect of gravitational fields on measuring instruments to be of the same type as all known effects of forces} \rzlp{**}{257}. What is characteristic of gravitation is that this force has an effect on the measuring instruments also used in geometry, \rac and light rays, that is on those measuring instruments that in previous theories could be insulated from the effects of physical fields \rzlp{**}{13}. The universal influence of the gravitational field on all our measuring instruments, however, does not imply that it is \q{\emph{the theory of gravitation that becomes geometry}}; on the contrary, that it implies that \q{\emph{it is geometry that becomes an expression of the gravitational field}} \rzlp{**}{256}. 

According to Reichenbach, in \gr, too, only a theory of matter can explain this peculiar behavior of the space-time measuring instruments. In the presence of a real gravitational field it is impossible to arrange rods and clocks in a rectangular grid, just like it is impossible to \scare{develop} a flat piece of paper around a sphere. That the geometry of space-time is a flat network of \rac, cannot explain the behavior of bodies—that explains why gravitational fields follow the geodesic path, and not \q{We know that a more detailed investigation would reveal the presence of molecular force-fields, which affect the molecules on the surface of the sphere and thus force it into a definite}{wir wissen, daß eine genauere Betrachtung das Vorhandensein molekularer Kraftfelder lehren W\"{u}rde, die die an der Oberfl\"ache der Kugel liegenden Molek\"{u}le angreifen und in bestimmte Bahnen zwingen} \rzl{295}{258} congruence relationship. Once again, only a dynamical expansion could complete the theory. That the Earth is spherical, however, clearly shows that we should explain how the surface acts to resist or react. This would be the proper explanation—in the case of space or space-time, only a theory of matter can provide it.



\section*{Conclusion}

As the preceding sections have shown, Reichenbach consistently argued that \sr is explanatorily inadequate without a \textit{detailed theory of matter} describing the behavior of \rac. On this basis, the paper concludes that, from a \emph{historical} point of view, Reichenbach's interpretation of \sr should be seen as the first attempt at a dynamical interpretation of special relativity\footnote{\citet[\S40]{Reichenbach1928} also pursued a dynamical interpretation of general relativity, but this lies beyond the scope of this paper} that does not advocate a return to the ether theory. More importantly, one can venture to claim that, from a \emph{systematic} point of view, Reichenbach's axiomatization offers a more compelling version of the dynamical approach than the one advocated today. In particular, Reichenbach provides a clearer articulation of the key features of what counts as a \s{good} dynamical explanation in this context:


\begin{itemize}
\item \latin{explanandum}: what must be explained is not \emph{Einstein contraction} but \emph{Lorentz contraction}. The former is frame-dependent and needs no explanation; the latter reflects the coincidental agreement between rod- and light-based distance measurements, which calls for explanation.

\item \latin{explanatio}: the sought-for explanation should not aim to account for the \emph{deviation} from a classical standard—since classical \rac behavior is not inherently natural—but for the \emph{adjustment} of the behavior of material \rac to that of electromagnetic signals.

\item \latin{explanans}: Minkowski's geometrical interpretation of special relativity merely provides a \emph{coordination} of the relativistic behavior of rods and clocks with the configuration of lines in a \spti diagram, due to the shared mathematical structure. An \emph{explanation} of the relativistic behavior of \rac requires a theory of their material structure.

\end{itemize}
%
On the one hand, Reichenbach rules out the geometrical approach, questioning the explanatory role of geometry in physics. On the other hand, he presents a version of the dynamical approach that addresses some common objections—particularly the claim that it misunderstands length contraction’s perspectival nature. In this sense, Reichenbach's interpretation helps to \emph{strengthen} the dynamical approach's framework. However, for this very reason, it more clearly exposes what I take to be the framework's \emph{fault line}.

%Einstein transformed this coincidence into a \textit{convention}, whereas it calls for an \emph{explanation}.  adjust by them- selves, that is, without human interference, to the requirements of the Lorentz transformation

In Reichenbach's reading, \sr claims that the Lorentz transformations, which are derived light-geometrically, also apply to \rac. This coincidence calls for an explanation. \Sr would need to \emph{find} a \emph{specific} Lorentz-invariant theory of matter accounting for the serendipitous behavior of rods and clocks. Here, in my opinion, lies Reichenbach’s \s{dynamist} misconception. \Sr needs only to \emph{require} that \emph{any} possible theory of matter be Lorentz-invariant. \Mink's formalism offers the advantage of directly verifying whether available theories satisfy this constraint\footnote{See, \eg, \textcites[340]{Einstein1914}[59]{Einstein1949}}. In this sense, \citet{Einstein1919} famously described \sr as a \s{principle theory} rather than a \s{constructive one}. Of course, Einstein did not deny that a detailed constructive theory of matter must be found; its equations might have solutions corresponding to the material systems used as \rac. But such a theory would be an \emph{instantiation} of \src requirement, not an \emph{explanation} of the relativistic behavior of \rac.

As Schlick sensed, Einstein's and Reichenbach's different reactions to Miller's unexpected outcome go to \s{the philosophical core of the issue}. For Reichenbach, if Michelson-type experiments yielded a positive effect, \s{nothing would change}: a Lorentz-invariant theory of radiation could be maintained alongside a non-Lorentz-invariant theory of matter. Within the dynamical approach, the fact that the \textit{particular} theories realized in nature \emph{happen to be} Lorentz invariant is a mere coincidence. On the contrary, for Einstein, if Miller's result were to be confirmed, \sr \s{would collapse like a house of cards}: the requirement that \textit{any} possible theories of matter and radiation \emph{must be} Lorentz invariant could not be upheld. \Sr entails nothing but the imposition of this constraint. Even if other theories held in nature, \ED experiments would still yield a negative result, provided the theory is Lorentz invariant. Following the suggestion of Marc \citet{Lange2016}, if one wants to frame \sr in explanatory terms, one might speak of an \emph{explanation by constraint}.

\citetrackerfalse
%\printshorthands
\printbibliography
%
\end{document}