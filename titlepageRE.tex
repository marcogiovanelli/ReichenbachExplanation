\documentclass[final,12pt,a4paper]{article}
\usepackage{wrapfig}
\usepackage[font=footnotesize]{caption}
\usepackage{fullpage}
\usepackage{els}
\usepackage{ii}

\title{Reichenbach and the Prehistory of the Dynamical Approach to Special Relativity}


\begin{document}
\maketitle


\begin{abstract}
The paper aims to revisit Reichenbach’s interpretation of special relativity, making two different but interrelated claims: \begin{inparaenum}[(I)] \item Reichenbach's interpretation is best characterized not as a variant of the conventionalist interpretation, but rather an early form of the dynamical interpretation; \item Reichenbach offers a more robust version of the dynamical interpretation than contemporary accounts. \end{inparaenum} On this basis, the paper argues that Reichenbach’s approach provides the conceptual resources to \begin{inparaenum}[(I)] \item strengthen the dynamical approach against common criticisms from defenders of the geometrical approach, \item expose on its true weak point of both approaches. \end{inparaenum} Unlike the dynamical approach, special relativity does not require a \emph{specific} theory of matter to explain ether drift experiments; rather, it demands that \emph{any} such theory be Lorentz invariant. Unlike the geometrical approach, \Mink’s formalism helps test this requirement but lacks explanatory power. The paper concludes that, following Lange, \sr provides an \s{explanation by constraint}.
\end{abstract}



\begin{keywords}
Hans Reichenbach \sep Lenght Contraction \sep Special Relativity \sep Dynamical Relativity \sep explanation
\end{keywords}


\thispagestyle{empty}

%\newpage
\vspace{2cm}
\section*{Acknowledgments}

This paper has been financed by PRIN: PROGETTI DI RICERCA DI RILEVANTE INTERESSE NAZIONALE – Bando 2022, Prot. 20224HXFLY


%\section*{Ethical Statement}
%
%\begin{itemize}
%\item Funding: None
%\item Conflict of Interest: None declared
%\item  Ethical approval: Not required
%\item Informed consent: Not required
%\item Author contribution: I am the sole author of this paper
%
%\end{itemize}
\end{document}