% !TEX encoding = UTF-8 Unicode
\documentclass[submitted,12pt]{article}
\usepackage{appendix}
\usepackage{els}
\usepackage{caption}
\captionsetup{font=scriptsize}
\usepackage{notate}
\usepackage{enumitem}
\newlist{enumerate*}{enumerate}{1}
\usepackage{mat2}
\newcommand{\Mink}{Minkowski\xspace}
\newcommand{\PRZL}{\citetitle{Reichenbach1928}\xspace}
%\newcommand[\rzla]{\citep[#1]{Reichenbach1928}; tr.~\cite[041-2101, #2]{HR}}
\newcommand{\crc}[2]{(\cite[#1]{Reichenbach1925}/#2)}
%; tr.~\cite*[#2]{Reichenbach2006}
%\cite[#3]{Reichenbach1926a}; tr.~\cite*[#4]{Reichenbach2006}}
%\cite[#3]{Reichenbach1924}; tr.~\cite*[#4]{Reichenbach1969}
\newcommand{\crw}[2]{(\cite[#1]{Reichenbach1926a}/#2)}

\usepackage{etoolbox}

\pretocmd{\everypar}{\citereset}{}{}

%; tr.~\cite*[#2]{Reichenbach2006}

\RenewDocumentCommand\qa{mmoo}{\blockquote[{\cite[#3]{Reichenbach1928}; tr.~\cite[041-2101, #4]{HR}}]{#1} \orig{#2}\xspace}
\RenewDocumentCommand\qah{mmoo}{\blockquote[{\cite[#3]{Reichenbach1928}; tr.~\cite[041-2101, #4]{HR}}]{#1} \hide{#2}\xspace}

\newcommand{\crz}[2]{(\cite[#1]{Reichenbach1928}/#2)\xspace}
\renewcommand{\rzl}[2]{(\cite[#1]{Reichenbach1928}/#2)\xspace}
%; tr.~\cite*[#2]{Reichenbach1958}

%\NewDocumentCommand\qrz{mmoo}{\blockquote[{\cite[#3]{Reichenbach1928}; tr.~\cite*[#4]{Reichenbach1958}}]{#1} \orig{#2}\xspace}
%\NewDocumentCommand\qrzh{mmoo}{\blockquote[{\cite[#3]{Reichenbach1928}; tr.~\cite*[#4]{Reichenbach1958}}]{#1} \hide{#2}\xspace}
\NewDocumentCommand\qrw{mmoo}{\blockquote[{\cite[#3]{Reichenbach2006}; tr.~\cite*[#4]{Reichenbach1958}}]{#1} \orig{#2}\xspace}

\NewDocumentCommand\qrwh{mmoo}{\blockquote[{\cite[#3]{Reichenbach1926a}; tr.~\cite*[#4]{Reichenbach2006}}]{#1} \hide{#2}\xspace}

\NewDocumentCommand\qrc{mmoo}{\blockquote[({\cite[#3]{Reichenbach1925}; tr.~\cite*[#4]{Reichenbach2006}})]{#1} \orig{#2}\xspace}
\NewDocumentCommand\qrch{mmoo}{\blockquote[{\cite[#3]{Reichenbach1925}; tr.~\cite*[#4]{Reichenbach2006}}]{#1} \hide{#2}\xspace}
\NewDocumentCommand\qax{mmoo}{\blockquote[{\cite[#3]{Reichenbach1925}; tr.~\cite*[#4]{Reichenbach1969}}]{#1} \orig{#2}\xspace}
\NewDocumentCommand\qaxh{mmoo}{\blockquote[{\cite[#3]{Reichenbach1924}; tr.~\cite*[#4]{Reichenbach1969}}]{#1} \hide{#2}\xspace}

\NewDocumentCommand\lK{mm}{\ensuremath{l^{K{#1}}_{K{#2}}}}
\NewDocumentCommand\LK{mm}{\ensuremath{L^{K{#1}}_{K{#2}}}}


\newcommand{\MME}{Michelson–Morley experiment\xspace}
\renewcommand{\qrw}{\qrwh}
\renewcommand{\qa}{\qah}
\renewcommand{\qrc}{\qrch}
\renewcommand{\qrz}{\qrzh}
\renewcommand{\qax}{\qaxh}
%\newcommand{\SN}{\cite{SN}}
%\newcommand{\Ap}{Appendix\xspace}
\renewcommand{\hr}[1]{HR 041-2101, #1\xspace}
\renewcommand{\rzla}[2]{(\cite[#1]{Reichenbach1928}; tr.~\cite*[041-2101, #2]{HR})\xspace}
\newcommand{\cax}[2]{(\cite[#1]{Reichenbach1924}; tr.~\cite*[#2]{Reichenbach1969})\xspace}
\newcommand{\ra}[2]{(\cite[#1]{Reichenbach1924}; tr.~\cite*[#2]{Reichenbach1969})\xspace}
\newcommand{\rc}[2]{(\cite[#1]{Reichenbach1925}; tr.~\cite*[#2]{Reichenbach2006})\xspace}
\newcommand{\rw}[2]{(\cite[#1]{Reichenbach1926a}; tr.~\cite*[#2]{Reichenbach2006})\xspace}
%\renewcommand{\rzl}[2]{(\cite[#1]{Reichenbach1928}; tr.~\cite*[#2]{Reichenbach1958})\xspace}
\renewcommand{\theequation}{\roman{equation}}
\begin{document}

\title{Reichenbach and the Prehistory of the Dynamical Approach to Special Relativity}
\maketitle

\newcommand{\rhp}[2]{(\cite[#1]{Reichenbach1920a}; tr.\ \citeyear{Reichenbach1969} #2)\xspace}

\newcommand{\wpo}{worldpoint\xspace}
\begin{abstract}
The paper aims to revisit Reichenbach’s interpretation of special relativity, making two different but interrelated claims: \begin{inparaenum}[(I)] \item Reichenbach's interpretation is best characterized non as a conventionalist interpretation, as is usually argued, but rather an early form of the dynamical interpretation; \item Reichenbach offers a more robust version of the dynamical interpretation than contemporary accounts. \end{inparaenum} On this basis, the paper argues that Reichenbach’s approach provides the conceptual resources to \begin{inparaenum}[(I)] \item strengthen the dynamical approach against common criticisms from defenders of the geometrical approach, \item put one's finger on the true weak point of both approaches. \end{inparaenum} Unlike the dynamical approach, special relativity does not require a \textit{specific} theory of matter to explain ether drift experiments; rather, it demands that \textit{any} such theory be Lorentz invariant. Unlike the geometrical approach, \Mink’s formalism helps test this requirement but lacks explanatory power. The paper concludes that, following Lange, \sr provides an \s{explanation by constraint}.
\end{abstract}

%In 1925, Reichenbach responded to Miller's claimed detection of ether drift by introducing a distinction in \sr between two forms of rod contraction: a kinematical \scare{Einstein contraction}, which arises from the chosen definition of simultaneity, and a dynamical \scare{Lorentz contraction}. He maintained that although both yield the same Lorentz factor, they differ conceptually. According to Reichenbach, only the \scare{Lorentz contraction} is relevant to the Michelson-Morley experiment. In both Einstein's and Lorentz's theories, the arm of Michelson's interferometer is shorter than classical mechanics would predict. However, Lorentz contraction, in either case, calls for an atomistic account grounded in a yet-to-be-formulated theory of matter. This paper argues that Reichenbach’s interpretation of \sr anticipates several aspects of contemporary neo-Lorentzian approaches.


\begin{keywords}
Hans Reichenbach \sep Lenght Contraction \sep Special Relativity \sep Dynamical Relativity \sep explanation
\end{keywords}

\intro

%Dürr, P, and J Read. 2024. "An invitation to conventionalism: a philosophy for modern (space-)times." Synthese 204 (1): 1-55. https://doi.org/https://doi.org/10.1007/s11229-024-04605-z.


%"Neue Experimente \"uber den Einfluss der Erdbewegung auf die Lichtgeschwindigkeit relativ zur Erde"

%That Reihcenbach to insited that dynancal. that should regarded as sort father of the dynalical as opposed to geometrical one.  This is already present in his 1924 \citetitle{Weyl1924} \citep{Weyl1924}, but it is probbale a 1925 article  A good opportunity to point to epxaon aout fas the alleged refutation of Michelson experiemnt by Miller 1970. The reust of the expeimernt were contested, but question could be addresed.: what would happen to special relativty if the Michelson-Morely expeormt was rejec,t that is when turent to be correct? 

Harvey \citets{Brown2005} book \citetitle{Brown2005} is widely credited with reshaping the debate on the foundations of space-time theory over the past two decades. Brown argues that special relativity as it stands is incomplete: phenomena like length contraction must ultimately receive a \textit{dynamical explanation} in a fundamental theory of the material structure of rods. Defenders of the traditional view, such as Michel \citet{Janssen2009}, object that special relativity was already completed by \Mink, who provided a theory of the mathematical structure of \spti: length contraction receives a \textit{geometrical explanation} from the fact that the world tube of the rod is intersected differently by the hyperplane of simultaneity. Regardless of one’s position, it is clear that the discussion surrounding the role of \rac in relativity theory has undergone a fundamental transformation. For a long time, the debate—rooted in logical empiricism—framed the role of \rac in \rt in terms of \textit{confirmation}, as a dispute between empiricism and conventionalism. Since Brown's work, the center of gravity has shifted toward the problem of \textit{explanation} of the behavior of \rac, recasting the discussion as a conflict between dynamical and geometrical approaches.

%Reichenbach denied the explanatory power to \Mink \spti, and argued that \sr requires the behavior of \rac to be explained by a specific, though still unknown, theory of matter. His proposal fell flat and has been largely forgotten \citep[see, however][]{Gruenbaum1955,Gruenbaum1963a}; indeed, most of his readers have focused on his well-known conventional $\epsilon$-definition of simultaneity. 

 
This paper aims to uncover an overlooked chapter in the pre-history of the dynamical approach. It argues that it was the logical empiricist Hans Reichenbach who, already in the 1920s, first to draw the attention to the problem of \s{explanation} in \spti theories. Reichenbach denied the explanatory power to \Mink \spti, and insisted that \sr requires the behavior of \rac to be explained by a specific, though still unknown, theory of matter. The paper seeks to draw renewed attention to this aspect of Reichenbach's interpretation of relativity, which has been largely overlooked\footnote{\citepp[See, however][]{Gruenbaum1955}[401]{Gruenbaum1963a}}. In particular, this paper advances two distinct but interrelated claims:

\begin{enumerate}[label=(a)] 
\item \emph{historical claim}: Reichenbach's axiomatization of special relativity was not a variant of the conventionalist interpretation paradigmatically articulated  by Moritz~\citet{Schlick1915}, as is usually argued, but an early variant of the dynamical interpretation.

\item \emph{systematic claim}: through a careful analysis of both the notion of \s{contraction} and the notion of \s{explanation}, Reichenbach presents, in many respects, a more robust version of the dynamical interpretation of \sr than the contemporary ones.
\end{enumerate}  
%
Building on these two claims, the paper maintains that Reichenbach's work provides the conceptual tools \begin{inparaenum}[(I)] \item to steel man the dynamical approach against the most common objections; \item to expose the crack in what seems to be its strongest armor \end{inparaenum} In contrast to the dynamical approach, special relativity does not need to explain the failure of ether drift experiments by \emph{finding} a \emph{specific} Lorentz invariant theory of matter; rather, relativity explains such negative results by \emph{requiring} that \emph{any} possible theory of matter \emph{must} be Lorentz invariant. \Mink's mathematical apparatus allows one to check directly whether available laws comply with this requirement, but, contrary to the geometrical approach, does not provide any further explanatory contribution. The paper wraps up by arguing that, if one wants to cast the contribution of \sr relativity in explanatory terms, it is better to speak of an \emph{explanation by constraint}, as suggested by Marc \citet{Lange2016}.

%\lipsum[2]

\section{The Schlick-Reichenbach Correspondence and Reichenbach's  Cautious Conventionalism}
\label{schlickreichenbach}


Reichenbach's habilitation thesis, \citetitle{Reichenbach1920a} \citep{Reichenbach1920a}, appeared in print during the famous 86th meeting of the \german{Gesellschaft deutscher Naturforscher und Ärzte} in Bad Nauheim (19–25 September), marking the beginning of a politically charged backlash against modern physics in Germany \citep{Weyl1920}. Schlick, who did not attend the meeting, received the booklet in those days \letterhrp{Schlick}{Reichenbach}{25}{9}{1920}[015-63-23]. Writing to Einstein, he praised it, but complained about Reichenbach's \s{Kantian} critique of conventionalism \lettercpaep{Schlick}{Einstein}{23}{9}{1920}[**]. Schlick articulated his stance corresponding with Reichenbach himself in the ensuing months \citep{Oberdan2009}. 

In the book, Reichenbach shared with \citet{Schlick1918} the idea that physical knowledge is, ultimately \s{coordination} (\german{Zuordnung}), the process of relating an axiomatically defined mathematical structure to concrete empirical reality \citep{Padovani2009}. However, Reichenbach attempted to give this insight a \scare{Kantian} twist. According to Reichenbach, in a physical theory, besides the \scare{axioms of connections} (\german{Verknüpfungsaxiome}) encoding the mathematical structure of a theory, one needs a special class of physical principles, the \scare{axioms of coordination} (\german{Zuordnungsaxiome}), to ensure the univocal coordination of that structure to reality. For the young Reichenbach, the latter axioms are \apr because they are \scare{constitutive} of the object of a physical theory. However, they are not apodeictic or valid for all time. 

Writing to Reichenbach, Schlick objected that, at closer inspection, Reichenbach's coordinating principles where nothing but arbitrary \scare{conventions} in the sense of Poincaré \letterhrp{Schlick}{Reichenbach}{26}{11}{1920}[015-63-22]. Reichenbach initially opposed some resistance. If the coordinating principles were fully arbitrary, he feared, they would be empirically meaningless\todo{check}. In Poincaré's conventionalism, Reichenbach missed a constraint in \q{the arbitrariness of the principles \textelp{}, if the principles are combined}; Therefore, he concluded \qt{I cannot accept the term \s{convention}{}}{daß die Willkürlichkeit der Prinzipien eingeschränkt ist, sowie man Prinzipien KOMBINIERT. Darum kann ich den Namen \s{Konvention} nicht annehmen} \letterhrp{Reichenbach}{Schlick}{26}{11}{1920}[015-63-22]. Schlick replied it would, of course, be unfair to assume that he was unaware of this fact \letterhrp{Schlick}{Reichenbach}{11}{12}{1920}[015-63-22]. Indeed, in January \citet{Schlick1921} mentioned favorably Reichenbach's book in an article for the \jt{Kant-Studien}, but rehearsed in public what he had already explained in private correspondence. 

Still, Reichenbach did not seem to have been fully turned on the conventionalist side at this point. Einstein's famous January lecture on \scare{geometry and experience}--- published in March in the Proceedings of the Berlin Academy \citep{Einstein1921}---might have been instrumental to bring Reichenbach closer to Schlick's position. Not only did Einstein mentions Schlick's work approvingly, but he also declared that Poincaré was \emph{sub specie aeterni} correct in claiming that so that in principle only geometry plus physics could be compared by experience. In reviewing the book version of Einstein's lecture Schlick could interpret this claim as a confirmation of the form of \s{holistic conventionalism} \citep{Schlick1921a} that he had attributed to Poincaré and, then, extended to Helmholtz in his edition of the latter's epistemological writings \citep{Helmholtz1921}. As Reichenbach recognized in his review, one of the merit of the Schlick-edition was to have shown that \qt{Poincaré did not express conventionalism more clearly}{klarer hat auch Poincaré den Konventionaliomus night ausgesprocaen} than Helmholtz already did \citep{Reichenbach1921b}.

By September 1921, Reichenbach wrote to Schlick that he considered their difference in opinion as resolved \letterp{Reichenbach}{Schlick}{17}{9}{1921}[][SN]. He hoped to finally meet Schlick in person in Jena at the meeting of the \GDNA, where he was going to present his project for an axiomatization of \sr that he had developed in the previous months (\letter{Reichenbach}{Schlick}{17}{9}{1921}[][SN]). Indeed, when he sent to Schlick the published version of the report (that was published in December), Reichenbach emphasized that his axiomatization \qt{obviously provides a confirmation of conventionalism}{Sie liefert natürlich eine Bestätigung des Conventionalismus}. However, he also qualified his remark by insisting that \q{reveals those facts that also conventionalism cannot interpret} (\letter{Reichenbach}{Schlick}{18}{1}{1922}[][SN]). Reichenbach clarified his position more thoroughly in a long review paper on the philosophical interpretations of relativity, completed around March 1922 (\lettercpae{Freundlich}{Einstein}{24}{3}{1922}[13][119] \lettercpae{Einstein}{Reichenbach}{27}{3}{1922}[13][119]). 

By sketching his own interpretation, Reichenbach emphasizes once again that he prefers to \emph{avoid} the term \s{conventionalism}, for the following reasons: (1) it does not express, the important Kantian intuition that the non-empirical principles are \s{constitutive} for the concept of the object; (2) it overemphasizes the arbitrary nature of the principles of knowledge, while downplaying the fact that their combination is no longer arbitrary \citep[**]{Reichenbach1922a}. Reichenbach was willing to admit with Schlick that Poincaré would probably acknowledge this point. Nevertheless, Reichenbach still insists that the problem is \q{not only uncovering the arbitrary principles of knowledge but also determining the entirety of [their] permissible combinations} \citep[**]{Reichenbach1922a}  Reichenbach presents his axiomatization as the solution to this problem. In particular, he emphasizes that latter was organized around two distinctions:

\begin{itemize}
\item \emph{axioms} vs.\ \emph{definitions}. In this context, \s{axiom} means, unlike in mathematics, an observable fact, that has already been experimentally proven or provisionally assumed as an hypothesis. Opposed to the axioms are the \s{definitions}, which contain rules for how certain empirical realities are to be assigned to specific mathematical concepts. In particular, according to Reichenbach, Einstein discovered that determining whether two spatially separated events are simultaneous depends on a convention regarding the ratio $\epsilon$ of the one-way speeds of light in the round-trip journey between the two events. Since we cannot measure the one-way speed without already having defined simultaneity, Einstein's convention $\epsilon=\frac{1}{2}$ is not \s{more correct} than any other. The separability of the \s{conventional} elements of scientific theories (definitions) from the empirical ones (axioms) shows that Poincaréan accommodation empirical findings through changes in conventions is strongly constrained. E.g. the definition the conventions of simultaneity must be \s{univocal} (independent of prehistory), and this a \s{fact} \citep[see][]{Reichenbach1922b}

\item \emph{light axioms} vs.\ \emph{matter axioms}. Light axioms assert only the properties of electromagnetic signals, and matter axioms the properties of material rigid measuring rods and natural clocks. Reichenbach claimed be able to show that a \emph{light geometry} alone can serve as the basis for the measurement of space and time. Whereas in Einstein s original theory of relativity light served merely to determine simultaneity, in Reichenbach's axiomatization light may be used for all measurements of time intervals and space distances (that can be measured by the time needed for a light signal to travel a certain segment). The advantage of light geometry is that it avoids the definition of the \s{metric} through material entities such as rigid bodies and clocks; these entities are complexed atomic system, whose behavior presupposes the knowledge of physical laws. In Reichenbach's axiomatization, the \emph{matter geometry} based on the behavior of \rac are introduced only after the development of the light-geometry in the form of a series of \s{matter axioms}.



%Whereas in Einstein's original theory of relativity light served merely to determine simultaneity, it became clear in later developments of the theory that light can be used for all measurements of time—for defining the unit of time, and even for the measurement of space. One may construct a geometry of light in which light determines the comparison of spatial distances. Thus, light comes to function as the ordering framework of physics, gathering within the mesh of its rays all the events of the world and arranging them in numerical order.

%According to this interpretation the measurement of space can be reduced to the measurement of time; spatial distance can be measured by the time needed for a light signal to travel a certain segment.  Definition of the equality of spatial distances in terms of measurements of time intervals.



\end{itemize}
%
The separation between this two conceptual pair essentially express the spirit of Reichenbach's axiomatization: Light axioms entails only facts which are already accepted by classical optics and therefore can be considered confirmed independent of special relativity. The difference between classical and relativistic light geometry is a \emph{matter of convention} about which event one consider as simultaneous. \cop{Einstein's merit is to have shown that the simultaneity of distant events cannot be \s{verified}, it can only be \s{defined}}. Only the matter axioms encodes the physical facts that are specific to the theory of relativity. Whether the matter geometry behave according to the classical or relativistic light geometry is a \emph{matter of fact}, which in principle empirically testable: \qt{As a fundamental hypothesis of Einsteinian kinematics, one can formulate the proposition that \textins{relativistic} light geometry is identical to the geometry of rigid measuring rods and natural clocks}{Als grundsätzliche Hypothese der Einsteinsehen Kinematik läßt sich dann der Satz aufstellen, daß die Lichtgeometrie identisch ist mit der Geometrie der starren Maßstäbe und natürlichen Uhren} \citep{Reichenbach1922}.  

In a popular article in French, which he submitted to the \jt{Revue philosophique} in May 1922, \todo{improve} \citet{Reichenbach1922} still emphasizes the continuity of his \q{axiomatic efforts} with his earlier neo-Kantian stance: (a) one can abandon the first sense of \apr, \ie, its \s{apodeictic} character, without forgoing its \s{constitutive} meaning; (b) in doing so one replaces Kant's \s{analysis of reason} with \s{scientific analysis}, that is, the study of the logical structure of scientific theories \citep[**]{Reichenbach1922}. Philosophy should leave the construction of theories entirely to the sciences (without any ambition of constrain this construction \apr) and focus only on making their implicit presupposition explicit as Reichenbach claimed to have done in his axiomatic. By restricting itself to the logical structure of scientific systems, philosophy becomes immune to experimental refutation but, at the same time, loses its ability to make claims about reality. Reichenbach concludes by paraphrasing Einstein’s famous dictum from his 1921 lecture: \s{If the principles of epistemology refer to reality, they are not certain; and if they are certain, they do not refer to reality} \citep[**]{Reichenbach1922}.

% Schlick was very laudatory upon receiving Reichenbach's papers , that they were devloping the very same program e support his carrier planni review.. 

At this point, Schlick, who had moved to Vienna in the winter term of 1922-1923, considered Reichenbach a conventionalist ally and sought to support him academically as his successor in Rostock\todo{check}. At around that time, Reichenbach was already working on expanding the 1921 report into a book. However, in \datemy{5}{10}{1922}, Weyl informed him of a significant flaw in his axiomatization \citep{Rynasiewicz2005}: it was not possible to distinguish the class of inertial frames solely based on light rays as Reichenbach as claimed \letterhrp{Weyl}{Reichenbach}{5}{10}{1922}[015-68-02]. The book, which was finished in March 1923, seems to patch the problem up without, however, mentioning Weyl. After facing challenges in finding a publisher, he finally signed a contract with Vieweg by the end of  year of 1923. 

\citetitle{Reichenbach1924} was published May-June 1924. In November, \citet{Weyl1924} published a rather scathing review in the \citejournal{Weyl1924}, complaining about the cumbersomeness of Reichenbach's presentation. The attack of Weyl---one of leading relativists of his time---dealt a significant blow to Reichenbach, who would still remember the episode with bitterness a decade later (\letter{Reichenbach}{Einstein}{12}{4}{1936}[10-107][EA])  It is unsurprising that Reichenbach felt compelled to defend his work. On \datef{28}{7}{1925} the journal \citejournal{Reichenbach1925} received a paper to take stance. Reichenbach complained, that Weyl had judge his work as a \s{mathematical investigation}; Reichenbach insists that the axiomatic was meant to be an \q{epistemological clarification} of the theory of relativity \crc{**}{**}. 
%\citep{Corry2003}
As Reichenbach explains, his axiomatization differs from the typical \scare{deductive axiomatization} championed by, e.g., David Hilbert. In the latter, one sets an abstract general principle as an axiom, such as a variational principle (see \cite[2]{Reichenbach1924}). Reichenbach, in contrast, put forward a \scare{constructive axiomatization}, which uses as axioms empirical assertions capable of experimental verification to distinguish them from arbitrary \emph{definitions}. His axiomatization \qt{has the great advantage for physics that the \emph{implications of each experimental result} can be immediately recognized}{Diese Form der Axiomatik hat fiir die Physik den groBen Vorzug, dab sle die Tragweite jedes experimentellen ResuTtat} \crc{**}{172}. Some statements of the theory depend only on definitions and cannot be tested empirically. Among assertions of the theory that are passible of empirical test, not all presuppose every axiom; thus one can be immediately whether a which assertion is founded on confirmed axioms or relies on uncertain ones \crc{**}{172}.

As we have seen, light \emph{axioms} (\rom{1}-\rom{5}) are facts derived from pre-relativistic optics and can be considered sufficiently established independently of relativity \ra{4}{**}; the distinction between classical and relativistic light geometry, however, is conventional, based on an arbitrary \emph{definition} of simultaneity. In Reichenbach's axiomatization, only the matter axioms contain the empirically testable part of relativity theory, the correspondence between relativistic light and matter geometry:

\begin{itemize}
\item axioms \rom{6} and \rom{7} concern clocks and rigid rods shared with classical theory, the remaining axioms are specific to relativity. 

\item Axiom \rom{8} is nothing but an abstract formulation of the \MME. 

\item the combination of Axioms \rom{9} and \rom{10} (Axiom D) corresponds to the so-called transverse Doppler effect. 
\end{itemize}
%
While the latter still awaited experimental confirmation, the \MME was generally regarded as well-confirmed. However, a few months\todo{check} before the publication of Reichenbach's article, doubts were raised about the latter by the American experimentalist Dayton C.~\textcites{Miller1925}{Miller1925a}. Reichenbach cleverly took advantage of the ensuing controversy to challenge Weyl and demonstrate the value of his axiomatization in addressing the question: \emph{what would happen to relativity, if Axiom \rom{8} were empirically refuted?}

\section{Michelson and Miller's Experiments}
%\section{Two Kinds of Contractions: Einstein Contraction vs.\  Lorentz Contraction} 

\citet{Reichenbach1924,Reichenbach1925,Reichenbach1926a,Reichenbach1928} repeatedly offers an account of the \MME that appears fairly conventional. Still, it deserves brief attention, since Reichenbach's goal is precisely to challenge the standard account. As is well known, the essential aim of the experiment was to determine whether the speed of light varies with direction due to the Earth's motion through the hypothesized ether. In the setup schematically depicted in \cref{mv}, a beam of light is split at point $O$ by a semi-transparent mirror, sending two coherent beams along two perpendicular arms of equal length, each ending at mirrors $M_1$ and $M_2$. The beams reflect back and recombine at $O$, creating an interference pattern---a series of alternating bright and dark fringes. If they return at the same time, their wave crests and troughs align, and the pattern remains stable. If one beam takes slightly longer, the waves arrive out of step, causing a shift in the fringes \rw{325}{195f.}. 

According to ether theory, the two beams return to point $O$ at the same time only if the apparatus is at rest with respect to the ether. Since the Earth moves through the ether, the theory predicts a \s{deviation}: the beam traveling along the arm parallel to the Earth's motion (e.g., toward $M_2$) would take slightly longer to return, leading to a shift in the interference fringes.  Thus, the experiment was expected to reveal the Earth's absolute motion through the ether by detecting such a shift. In the 1880s Albert A.~\citet{Michelson1881}, later with the assistance of Edward W.~Morley (\cite*{Michelson1887}), showed that \qt{in spite of the extreme precision of the measurement, there is no difference in the time to traverse either arm of the apparatus}{trotz allergr\"oßter Meßgenauigkeit zeigte sich keine Differenz f\"{u}r die Durchlaufungszeiten auf beiden Armen der Apparatur} \crw{326}{195}. At the turn of the century, \qt{Morley and Miller [\cite*{Morley1905,Morley1905a}] replicated this negative result in spite of the renewed increase in precision}{Eine 1904/05 von Morley und Miller unternommene Wiederholung hatte denselben negativen Ausfall, trotz abermaliger Steigerung der Meßgenauigkeit} \crw{326}{195\tm}. 

How can this negative result be explained? Classical aether theory could account for it by postulating that the aether is dragged along with the Earth in its orbit. However, this stratagem contradicts the phenomenon of stellar aberration \crw{**}{**}. If instead one preserves the existence of a stationary aether, with respect to which light travels at a fixed speed $c$, it becomes puzzling how the equality of the travel times is maintained, despite the fact that the apparatus is moving through the ether. As a result, 19th-century optics found itself in a seemingly irresolvable dilemma. The subsequent unfolding, Reichenbach argues, was usually considered uncontroversial.

%How can this negative result be explained? Classical aether theory attempted to account for it by postulating that the aether is dragged along with the Earth in its orbit. However, this stratagem contradicts the phenomenon of stellar aberration \crw{**}{**}. As a result, 19th-century optics found itself in a seemingly irresolvable dilemma. According to Reichenbach, the standard view in physics was to preserve the existence of a stationary aether by introducing auxiliary hypotheses—such as length contraction—to explain why the experiment yielded no observable effect.

%According to Reichenbach, the received view of how physics attempted to resolve this puzzle was as follows: the equality holds in all frames.

%This principle must now be formulated in a more exact manner. The velocity of light is identical in all directions in a uniformly moving frame of reference, provided simultaneity is correspondingly defined. This additional

At the turn of the century, \q{Lorentz [\citeyear{Lorentz1895}] in Leyden presented his \myemph{explanation}} of this equality \qt{that assumed that all rigid bodies moving in opposition to the ether undergo a contraction}{H. A. Lorentz in Leyden eine Erkl\"arung gab, die eine Verk\"{u}rzung aller starren K\"orper bei Bewegung gegen den Aether annahm} \crw{325}{197\me}. When moving uniformly through the ether with velocity $v$ relative to the ether frame $K$ (where light travels rectilinearly at speed $c$), the arm $\mathrm{OS}_2$ contracts in the direction of motion by a factor \kappafactor—like a metal bar exposed to cold—compensating for the slower light speed and explaining the experiment's null result. This deformation cannot be measured—for example, by rotating the uncontracted $\mathrm{OS}_1$—since $\mathrm{OS}_1$ would contract equally. Lorentz justified this hypothesis by arguing that molecular forces are electromagnetic and influenced by motion. He thus maintained the existence of an undetectable aether wind over Earth’s surface.

%When moving uniformly through the ether with velocity $v$ with respect to the ether frame $K$ (in which light propagates rectilinearly with velocity $c$), the arm $\mathrm{OS}_2$ shrinks in the direction of its motion, just as a metal bar contracts when exposed to cold, and this compensates for the slower speed of light rays in that direction—accounting for the null result of the experiment. The deformation of the rod cannot be ascertained by measurement—for instance, by rotating the non-contracted $\mathrm{OS}_1$ around $O$ to check whether $S_{1}$ can be brought into coincidence with $S_{2}$. Indeed, in this case, $\mathrm{OS}_1$ would also be contracted by the same amount as $\mathrm{OS}_2$. Thus, Lorentz justified the contraction hypothesis on the grounds that the molecular forces which hold material bodies together are electromagnetic in nature and are affected by translational motion. Lorentz could then conclude that there exists an aether wind over the surface of the Earth, but it is not detectable.

In \citeyear{Einstein1905}, Reichenbach continued, \qt{a more basic explanation was proposed by A.~Einstein in which these contractions occur as a result of a universal principle, the principle of relativity}{Noch tiefer ging die 1905 von A. Einstein aufgestellte Relativit\"atstheorie, welche diese Verk\"{u}rzung als Ausfluß eines universellen Prinzips betrachtet, des Relativit\"atsprinzips;} \crw{326}{197}. Instead of introducing a \emph{deformable} rod, Einstein proposed a new definition of what counts as a \emph{rigid} rod: spatial measurements depend on temporal simultaneity, which in turn is frame-dependent. From this perspective, the horizontal arm $OM_2$ of the Michelson interferometer is \emph{not} shortened with respect to the coordinate system $K'$ co-moving with the Earth; however, it is shortened by a factor \kappafactor with respect to a coordinate system $K$ at rest relative to the Sun. Indeed, there is a disagreement about when $O$ and $M_{2}$ are measured \emph{at the same time}. Whereas Lorentz needed to \s{explain} the cause of the deformation of the rod, \cop{from Einstein's perspective, the null results of the Michelson-Morley experiment can be interpreted as confirming that the material composing the arms of the interferometer behaves like a \emph{rigid} rod}.

Reichenbach concurs with the standard semi-historical account on one point: Lorentz's and Einstein's theories \emph{agree} on the same empirical fact, stated in axiom \rom{8} of his axiom system: \s{Two space intervals which are equal when measured by rigid rods, are also light-geometrically equal}, meaning they are equal when measured by the round-trip time of a light signal. To avoid invoking simultaneity, a mirror at $B$ reflects the signal back to $A$, and the round-trip time determines the distance. The Michelson experiment tests this: if the arms $OM_1$ and $OM_2$ (\cref{mv}) are equal in terms of round-trip light time ($\overline{OM_1O} = \overline{OM_2O}$), they are also equal when measured by rigid rods ($M_1 = M_2$). The null result of the \MME confirmed this equality and was widely accepted. Further support came from Rudolf~\citet{Tomaschek1924}, a critic of relativity influenced by Philipp Lenard, who repeated the interferometer experiment using starlight \rc{39}{180}.

However, Reichenbach pointed out that \qt{\textins{r}ecently, doubts have been raised by Dayton C.~Miller, who obtained a positive result on Mount Wilson}{Neuerdings sind dagegen Zweifel erhoben worden durch Dayton C. Miller), der auf dem Mount Wilson einen positiven Effekt erhgl} \crc{39}{180}. If Miller's experiment were to be confirmed, then the round-trip times of the light signals along the two arms would become unequal ($\overline{OM_1O} \neq \overline{OM_2O}$). Therefore the matter-geometrical (rigid-rod-based) equality of distances would no longer agree with the light-geometrical (signal based) equality. Axiom \rom{8} would be disproved. At the time, there was no consensus on the reliability of Miller's results. Nevertheless, Reichenbach swiftly seized the opportunity presented by the debate after the publication of Miller's paper to persuade the many skeptics of the physical implications of his axiomatization: \qt{In this context, the axiomatization is proved to be extremely useful because it shows what particular role the Michelson experiment plays in the theory, what follows from it, and what is independent of it}{In diesem Zusammenhang erweist sich die Axiomatik als zuerst n\"{u}tzlich, well sle erkennen lgl]t, welehe Rolls der Michelsonversueh ia der Theorie iiberhaupt spielS, was aus ibm gefolgert wird und was yon ibm unabhgngig ist} \crc{39}{180}. 


\begin{figure} \hide{{r}{0.5\textwidth}} \begin{center} \includegraphics[scale=0.30, trim = 0mm 0mm 0mm 0mm, clip]{graph/mv} \caption{Reichenbach's stylized Michelson-Morley apparatus \citep[from][labels have been changed]{Reichenbach1926a} \label{mv} The experiment was meant to test whether the rods satisfy the light-geometrical definition of length in every inertial system:  $OM_{1}=OM_{2}$ when  $\overline{O M_{1} O}=\overline{O M_{2} O}$. According to classical theory, the equality is satisfied only in ether system}\end{center}
 \end{figure}
 
\section{Two Kinds of Contractions: Einstein Contraction vs.\  Lorentz Contraction} 

The question, then, is this: what would become of special relativity if the Michelson–Morley experiment were rejected—that is, if Axiom \rom{8} turned out to be incorrect?  Before addressing this question directly, Reichenbach cautioned his readers against uncritically accepting the standard interpretation of the Michelson experiment that we have just presented. Lorentz's dynamical contraction of one arm of the apparatus is usually considered an \emph{ad hoc} hypothesis. On the contrary, Einstein's hypothesis of the contraction resulting from the relativity of simultaneity appeared to be less contrived \citep[see, e.g.][]{Schlick1915}. According to Reichenbach, both of these claims are incorrect. 

Einstein's theory, as well as Lorentz's, agree on the fact that the behavior of rigid rods differs from that predicted by classical theory and conform to the relativistic light geometry; therefore, the contraction hypothesis is not \emph{ad hoc}. However, contrary to the received wisdom, the relativity of simultaneity has nothing to do with the contraction involved in \MME. The contraction of the arm of the apparatus occurs for the system relative to which the apparatus is at rest (the Earth), not with respect to the system relative to which the apparatus is moving (the Sun):

\qt{\textins{W}e should examine a particular error that has crept into the understanding of the theory of relativity. It concerns the problem of Lorentz contraction and thereby leads us to the Michelson experiment. One frequently hears the opinion expressed that in the Lorentzian explanation of the Michelson experiment the contraction of the arms of the apparatus is an \scare{\latin{ad hoc} hypothesis}, whereas Einstein explains it in a most natural way, namely, as a result of the relativization of the concept of simultaneity. But this is false. \myemph{The relativity of simultaneity has nothing to do with length contraction in the Michelson experiment}. That this opinion is false already follows from the fact that the contraction of one of the arms of the apparatus occurs precisely in the system in which the apparatus is at rest}{Man hiirt oft die Meinung uusgesproehen, in der \ls{Lorentz} sehen Erklartmg des Michelsonversuehs sei die Kontraktion des eiaea Appuraturines eine ,,ad hoe ersonnens Hypothese", wahrend sie bei Einstein uuf die nutiirliehste Weise erklart sei, namlieh uls Folge der Relativlerung des Gleiehzeitigkeltsbegriffs. Abet dies ist fulseh. Die Re]ativitat der Gleiehzeitigl~eit hut mit der Stabkontraktion des Miehelsonversuehs niehts zu tun, and die E in s t e i n sehe Theorie gibt hierfiir ebensowenig eine Erklarung wie die Lorentzsehe}[\crc{43}{187\me}] 
%
As we have mentioned, in order to explain the negative result of the Michelson experiment, Lorentz made the assumption that the arm of the apparatus aligned with the direction of motion is \emph{contracted} by the amount \kappafactor when it moves relative to the ether. The theoretical asymmetry between the ether frame and frames moving with respect to it is hidden from observation by introducing a sort of universal conspiracy of nature. We find ourselves in an aether \s{storm}, yet the laws of nature are so precisely arranged that we are entirely unable to perceive it. Einstein, on the contrary, declared both arms \emph{equally long} when measured in their own rest frame, but one arm would appear contracted by the factor \kappafactor when observed from a relatively moving system. In this way, the theoretical symmetry between rest and moving systems is reestablished, and the ether could be expunged from the theory.

%The theory predict that $A'B'<AB$.  How the length of this projection. This also on the shape of a moving object indeed, Since the shape is determined by the simultaneous projection of all the points.This result is the basis of Einstein's principle of the constancy of the velocity of light; the motion of light can be considered as a spherical wave for any uniformly moving system. Our presentation enables us to visualize Einstein's result. The shape of the surface of the light wave is not uniquely determined but depends on the definition of simultaneity. Its so-called shape is always

For this, reason Einstein contraction has often been called \s{apparent}, for it is not the arm of the interferometer that contracts, but its \s{projection} onto a system at rest. Thus, the contraction disappears if the arm is measured in the co-moving earth frame\footnote{Let $K$ be a stationary frame endowed with synchronized clocks. A bar $AB$ is moving along the $x$-axis of $K$. We could imagine a device that marks two points $A'$ and $B'$ at the moment when $A$ and $B$ coincide with the $x$-axis at the same time as determined by the stationary clocks in $K$. $A'B'$ is what Reichenbach calls the \emph{projection} of $AB$ in $K$. We first measure the length of the bar $AB$ using a rigid unit rod at rest relative to it; then we carefully decelerate the bar so that it does not change its length, and measure the length of the projection $A^{\prime}B^{\prime}$ using the same rod at rest in $K$. The theory of relativity predicts that $A^{\prime}B^{\prime} < AB$. If we then accelerate the rod again and remeasure $AB$, we of course find the same length as before; thus, the contraction disappears}. In contrast, Lorentz contraction which is regarded as \s{real}, since a true dynamical phenomenon cause by effect of the ether on molecular forces, similar to the contraction of a metal caused by a drop in temperature. Reichenbach, however, explicitly rejects this widespread interpretation. In particular, Reichenbach denies that Einstein contraction has anything to do with the null result of the \MME:

\qt{That this opinion is false already follows from the fact that the contraction of one of the arms of the apparatus occurs precisely in the system in which the apparatus is at rest. The \scare{Einstein contraction} only explains that the arm is \myemph{shortened if it is measured from a different system}. But that does not explain the Michelson experiment. \textins{The latter} proves that the rod lying in the direction of motion \myemph{is shorter when measured in the rest system than it should be according to the classical theory}. \textelp The Einsteinian theory, as well as Lorentz's, differs from the classical theory in asserting a measurably different effect on rigid rods that \myemph{has nothing to do with the definition of simultaneity}}{Dal die genunnte Meinung falsch ist, erhellt sehon daraus, dal3 die Kontraktion des einen Apparatarmes gerade fiir das mitbewegte System eintritt, in dem der Apparat ruht. Die \scare{Einsteinsche Kontraktion} wiirde nut erklaren, dab der Arm verkiirzt wird, wenn er yon einem anderen System gemessen wird. Aber das wiirde zur Erklarung des M i c h e 1 s o n versuchs nicht geniigen. Denn dieser bewelst, dai] der in der Lingsrichtung der Bewegung ]iegende Stab, im R.Mtsystem gemessen, kiirzer ist, als er nach der klassisehen Theorie sein sollte. Wiirde es ein ausgezelchnetes Inertialsystem J geben, und hgtte man hierin zwei gleich lange s~arre Stgbe, yon denen der eine sigh naeh der klassischen Theorie, der andere nach der E i n s t e i n schen rlehten wiirde, so waren diese beiden Stabe, in ein Inertialsystem S gebracht, nlcht mehr gleieh lang, weun sie dort in der Lgsrichtung der Bewegung liegen; der Einsteinsehe Stab wgre kiirzer. Und zwar wiirde dleser Unterschied sowohl in S a]s Unterschied der ,Ruhlange", als aueh yon ~edem anderen Inertialsystem aus als Un~ersehied in der ,,Lgnge der beweg en Stgbe" gemessen werden. Es wird also in der Einstelnschen Theorie, genau so wie in der Lorentzschen, ein reel]bar anderes Verhalten der starren Stgbe als in der klassisehen Theorie behanpet, das mit der Glelchzeitigkeitsdefinition gar nichts zu tun hat}[\crc{43--44}{187--188\tm\me}]
%
Einstein contraction depends on the relativity of simultaneity and \qt{is related to the comparison of \emph{different magnitudes} within the \emph{same theory}}{zweier verschiedener Gr\"oßen, die derselben Theorie angehSren} \crc{44}{188}. A comparable case is annual parallax, the apparent displacement of a star observed from opposite sides of Earth's orbit. Lorentz contraction is related to \qt{the behavior of the \emph{same magnitudes} according to \emph{different theories}}{Diese vergleicht die Verhaltungsweisen derselben Gr\"oße, wie sie sieh nach verschiedenen Theorien ergeben} \crc{45}{188} (the classical and relativistic proper length). A analogous case is the difference in gravitational light deflection, where in \gr the deflection is twice as large as it would be according to Newtonian theory.

According to Reichenbach, only Lorentz contraction is at stake in the Michelson experiment, not Einstein contraction: \qt{It just happens that both contractions depend upon the same factor [\kappafactor], and this is probably the reason why they are always confused with one another}{Diese beiden Verkfirzungen haben zuf\"allig denselben Faktor und dies ist wohl der Grund, warum man sie immer verwechselt hat} \crc{46}{189}. The fact that Lorentz contraction and Einstein contraction amounts to the same factor, is in Reichenbach's assessment, a mathematical \emph{coincidence}, it is the consequence of the linearity of the transformation of the Lorentz transformations. However, one should not miss the deep conceptual difference between the two contractions, which is hidden behind the coincidental numerical equality of the two factors \rc{45f.}{189f.}. 

%Indeed, Reichenbach argues one can artificially construct cases in which in which there is no Lorentz contraction but an Einstein contraction \rzl{**}{**} and viceversa \rzl{**}{**}.\todo{check is movin donw}

% !TEX root = reichenbach_explanation.tex

%If bodies did not behave like $l$ in the relativistic sense, but like $L$ in the classical sense, there would be ne the Einstein contraction that $l$ is a rigid rod.

%


%Thus, the relativistic length of a moving rod from the rest frame is $\lK{}{'}$; the relativistic length of a stationary rod measured in the moving system is $\lK{'}{}$:
%\todo{The notation serves to highlight the relational nature of the notion of length}

%\begin{equation*}
%\lK{\notateol{'}{0.5}{\texts{measured in the moving frame}}}{\notateul{\textcolor{white}{'}}{0.5}{\texts{at rest in the rest frame}}} \quad \quad \LK{\notateor{\textcolor{white}{'}}{0.5}{\texts{measured in the rest frame}}}{\notateur{'}{0.5}{\texts{at rest in the moving frame}}} 
%\end{equation*}
%


In order to provide a proof of this claim, Reichenbach resorts to the somewhat idiosyncratic notation introduced in his \citeyear{Reichenbach1924} monograph. He labels $l$ a rod following the Lorentz–Einstein theory, and $L$ one with classical behavior. $K$ and $K'$ are resp.\ the stationary and moving systems. Then he uses index notation where the upper index $^{K}$ marks the frame in which the rod is measured, the lower the frame in which the rod is at rest $_{K}$. In both the classical and Lorentz–Einstein theories, the lengths of unit rods in the rest frame $K$ are equal, or $\lK{}{} = \LK{}{} = 1$. The difference emerges when one considers the length of the rod in the system in motion $K'$:

\begin{itemize}
\item \emph{classical theory}:  \hide{the rest length of the rod as measured from the moving system is equal to the rest length in the rest frame $\LK{'}{'}=\LK{}{}$
%
%\footnoteh{In classical mechanics $x' = x - vt$, then at the time $t = 0$, when $K$ and $K'$ coincide, we have $vt = 0$, and thus $x = x'$}
%
,} the rod has a unique length, regardless of motion:

\begin{equation}\label{eq:CT}
\frac{\LK{}{}}{\LK{'}{'}} = \frac{1}{1} \,\,\ \text{no contraction}
\end{equation}
%
\item \emph{Lorentz theory}: the length of the moving rod in the moving system $K'$, is \textit{contracted} \hide{$\lK{'}{'}<\LK{'}{'}$ by a factor \kappafactor} with respect to the length that it would have in the classical theory in the same frame $K'$:

\begin{equation}\label{eq:LT}
\frac{\lK{'}{'}}{\LK{'}{'}}= \frac{\kappafactor}{1} \,\,\ \text{Lorentz contraction}
\end{equation}

\item \emph{Einstein theory}:  the length of the moving rod measured in the rest system $K$ is contracted \hide{$\lK{}{'}<\lK{'}{'}$, by a factor \kappafactor} with respect to the length of the moving rod measured in the moving system system $K'$:

\begin{equation}\label{eq:ET}
\frac{\lK{}{'}}{\lK{'}{'}} =\frac{\kappafactor}{1}  \,\,\ \text{Einstein contraction}
\end{equation}
%
\end{itemize}
%
and viceversa\footnoteh{The length of the rest rod measured in the moving system $K'$ is contracted $\lK{'}{}<\lK{}{}$, by a factor \kappafactor with respect to the length of the moving rod measured in the moving system system $K'$: $\frac{\lK{'}{}}{\lK{}{}} =\frac{\kappafactor}{1}$.}. Reichenbach aims to prove that the fact that Lorentz contraction and Einstein contraction amounts to the same factor \kappafactor, is the consequence of the linearity of the Lorentz transformations. His proof runs as follows. According to classical theory we have $\LK{'}{}=\LK{}{}$, and since $\lK{}{}=\LK{}{}$ we obtain $\LK{}{'}=\lK{}{}$. Relation \cref{eq:ET} therefore becomes

\begin{equation}\label{eq:ETCT}
\frac{\LK{}{'}}{\lK{}{'}} = \kappafactor
\end{equation}
%
Because of the linearity of the transformation ratio \cref{eq:ETCT} is the same as ratio \cref{eq:LT}, which means that \cref{eq:ET} is also the same as ratio \cref{eq:LT}: the ratio is the same in all case and in particular is equal to \kappafactor.  However, Reichenbach insists that there is the deep \emph{conceptual difference} between the two contractions despite the their coincidental \emph{numerical equality} \rc{45f.}{189f.}.




%Indeed, shows that one can imagine cases in which in which there is no Lorentz contraction but an Einstein contraction\footnote{With not Lorentz contraction; the Einstein contraction from $K'$ to $K$ would also disappear. However, in the classical theory \cop{we might define the simultaneity in $K^{\prime}$ according to Einstein's convention, setting $\epsilon=\frac{1}{2}$, the inverse comparison from $K$ to $K^{\prime}$ will show the Einstein contraction}. The magnitude of which is the square of that of the Lorentz-Einstein contraction\todo{why?}.} and viceversa\todo{improve}.



\section{Two Kinds of Explanations: Deviation vs.\  Adjustment}
%
As we have seen, according to Reichenbach, Lorentz and Einstein theory agree on the physical content, as both assert an agreement between relativistic light geometry and matter geometry. Axiom VIII refers to rods: \s{Two space intervals which are equal when measured by rigid rods, are also light-geometrically equal} \ra{69}{89}. The difference between Lorentz and Einstein theory must be sought in the way they account for this set of physical facts. According to Reichenbach (1), Lorentz \emph{explains} this empirical fact by claiming that the moving rods and clocks, because of their motion through the ether, \emph{must have} a \emph{shorter length} than the rest \scare{ether} rod by a factor \kappafactor. Einstein \emph{stipulates} that the moving and rest rods have the \emph{same length}, despite the fact that the moving rod \emph{as measured in the rest system} is shorter by a factor \kappafactor than the rest rod. 



\emph{Both} theories, despite being empirically equivalent, are, according to Reichenbach, ultimately unsatisfying in accounting for the same set of facts: (1) Lorentz's theory is unsatisfying since it provides a \emph{bad explanation}, as there is no reason to assume that behavior of rigid rods must be classical. (2) Einstein's theory is unsatisfying because he provided \emph{no explanation} at all and simply declared by convention that relativistic rods are rigid. The superiority of Einstein's approach is that it frees us from the prejudice that the classical light geometry is more \s{natural}, so \rac that departing from the latter must mean \s{distortion}. Einstein shows that that we could just as good redefine rigid rod as those rods that they conform to relativistic light geometry. However, once this prejudice is overcome, the task is far from complete. 

\todo{Einstein (a) defines simultaneity by synchronizing clocks using the assumption of equal round-trip light speed; (b) defines the length of a moving object as the distance between its endpoints measured simultaneously in a given frame; (c) thus, it is not surprising that equal distances correspond to equal round-trip travel times of light in within each inertial frame}

The realization that what counts as a rigid rod is a matter of \emph{convention} should not lead us to circumvent the problem of \emph{explanation}. It should rather serve as a stepping stone for posing it in a different form. Unlike Lorentz, one does not need to explain why measuring rods and clocks \emph{disagree} with the classical light geometry. However, unlike Einstein, one must still explain why they all happen to \emph{agree} with the relativistic light geometry, and not with the classical one. Reichenbach suggests that the expression \emph{adjustment} \origg{Einstellung}, introduced by his nemesis Weyl, was effective in distinguishing this peculiar form of \s{causality} from the more traditional idea of a \emph{deflection} \origg{Abweichung}, on which Lorentz implicitly relied. Reichenbach already made this suggestion in his 1924 book, but it played somewhat marginal role \ra{70--71}{90-91}; in the 1925 paper the argument took the center stage. 

As Reichenbach explains, \citet{Weyl1920a} had introduced the expression to account for the surprising Riemannian behavior of physical systems, say a rock salt crystal, that we use as rods: they have always have the same length when compared side by side after being transported along different paths \citep[366]{Reichenbach1921}. This cannot be a coincidence. This fact suggests that, each time, they \scare{adjust} anew to a certain equilibrium value, rather than \scare{preserve} it. Indeed, the length of an ideal crystal i determined by the lattice constant, i.e., the fixed distance between atoms in the repeating unit cell\footnote{Following Weyl, Reichenbach used the following analogy: (1) the vertical orientation of the spinning top is maintained by perseverance \origg{Beharrung} due to the constancy of angular momentum. However, the spinning top easily \emph{loses} its vertical orientation if it is \emph{affected} by external forces \todo{Fogel}. (2) the vertical orientation of a wobble doll is maintained by adjustment \origg{Einstellung} because of a low center of mass. The wobble doll inevitably \emph{returns} to the original vertical orientation, if it is not \emph{hindered} by external forces.}. Thus, in Reichenbach's reading, the \textit{axiom} of Riemannian geometry—the empirical fact of the path independence of rod length—is \textit{explained} by a particular theory of matter which implies, say, the constancy of the lattice constant. This explanation does not aim to account for the deviation from a standard non-Riemannian geometry, but rather for the convergence of all rods to a non-trivial Riemannian one. Which among the Riemannian geometry happens to hold in reality is on the contrary a matter of \textit{convention}.

The analogy with \sr seems to be the following. A we have seen, the empirical content of Lorentz-Einstein theory  \qt{can be formulated as meaning that \emph{light geometry and matter geometry are identical}}{Der Gedanke dei \emph{Einsteins} l\"aßt sich dann dahin formulieren, \emph{daß Lichtgeometrie und K\"orpergeometrie identisch werden}} \cax{11}{14}. It is an odd coincidence that any physical system we use as a rod---whether it is made of steel, wood\etc whatever processes might have be gone true---always measures at equal lengths that are light-geometrically equal. As Reichenbach points out, \qt{[l]ight is a much simpler physical object than a material rod, and, when searching for a relation between the two, it should be initially supposed that it would not correspond to so ideal a scheme as the posited matter axioms}{Licht ist ein physikalisch sehr viel elnfacheres Gebilde als ela materieller Stab, und wenn man einen Zusammenhang zwischen beiden sucht~ sollte man zunichst annehmen, dab er nicht einem so idealen Schema entspricht, wie es die K\"orperaxiome behaupten} \crc{47-48}{95}. This coincidence cries out for an explanation. However, the explanation should not account for the \emph{divergence} from an alleged standard classical behavior, but a for the \emph{convergence} toward a non-trivial relativistic one:

\qt{The word adjustment, first used in this way by Weyl, is a very good characterization of the problem. \textelp All metrical relations between material objects, including the observed fact of the Michelson experiment, must therefore be explained in terms of the particular way in which rigid rods adjust to the movement of light. Of course, the answer can only arise from a detailed theory of matter about which we have not the least idea \textelp. The word \scare{adjustment} here thus only means a problem without providing an answer; the relevant fact is strictly formulated in the matter axioms without using the word \scare{adjustment}. Once we have this theory of matter, we can explain the metrical behavior of material objects; but at present the explanation from Einstein's theory is as poor as Lorentz's or the classical terminology}{Das Wort Einstellung, yon Weyl zum erstenmal in diesem Zusammenhang gebraucht, charakteris;ert das Problem sehr gut \textelp Alle metrischen Beziehungen zwisehen materiellen Gebilden mtissen so erklart werden, also anch der im Miehelsonversuch beobaehtete Tatbestand, wonach sich starre Stabe in bestimmter Weise auf die Lichtbewegung einstellen. Die Antwort kann natiirlich nur eine ausge~iihrte Theorie der Materle geben, yon der wir noch nicht die leiseste Vorstellung besitzen; Das Wort Einstellung deutet bier also nur auf eine Aufgabe bin, ohne selbst eine Antwort zu sein; der vorliegende Tatbestand ist ohne Benutzung des Wortes Einstellung in den K\"orperaxiomen streng formuliert. Wenn wlr dlese Theorie der Materie einmal besitzen, k5nnen wir das metrische Verhalten der materiellen GebJlde erkl~ren; vorerst aber kann yon einer Erklarung in der Einsteinschen Theorie so wenig die Rede sein wie in der Lorentzschen oder der klassisehen}[\crc{46--47}{191}]
%
According to Reichenbach, the difference between Lorentz and Einstein's theories is not in their empirical content they are indeed empirically equivalent.  Both assert the facts encoded in axiom \rom{8}, whereas the classical theory denies them. However, Lorentz's theory assumes the classical behavior of rods as \scare{self-evident}, so that any \emph{deviation} from the classical behavior must have a cause. Einstein's theory \emph{renounces} the explanation and axiomatically \emph{defines} two rods as equal if they behave in accordance with the \MME: \qt{The superiority of Einstein's theory lies in the recognition of the epistemological legitimacy of this procedure}{in dem Bewußtsein des erkenntnistheoretischen Rechtes hierzu liegt ihre \"{u}berlegenheit.} \crz{233}{202}. Indeed, Einstein's approach remove the prejudice, that the classical behavior of \rac is \apr correct. However, Einstein's agnosticism is unsatisfying, since it does not explain why \rac happen to behave relativistically. 

In Reichenbach's perspective the light axioms have a higher degrees of certainty since they contain only statements that are already accepted by classical optics, plus the definition of simultaneity. On the contrary, the matter axioms make \emph{new} statements about very complicated material structures, and are only partially verified. Einstein was right in challenging Lorentz's implicit view that the classical light geometry is the natural one, thus declared relativistic rods as rigid. However, one still need an explanation about why rigid rods do happen to behave according to a relativistic light geometry and not with another one. According to Reichenbach, without a suitable \emph{theory of matter} describing those physical systems we happen to use as rods and clocks, \q{Einstein's theory provide just as little an explanation \origins{Erkl\"arung}} of the metrical behavior of material objects \qt{as Lorentz's}{Einsteinsehe Theorie gibt hierfiir ebensowenig eine Erklarung wie die Lorentzsehe} \crc{43}{87}.

Once he had clarified the distinction between Lorentz contraction and Einstein contraction, Reichenbach was in a position to explore the implications of a hypothetical positive result from a Michelson-type experiment. As previously noted, this was far from a purely theoretical concern at the time. Just a few months before Reichenbach's paper appeared, Miller had published the results of his Mount Wilson experiments \citep{Miller1925}. Miller had been collaborating with Morley on ether drift detection since two decades earlier, and their joint efforts had previously yielded null results \citep{Morley1905,Morley1905a,Morley1907}. Miller's new findings quickly ignited significant discussion within the physics community. By the end of July, only a few weeks later, Reichenbach became the first \scare{philosopher} to attempt to engage in the debate—a debate he saw as a valuable chance to persuade skeptics of his axiomatic approach. His response underscores the extent to which the implications of Einstein’s experimental framework were still far from widely accepted:

\qt{Now we can also address the question what would change in the theory of relativity if Miller's experiment were held to prove that the hitherto negative result of the Michelson experiment is in principle wrong. \myemph{Nothing would change} in Einstein's theory of time as it has nothing to do with the Michelson experiment. Also nothing would change with the light geometry; it remains in any case a possible definition for the space-time metric and probably a much better and more accurate one than the geometry of rigid rods and natural clocks. \myemph{But what would change is our knowledge about the adjustments of material things to the light geometry}. With respect to the matter axioms, as far as they differ from the classical theory, the Michelson experiment is the only one that has been confirmed. If this should be refuted, one has to develop a more complex view of the relationship between material objects and the light geometry}{Jetzt kSnnen wir aueh die Frage beautworten, was sigh in der Relativit\"ats\"aheorie andern wiirde, wenn die Versuehe ~[ill e r s als Beweis angesehen werden mii~ten, da~ der bisherige negative Ausfall des ~[iehelsonversuehs nicht prinzipieH Iestgehalten werden darf. Nicht andern wfirde sieh die Einsteinsche Zeitlehre, sie hat mlt dem Miehelsonversuch gar nichts zu tun. Ni e h t ~ndern wiirde sieh auch die Lichtgeometrie; sle bleibt auf ieden Fall eine m~gllche Definition der raumzeitlichen Metrlk, und wahrseheinllch eine viel bessere und genauere als die Geometrie der starren St~be und natiirllchen Uhren. ~ndern aber wiirde sich unser Wissen fiber die Einstellung tier materiellen Gebilde auf die Lichtgeometrie. Von den K~rperaxiomen, sower sie sich yon denen der klasslschen Theorie unterseheiden, ist tier Michelsonversuch bisher als einziges best~tigt. F~llt dieses, so wird man sich ~ber den Zusammenhang der materiellen Gebilde mit der Liehtgeometrie eine verwlekeltere Auffassung bilden mfissen}[\crc{47}{192\me}]
%
In Reichenbach's axiomatization, the \MME is summarized in Axiom \rom{8}. Thus, in the event that Miller's experimental results were not spurious, only this axiom would change. The principle of \emph{constancy} of the velocity of light could be maintained; by adopting Einstein's conventional definition of simultaneity one could still claim that the motion of light is a spherical wave for any uniformly moving system. The assertion that that the velocity of light as a \emph{limiting} velocity could also be maintained. Indeed, this empirical fact is confirmed by measurements on fast traveling electrons ($\beta$ rays), whose kinetic energy approaches infinity as their velocity nears the speed of light. \todo{references}The assertion the velocity of light has always a \emph{numerical value} $c$, if measured by \rac on the contrary would be refuted. Indeed, a refutation of \MME would mean that rods do not adjust to the relativistic light geometry as Einstein expected. This would only mean that \qt{rigid rods do not after all possess the preferred properties that Einstein still attributes to them}{daß die starren K\"orper doch nicht jene einfachen Vorzugseigenschaften besitzen, die Einstein ihnen immer loch l\"aßt} \crc{328}{203}, that is they do not conform to the relativistic light geometry.



%In Reichenbach's view they might have the \qt{the validity of a first-order approximation in the same way that the ideal gas law cannot be maintained if the accuracy is increased}{nur die Geltung einer ersten Ann\"aherung, etwa wie die idealen Gasgesetze, die sich bei gr\"oßerer Genauigkeit auch nicht aufrecht halten lassen} \crc{48}{192}. \label{mc}


\section{Schlick-Einstein Correspondence on Reichenbach's Axiomatic} 
 
It is worth noting that Einstein drew very different conclusions from Miller's experiment. Two days before Reichenbach's paper was submitted to the \citejournal{Reichenbach1925}, on \datef{26}{6}{1925}, Edwin E.~Slosson, the frist director of the Science Service and editor of the popular magazine Science News-Letter, asked Einstein for a comment. Einstein's view at the time is clearly conveyed in a letter sent a few days later to Robert A.~Millikan, Caltech's \scare{chairman of the executive council}: if Miller's result were to be confirmed, Einstein wrote, then \qt{the whole theory of relativity would \myemph{go down like a house of cards}}{f\"allt die ganze Relativit\"atstheorie zusammen wie ein Kartenhaus} \lettercpaep{Einstein}{Millikan}{13}{7}{1925}[15][20\me]. A few days later, Einstein sent a very similar statement to Slosson, which was published in \jt{Science News-Letter} on \datef{8}{8}{1925} \citep{Einstein1925g}. The analogy of the \s{house of cards} seems to allude to the \s{theoretical rigidity} that Einstein demanded from a good theory: if any one of its conclusions proves to be false, the theory must be completely abandoned, since it is constructed in such a way that any modification would bring about its complete collapse \citep[see][]{Einstein1919}.

%\citep{Hentschel1992a} 
%


In this context, it is not surprising that Reichenbach's markedly different attitude toward Miller's results might have seemed puzzling. Indeed, at the end of 1925, Schlick expressed his surprise in correspondence with Einstein \letteraeap{Schlick}{Einstein}{26}{12}{1925}[21-591]\todo{check vol. 15}. \qt{Mr. Reichenbach}{Herr Reichenbach,}  he wrote \qt{has recently published a paper \citetitle{Reichenbach1925} in the \citejournal{Reichenbach1925}}{hat vor kurzem in der \citejournal{Reichenbach1925}  34, S. 32 eine Arbeit \citetitle{Reichenbach1925} publiziert}; Schlick was eager to know Einstein's opinion, since the paper \qt{quite clearly shows the limit of the axiomatic method}{weil sie ziemlich deutlich die Grenzen der axiomatischen Methode zu zeigen scheint} \letteraeap{Schlick}{Einstein}{26}{12}{1925}[21-591]\todo{??}. Schlick was understandably puzzled by Reichenbach's claim that the Lorentz contraction is not \latin{ad hoc}. After all, this had been Schlick's \s{conventionalist} reading of \sr for at least a decade \citep{Schlick1915}. He likely saw Reichenbach's remark as a barely concealed jab. Indeed, it is hard to deny that, in criticizing an unidentified mainstream interpretation of \sr, Reichenbach appears to have aimed at Schlick's own position. In fact, Schlick had recently defended this interpretation at the Leipzig meeting \GDNA, in which Reichenbach also participated \citep[60\psq]{Schlick1923}.

%Axiomatiken durch Zilsel in einem Brief an diesen vom 7. Mai 1925 nachdrücklich zu, vermerkte aber, daß trotz unterschiedlichster Motivation einzelne Axiome "fast wörtlich identisch" seien.

%, the Lorentz vs.\  Einstein debate could be framed as a classical case of theoretical underdetermination. 

According to Schlick’s conventionalist interpretation, the theories of Lorentz and Einstein are empirically indistinguishable. Lorentz accounts for the null results of experiments like Michelson–Morley by positing compensatory length contractions and time dilations, while Einstein’s framework dispenses with such auxiliary hypotheses, making it preferable on grounds of theoretical economy \citep{Schlick1915}. The decision between the two is thus guided not by considerations of truth but by simplicity. At first, Schlick and his associates seemed to interpret Reichenbach’s axiomatization as a more refined articulation of this conventionalist stance \citep{Zilsel1925a}. Yet Schlick admitted to Einstein—clearly with some disappointment—that he had come to understand Reichenbach was pursuing a fundamentally different line of thought:

%Hans Reichenbach an Edgar Zilsel, 7. 5. 1925 HR-016-24-07. Tvposkript-Durchschlag (3 Seiten) Edgar Zilsel an Hans Reichenbach, 22.5.1925  HR-016-24-06, Typoskript (2 Seiten) 

\qt{The reflection on p.~43 \textelp{} show in my opinion that his axiomatic cannot distinguish between special relativity and Lorentz's theory (with the contraction hypothesis) which seems to me obvious since the equations are the same. The real difference between the two theories is a philosophical one and cannot be grasped in the logical way of the axiomatic. This difference can be aptly expressed through the parlance that Reichenbach rejects. It is a an \latin{ad hoc} hypothesis. Even if, from a logical point of view, spec. rel. theory must make as many assumptions as Lorentz's one. In the first case they \textelp{} in the framework of and the contraction hypothesis is psychologically really not \latin{ad hoc}, while in the case of the Lorentz-Fitzgerald is a piece added \latin{ad hoc}.}{Die Ausf\"{u}hrungen S.~43 \textelp{} zeigen aber m.~E.\ nur, dass die seine Axiomatik zwischen der spez. Reltheorie und der Lorentzschen Theorie (mit der Kontraktionshypothese) \"{u}berhaupt keinen Unterschied finden kann, was mir selbstverst\"andlich erscheint, da die Gleichungen ja in beiden dieselben sind. Der wirkliche Unterschied zwischen beiden Theorien, der eben ein philosophischer und auf dem rein logischen Wege der Axiomatik nicht fassbar ist, wird wohl gerade durch die von Reichenbach verworfene Sprechweise, es handle sich bei Lorentz um eine \latin{ad hoc} ersonnene Hypothese, recht treffend angedeutet. Denn wenn auch, logisch gesprochen, die spez. Rel-theorie ebenso viele Grundannahmen machen muss, wie die Lorentzsche, so f\"{u}gen sie sich doch bei der ersteren ganz von selbst in den Rahmen des Relativit\"atsgedankens ein und die Kontraktionshypothese ist psychologisch tats\"achlich nicht \latin{ad hoc} ersonnen, w\"ahrend sie bei Lorentz-Fitzgerald als ein ad hoc angef\"{u}gtes St\"{u}ck auftritt.}[\lettercpaep{Schlick}{Einstein}{27}{12}{1925}[15][140]]
%
%As we have seen, for Reichenbach, Lorentz contraction is not \latin{ad hoc} at all, since both Lorentz and Einstein theory presuppose Lorentz contraction, that is the agreement between relativistic light and matter geometry. The key novelty of the theory was the introduction of Einstein contraction, which, however, has nothing to to with such agreement, it serves only as to Einstein could Lorentz and nevertheless consider rods as rigid thus avoiding getting bogged down in the shallows of Lorentz's dynamic explanations of the contraction. Schlick was indeed, right that the difference between their approaches emerge with particular clarity from Reichenbach's reaction to Miller's experiment. If the experiment were to be confirmed, Schlick argued, the universal conspiracy of nature that hide the aether from detection would be broken and we had to return to the ether theory, that is the Lorentz's theory without the contraction hypothesis. On the contrary, according to Reichenbach, no such return was required. In this sense, Schlick argued, Reichenbach's axiomatic method has no \s{physical consequences}:
%
As we have seen, Reichenbach did not consider Lorentz contraction to be \latin{ad hoc}, since both Lorentz's and Einstein's theories assume it—that is, they share the notion of a consistent geometry between light propagation and material objects. The real innovation introduced by Einstein's theory lay elsewhere: in the idea of Einstein contraction, which does not establish this geometric agreement. Instead, it allowed Einstein to adopt Lorentz’s formal results while still treating measuring rods as rigid, thereby sidestepping Lorentz's intricate dynamical account of contraction. Schlick rightly observed that the contrast between their interpretations is particularly evident in Reichenbach’s response to Miller’s experiment. If Miller’s findings had been verified, Schlick contended, the presumed natural \s{conspiracy} concealing the aether would have been exposed, and we would have had to revert to the ether theory—essentially Lorentz’s framework without the contraction hypothesis. Reichenbach, however, denied the necessity of such a reversal. 

In this sense, Schlick concluded, Reichenbach’s axiomatic approach lacks any \s{physical consequences}---in direct contradiction with the title of his paper:

\qt{Also the paper's remark---about the possible interpretation of Miller's experiments---does not seem to be to grasp the philosophical key point. If the experiments would really prove (and this is surely not the case), that a particular direction (that of the \scare{aether wind}) were privileged, one would certainly abandon the relativistic physics; even if it were possible to keep relativity through the assumption of certain \scare{matter axioms}, one would certainly not take this path. Against this the axiomatic consideration remains indifferent. In the strict sense one cannot speak of physical consequences of the axiomatic. The question seems to me philosophically relevant, I would be deeply grateful if you could tell me in a few lines if I'm right}{Auch die letzten Ausf\"{u}hrungen des Aufsatzes---\"{u}ber die m\"ogliche Interpretation der Millerschen Versuche---scheinen mir den philosophischen Kern der Sache nicht zu treffen. Wenn durch jene Versuche wirklich bewiesen w\"are (was ja gewiss nicht der Fall ist), dass eine bestimmte Richtung (die des \scare{Aetherwindes}) ausgezeichnet w\"are, so w\"{u}rde man gewiss die relativistische Physik aufgeben, und wenn es auch m\"oglich sein sollte, die Relativit\"at durch Annahme bestimmte \scare{K\"orperaxiome} aufrecht zu erhalten, so w\"{u}rde man doch diesen Weg nicht einschlagen. Aber hiergegen verh\"alt sich eben die axiomatische Betrachtung indifferent. Es scheint mir dabei, dass man daher in ganz strengem Sinne von physikalischer Konsequenzen der Axiomatik eigentlich doch nicht sprechen kann. Die Fragen scheinen mir philosophisch doch wichtig, und ich w\"are Ihnen von ganzen Herzen dankbar, wenn Sie mit einer Zeile mir sagen wollten, ob ich recht habe}[\lettercpaep{Schlick}{Einstein}{27}{12}{1925}[15][140]]
%
It is unlikely that Einstein ever read Reichenbach’s paper. Thus, it remains difficult to say definitively which position he would have endorsed. In earlier correspondence, Einstein had acknowledged that labeling Lorentz contraction as \emph{ad hoc} was misleading, in a way that resonates with Reichenbach’s own critique \lettercpaep{Einstein}{Lorentz}{23}{1}{1915}[8][47]. Still, with respect to Miller’s experiment, Einstein’s stance aligned more closely with Schlick than with Reichenbach. On \datef{19}{1}{1926}, a concise yet unambiguous statement by Einstein appeared in what was then Germany’s most influential newspaper, the \citejournal{Einstein1926a}: \q{If the results of Miller's experiments should indeed be confirmed, the relativity theory could not be upheld} \citep{Einstein1926a}. 

%Einstein was firmly convinced that Miller’s findings were likely flawed, and he even declared himself willing to stake money on that judgment, as he plainly put it \citep{Einstein1926a}. Yet he also made it clear that if Miller’s results were validated, the principle of relativity would have to be entirely relinquished \citep{Hentschel1992a}.

A few months later, Reichenbach submitted a popular article on Miller’s experiment titled \citefulltitle{Reichenbach1926a}, which was published in the weekly magazine \citejournal{Reichenbach1926a} on \datemy{24}{4}{1926}. By then, he was fully aware of what \qt{Einstein himself has recently said in the newspapers,}{Einstein selbst hat kürzlich in den Tageszeitungen ausgesprochen}; nevertheless, he saw no reason to revise his \qt{less radical opinion}{eine weniger radikale Ansicht entwickelt} \crw{327}{202}, namely that \qt{\myemph{Miller’s result in no way affects the philosophical consequences of the theory of relativity}}{Auf keinen Fall werden jedoch die philosophischen Konsequenzen der Relativitätstheorie von den Millerschen Versuchen betroffen} \crw{328}{203\me}. For Reichenbach, the outcome would merely require a change in our understanding of the physical processes that govern rods and clocks.

%\citep[308]{Hentschel1982}.

%A few months later, Reichenbach submitted a popular paper on the Miller's experiment entitled \citefulltitle{Reichenbach1926a} which appeared in the weekly magazine \citejournal{Reichenbach1926a} in \datemy{24}{4}{1926}. By that time, Reichenbach was fully aware of what \qt{Einstein himself has recently said in the newspapers,}{Einstein selbst hat k\"{u}rzlich in den Tageszeitungen ausgesprochen}; however, he saw no reason to abandon his \qt{less radical opinion}{eine weniger radikale Ansicht entwickelt} \crw{327}{202}, namely that \qt{\myemph{Miller's result in no way affects the philosophical consequences of the theory of relativity}}{Auf keinen fall werden jedoch die philosophischen Konsequenzen der Relativit\"atstheorie von den Millerschen Versuchen betroffen} \crw{328}{203\me}. It would only imply a change in our knowledge of the physical mechanism governing rods and clocks \citep[308]{Hentschel1982}.

Reichenbach expressed some doubts about the correctness of Miller's experiment. However, the philosophical point lied elsewhere: \qt{\lse{What then does the theory of relativity have to infer from Miller's experiment?}}{\s{Was hat nun die Relativit\"atstheorie aus dem Millerschen Versuch zu schließen?}} \crw{327}{202}. On this matter, Reichenbach did not hesitate to voice a view that sharply diverged from those of both Einstein and Schlick. Somehow anticipating his later famous distinction between the context of discovery and the context of justification, Reichenbach claimed that \qt{\textins{t}he Michelson experiment, of course, played a crucial role in the \myemph{historical development} of the theory}{Zwar hat der Michelson-Versuch in der historischen Entwicklung der Theorie eine entscheidende Rolle gespielt} \crw{327}{202\me};  however, according to Reichenbach, \qt{it does not occupy this same significant place in the relativistic theory's \myemph{logical structure}}{Aber im logischen System der Relativit\"atstheorie kommt ihm nicht dieselbe f\"{u}hrende Stellung zu} \crw{327}{202\me}. Logical structure of the theory was of course captured by his own axiomatic formulation:

%\footnote{This claim is actually problematic \cites[cf., e.g.,][]{Stachel1982}{Dongen2009-07}}

\qt{Under the ten axioms of the theory of relativity as I have laid them out, i.e., its ten most basic empirical propositions, there is only one that entails the Michelson result; it is only this axiom then that is thereby threatened. The principle of the constancy of the speed of light could be maintained in a more limited form even if the Michelson experiment's negative result were overturned. One could construct a \scare{light geometry} using light signals but employing no rigid rods to maintain a metrical understanding of the world and allow the previous formulation of all physical laws. From this perspective, the Michelson experiment serves only as a bridge between the light geometry and the geometry of rigid rods. Should this connection be lost, this would only mean that rigid rods do not after all possess the preferred properties that Einstein still attributes to them. This would not mean a return to the old aether theory, but rather a step towards the renunciation of a preferred system of measurement in nature}{Unter den vom Verfasser aufgestellten zehn Axiomen der Relativit\"atstheorie, d. h. ihren zehn obersten Erfahrungss\"atzen, enth\"alt nur ein einziges die Behauptung des Michelson-Versuches; nur dieses Axiom w\"are also ersch\"uttert. Das Prinzip der Konstanz der Lichtgeschwindigkeit l\"aßt sich in eingeschr\"ankter Form auch noch festhalten, wenn der Michelson-Versuch nicht negativ ausf\"allt. Man kann, indem man ohne Benutzung starrer Maßst\"abe nur mit Lichtstrahlen arbeitet, eine "L chtgeometrie" konstruieren, di,e eine metrische Erfassung der Welt leistet und die Formulierung aller physi kalis ehen Gesetze bereits gestattet. Von diesem Gesichtspunkt gesehen kommt dem MichelsonVersuch nur die Rolle eines Verbindungsgliedes zwischen Lichtgeometrie und Geometrie der star}[\crw{327}{203}]
%
As we have seen, even if the aether-drift experiments yielded a positive outcome, the geometry of light in relativity could still be preserved, as it rests on a matter of \emph{definition}. What would be challenged instead is a material axiom \rom{8}—namely, the assumption that rods behave according to the relativistic light geometry. A positive result of a Michelson-like experiments would suggest that rods instead follow the classical geometry. Unlike Schlick, however, Reichenbach explicitly rejects the idea that this would entail a revival of the ether theory. The distinction between classical and relativistic light geometries would continue to depend on the conventioal definition of simultaneity. There would be no return to pre-relativistic physics.

Schlick seems to have missed Reichenbach’s point, which is admittedly somewhat contrived. Reichenbach did not dispute that Miller’s result would impact the \s{physical theory of relativity}—the empirical claim that ideal \rac follow relativistic light geometry could no longer be upheld; this is a factual matter. His point, rather, is that it would not undermine the \s{philosophical theory of relativity}, that is, Einstein’s insight that simultaneity is not an empirical fact but a matter of definition. Einstein had shown that determining the simultaneity of distant events requires knowing the speed of light, but measuring the speed of light presupposes knowledge of simultaneity. This circularity implies that simultaneity of distant events cannot be known in principle. For Reichenbach, Einstein’s realization that time is not absolute is---like the diamonds in the old advertisement---forever: it \q{is independent of specific, physical observations} \crw{203}{328}. While this insight arose \q{from a particular physical theory}, namely \sr, it has \q{given rise to philosophical insights which no longer belong to the realm of physics but rather to the philosophy of nature} \crw{204}{328}.


%by adopting Einstein's conventional definition of simultaneity one could still claim that the motion of light is a spherical wave for any uniformly moving system. Also the assertion that that the velocity of light as a \emph{limiting} velocity could also be maintained. Indeed, this empirical fact is confirmed by measurements on fast traveling electrons ($\beta$ rays), whose kinetic energy approaches infinity as their velocity nears the speed of light. The assertion the velocity of light has always a \emph{numerical value} $c$, if measured by \rac on the contrary would be refuted.


\hide{We have now succeeded in distinguishing between the physical assertions of the relativistic theory of space and time and its epistemological foundation. This epistemological foundation is supplied by the discovery that coordinative definitions are needed far more ... The physical core of the theory, however, consists of the hypothesis that natural measuring instruments follow coordinative definitions different from those assumed in the classical theory. This statement is, of course, empirical. On its truth depends only the p11ysic1tl theory of relativity. However, the philosophical theory of relativity, i.e., the discovery of the definitional character of the metric in all its details, holds independently of experience. Although it was developed in connection with physical experiments, it constitutes a philosophical result not subject to the criticism of the individual sciences.}
%
%\begin{figure} \begin{center} \includegraphics[scale=0.33, trim = 0mm 0mm 0mm 0mm, clip]{graph/mr2} \caption{Realization of the indefinite metric by means of light rays, clocks and measuring rods; from \cite[215]{Reichenbach1928}; the label $S_2$ has been added} \label{mr} \end{center} \end{figure}

\begin{figure}
\begin{center}
\includegraphics[scale=0.6, trim=0mm 0mm 0mm 0mm, clip]{graph/mrmod-}
\caption{From \cite[215]{Reichenbach1928}, slightly modified; $S_{2}$ has been added. Minkowski's four-dimensional manifold is one in which the distance between two points is given by $ds^{2} = dx_{1}^{2} + dx_{2}^{2} + dx_{3}^{2} - dx_{4}^{2}$. The Lorentz transformation results from the invariance of $ds$.\todo{check} This \emph{mathematical structure} can be \emph{physically realized} in two ways: (a) \rac, where $ds > 0$ implies rods, $-ds < 0$ implies clocks, and $ds = 0$ implies light rays; (b) by lines on a Minkowski diagram, where $ds > 0$ corresponds to the horizontal axis, $ds < 0$ to the vertical axis, and $ds = 0$ to a dotted line tilted at 45°. Since (a) and (b) are both realizations of $A$, (a) can be \emph{graphically represented} by (b). $OQ$ is the time axis and represents a clock at rest, while the horizontal axis $OS$ represents a rod at rest in $K$. The right hyperbola represents the locus of points from which a light round-trip yields the same spatial separation; the upper hyperbola represents the locus of points from which a light round-trip takes the same total time on the origin clock. Einstein's relativity of simultaneity is shown diagrammatically by the fact that both the time and space axes can be rotated. The rotation of $OQ$ represents a clock in motion with respect to $K$, while the rotation of $OS$ represents a rod in motion with respect to $K$. The agreement between light geometry and matter geometry is symbolized by the fact that $Q$ and $S$ follow the contours of the hyperbolas.}
\label{mr}
\end{center}
\end{figure}


\section{Dynamical vs.\  Geometrical Explanation in the \citetitle{Reichenbach1928}}
%
There is no evidence that Reichenbach ever became aware of Schlick’s critical stance toward his interpretation of \sr. An entry in Rudolf Carnap’s diary recording a meeting with Reichenbach near Berlin on \datef{2}{9}{1926} indicates that Reichenbach insisted to place great importance on the distinction between the two types of contraction: \qt{He explained me the difference between Lorentz and Einstein contractions}{Er erkl\"art mir den Unterschied zwischen Lorentz- und Einstein-Verk\"{u}rzung} (RC 025-72-05). Indeed, Reichenbach’s line of reasoning reappears in the \citetitle{Reichenbach1928} \citep{Reichenbach1928}, which, according to a letter from Reichenbach to Schlick, had already been completed by the end of 1926 (\letter{Reichenbach}{Schlick}{6}{12}{1926}[][SN]). By the time the book appeared in early 1928, the relevance of Miller’s experiment was already waning, especially in Germany. Reichenbach could now state that the \qt{Michelson experiment has been confirmed to a very high degree}{} and regarded \qt{this matter closed}{} \rzl{225}{195}. Einstein had expressed a similar view the year before \citep{Einstein1927d}. Nonetheless, Reichenbach still found it necessary to address the \qt{erroneous interpretations in the usual discussions on relativity}{} that had surfaced in the context of Miller’s experiment \crz{225}{195}. 

In §31, Reichenbach restates the position he had advanced in his 1925 paper, with little or no revision. However, the book's presentation introduces a novel element, as it frames the difference between Lorentz and Einstein contraction in geometrical terms—that is, in terms of \Mink diagrams. The name of \Mink is, surprisingly, never mentioned in his 1924 book. This is ultimately not simply an oversight. One might perhaps expect that Reichenbach saw in \Mink's \s{geometrical explanation} the replacement of that \s{dynamical explanation} that Einstein had failed to provide. However, this is not the case. Reichenbach seems to believe that \Mink's geometrical-diagrammatic presentation of Einstein's kinematics does not add anything substantial to the theory. In Reichenbach's \Mink-\spti, was nothing but a \scare{\emph{graphical representation}} of the agreement between light and matter geometry. 


%The control of natural phenomena is achieved by means of mathematical concepts. These concepts arc defined by implicit definitions and are not dependent on a unique and specific kind of visualization.

%\subsection{Dynamical vs.\  Geometrical Explanation Special Relativity}

Reichenbach borrows the expression \scare{\emph{graphical representation}} from Arthur Stanley \citet{Eddington1925a}. However, in section \S15 of the book, he takes care to provide a more precise definition. Reichenbach defines \scare{graphical representations} as structural analogies between different physical systems\todo{??}: Since same \emph{mathematical system} $A$ can be \emph{physically realized} by different systems $a, b, c, \ldots$, one can use, say the system $b$ to \emph{graphically represented} the system $a$\todo{reference}. Reichenbach makes several examples, but for the sake of brevity, let's apply this definition to our case.  \Mink presented the dependence of the measurement of space upon simultaneity, in the combining of space and time into a unique mathematical structure $A$, a four-dimensional manifold with an indefinite metric\footnote{See the caption to \cref{mr} for more detail}. This structure $A$ can be physically realized by the system ${a}$ (rods, clocks, light rays), the system ${b}$ (the lines of a \Mink diagram as in \cref{mr}), and possibly by many other systems ${c}, {d}, \cdots$ as well. For this reason, one can also \emph{graphically represent} $a$ through $b$. In other terms, while the physical realization is a \emph{vertical coordination} of a mathematical system with a different physical systems, the \emph{graphical representation} is a horizontal coordination among different physical systems that realize the same mathematical structure.

%It is because of this analogy that we can \emph{graphically represent} $a$ via $b$, if the latter are easier to visualize. E.g. the same Euclidean geometry $A$—that is, an $n$-dimensional $ds^{2} = \sum_{\nu} dx_{\nu}^{2}$—can represent: ($a$) rigid rods measuring $ds$, ($b$) thermodynamic relations, ($d$) electric pipes, etc. The graphical representation of, say, thermodynamic relationships $b$ in a $PV$ Euclidean diagram $c$ is based on the fact that both are realizations of the same system $A$ of Euclidean geometry. In other temerm while usally of a vertical coordiantes, Eucldae with \rac (a); the graphcial rpreatiat is a horizioanl coordiantion.\todo{improve}

Of course, the mathematical content of \sr is not altered simply because we have visualized it using a \Mink diagram: \qt{we only give a graphical representation, which means that the logical structure \origins{Beziehungsgef\"{u}ge}}{so wird sie nur graphisch dargestellt; es wird damit behauptet, dass das Beziehungsgef\"{u}ge, welches r\"aumliche St\"abe von der Art des vorangehenden \S~28 enthalten, zugleich auch; von der Raum-Zeit-Mangfaltigkeit realisiert wird.} \rzl{220}{190} exhibited by rods, clocks, and light rays is also presented through the relations among the lines in the \Mink diagram of \cref{mv}. Reichenbach concludes with a passing, yet significant remark: if, after \Mink's work, we \qt{speak of a \myemph{geometrization} of physical events, this phrase should not be understood in some mysterious sense; it refers to the identity of types of \emph{structure} and not to the \emph{identity of the coordinated physical elements}}{Wenn man von einer Geometrisierung des Weltgeschehens gesprochen hat, so darf dies auf keinen Fall in irgendeinem geheimnisvollen Sinn aufgefaßt werden; es besagt nur die Identitat von Strukturtypen, nicht der zugeordneten dinglichen Elemente.} \rzl{220}{190\me}.  \Mink's approach serve as a \emph{geometrical illustration} of the relativistic behavior of \rac, not as an \emph{geometrical explanation}.

An example of such an illustration is Reichenbach's use of a \Mink diagram \cref{mr} to illustrate the difference between Lorentz and Einstein contraction. The length of a rod in the rest system $K$, as measured with unit rods at rest, $\LK{}{}=\lK{}{}$, is represented by $OS$ in the diagram. The various states of motion of the same rod are represented by rotations of $OS$ about the point $O$. The relativistic behavior of the length $l$ is represented by the fact that $S$ lies on the right hyperbola, which graphically depicts the locus of points from which a light round-trip yields the same spatial separation from $O$. The classical behavior of the length $L$ is represented by the fact that distance measured by the round-trip speed of light is shorter that measure by material rods, that is $OS_{1}< OS^{\prime}_{2}$. The difference between the two contractions can be then illustrated as follows with reference to \cref{mr}:

\begin{itemize}
\item Lorentz contraction: the rest-length $OS'=\lK{\prime}{\prime}$ of the moving rods as measured in the moving frame $K'$ is shorter than the rest-length $OS^{\prime}_2=\LK{\prime}{\prime}$ that the rod would have according to the classical theory. This an \emph{objective difference} on which both Einstein and Lorentz agree. It is represented graphically by the fact that the relativistic theory \cop{locates the right-hand boundary, parallel to $OQ^{\prime}$, of the moving rod with its left boundary at $O$, along the tangent to the hyperbola that passes through $S^{\prime}$; the classical theory asserts that the boundary must pass through $S$}:

\item Einstein contraction: the moving bar $OS^{1}$, as measured in the rest frame $OS^{\prime} = \lK{\prime}{\prime}$, is shorter than its rest length in the co-moving frame $OS_{1} = \lK{\prime}{\prime}$. This is a \textit{perspectival difference} that depend on the choice of the reference frame. It is represented graphically by the fact that length of the \textit{projection} $OS^{\prime}_{1}$ of the moving rod onto the horizontal axis is shorter than $OS$. Indeed, with respect to the co-moving frame $K'$, the endpoints $O$ and $S^{\prime}$ of the moving bar are simultaneous; with respect to the rest frame $K$, $O$ and $S_{1}$ are. Note that the same rod $OS^{\prime}$ projects differently onto tilted space axes, yielding different contractions in different frames.

%\footnote{Needless to say, a moving system $K^{\prime\prime}$, being slower, would correspond to a line that is less tilted, while a faster one $K^{\prime\prime\prime}$ would be more tilted than $K'$ in the given frame. }. \todo{improve}

%The effect is perspectival: $OS_{1}$ is the projection of $OS^{\prime}$ onto the horizontal axis \rzl{220}{190}. If one place a rod along $OS^{\prime}$ and the the \emph{same} rod $OS_{1}$, one discover $OS_{1} < OS^{\prime}$.

%\cop{locate the right-hand boundary , parallel to $OQ^{\prime}$ of the rod along the tangent to the hyperbola that passes through $S^{\prime}$; classical theory asserts that the boundary must pass through $S$}.

%The effect is reciprocal: the rest length $OS = \lK{}{}$, as seen from the moving frame, appears as the projection $O^{1}_{3} = \lK{'}{}$, so $O^{1}_{3} < OS$. For co-moving observers, both effects vanish: $OS = OS^{\prime}$ when at relative rest.


%That is if measured if $OS = OS_{1}$ if they are compared at relative rest: $\lK{\prime}{\prime} = \lK{}{}$.

\end{itemize}
%
%
%; the length as measured by a rod at rest in the rest system $K$ ($\lK{\prime}{}$) is \q{symbolized} by
%
%}\footnote{That is, \sr makes the specific prediction that $OS_{1} < OS$, whereas the classical theory would predict that they have equal length; indeed, that if one cacept Einstein defin $\epsilon$ definiton of   simultaneity this a specific prrection can be true or flase.}
%
\begin{inparaenum}[(a)] As we have seen, \item the classical theory entails neither type of contraction: \emph{rigid} rods behave like $L$. The theory is refuted by \MME. \item Lorentz theory includes only the Lorentz contraction: \emph{distorted} rods behave like $l$. \item Einstein’s theory includes both contractions: \emph{rigid} rods behave like $l$ \end{inparaenum}. Einstein simply declared that the moving rod retains its length in its own frame, despite appearing shorter when measured in the stationary frame. Einstein achieved this result by introducing \emph{by convention} a new definition of length as a frame-dependent quantity: 

%the length of a moving rod is the distance between its endpoints measured simultaneously in a particular frame (in which, of course, simultaneity is defined by clock synchronization using light signals)\todo{improve}:

%It was one of Einstein's philosophical insights to realize that we are free to call the length of the moving rod in the moving coordinate system one if we wish, despite its having a coordinate difference of only $\sqrt{ }\left(1-v^2-c^2-\right)$. If we make this decision, then while we

%It was one of 

%

%In discussing the Lorentz contraction we did not talk about the length of the rod in the moving system, but merely about its coordinates in a particular coordinate system. It was one of Einstein's philosophical insights to realize that we are free to call the length of the moving rod in the moving coordinate system one if we wish, despite its having a coordinate difference of only / (1 - 02/c?). If we make this decision, then while we


%The difference between the two contractions is clear: Lorentz contraction means that the world-strip of the relativistic length is shorter than that of the classical length in the same system $K'$. On this both, theories agrees. Einstein’s contraction refers to the fact that the same shorter strip can be in a different way by the plane of simultaneity, that is by horizontal space axis of \Mink diagram if the rod $K'$ or $K$: 

%Einstein could achieve this result, by introducing \emph{by convetion} a new definition of length, which requires measurement at the same time. In this setting, identical rigid rods, measured the distance between different points: with respect to the co-moving frame $K'$  the end points $O$ and $S^{\prime}$ of a moving bar are simultaneous; with respect to rest frame $K$, $O$ and $S_{1}$ are:

%

%A lightlike line in a Minkowski diagram has equal projections onto the space and time axes when using units where \( c = 1 \). Rods and clocks at rest in a moving system measure the same speed of light as those at rest in the original frame. Even when lengths and times are projected differently due to relative motion, the ratio of distance to time—used to measure the speed of light—remains \( c \).


\qt{[T]he dependence of the length of the moving segment on the definition of simultaneity \textelp{makes} particularly clear that the Einstein contraction is a metrogenic phenomenon. In the geometrical representation this means that we may choose as the length of the rod differently directed sections through the world-strip of the rod. On the other hand, the geometrical representation of [\cref{mr}] shows very clearly that through the difference in the width of the strip, the Lorentz contraction indicates a difference in the actual behavior of the rod. These considerations also explain how it is possible to compare rods $l$ and $L$, although only one of them is physically realized. $OS$ is the same in both theories; the classical theory claims that the right-hand boundary of the strip parallel to $OQ'$ must be drawn through $S$, whereas the new theory places the boundary along the tangent to the hyperbola which passes through $S'$}{Gerade dieses Beispiel \textelp macht es besonders deutlich, daß die EinsteinVerk\"{u}rzung eine metrogene Erscheinung ist; es kommt in der geometrischen Darstellung darauf' hinaus, daß man als L\"ange des Stabes verschieden gerichtete Schnitte durch den Weltstreifen des Stabes ausw\"ahlt. Andrerseits zeigt die geometrische Darstellung der Fig. 32 (S. 215) deutlieh, daß die LorentzVerk\"{u}rzung mit dem Unterschied der Streifenbreiten einen Unterschied des realen Verhaltens betrifft. Auch erkennt man hier, wie es \"{u}berhaupt m?glich ist, die St\"abe $l$ und $L$ zu vergleichen, obgleich nur der eine von ihnen realisiert ist: die Strecke OS ist f\"{u}r beide Theorien dieselbe; die alte Theorie behauptet, daß die rechte, zu $OQ'$ parallele Begrenzung des Streifens durch S gezogen werden muß, w\"ahrend die neue Theorie behauptet, daß diese Begrenzung als Tangente an die durch $S$ gehende Hyperbel gezogen werden muß}[\rzl{232}{200}]
%
Einstein contraction compares the same rod \emph{with respect to different reference frame within the same theory}. Lorentz contraction, by contrast, compares the same rod \emph{in the same frame with respect to different theories}. The term \s{contraction} is in both cases somewhat misleading: Einstein's \s{contraction} does not imply any physical change, whereas Lorentz's \s{contraction} is only a counterfactual change\todo{improve} \rzl{**}{197}. 

%Indeed, with the same rod, we measure different things. This is a perspectival difference between the length of the moving rod in the moving frame $K'$ and of its projection in the rest frame $K$.  
%
%\cop{It has been objected to previous remarks of mine on this subject ${ }^1$ that it is impossible to compare two magnitudes belonging to different theories. This objection is incorrect. By reference to a third body, we are able to establish a comparison, if we calculate how the two magnitudes under consideration compare with that third body. Furthermore, this mode of expression is frequently used in physics. We may say, for instance, that a highly-compressed gas behaves differently than it would according to Mariotte's law. This means nothing but that the real gas $g$, when compressed to a certain degree, occupies a larger volume than a gas $G$ which satisfies the MariotteBoyle relations. The third body used in this comparison is the rigid}. \cop{Measuring rod which measures the volume. The third body is not always explicitly mentioned, and the abbreviated formulation is often preferred, because it clearly suggests a difference in actual behavior. The Lorentz contraction must indeed be considered a real difference in this sense. In this case, the tertium comparationis is light, which in terms of light-geometrical definitions supplies a standard to which the rods of the different theories can be compared. }. 
%
%In this case, the fertium comparationis is light, which in terms of light-geometrical definitions supplies a standard to which the rods of the different theories can be compared. It would be an

%It would be an incorrect mode of speech, however, to say that the Einstein contraction is an apparent difference. This contraction has nothing to do with the difference between the real and the apparent, but results from a difference in the conditions of measurement. We shall speak of a metrogenic difference because this difference originates in the nature of the measurement. Since we are specifically concerned with kinematic

Reichenbach concluded his reconstruction by arguing against not better specified objection he had received since the publication of his \citeyear{Reichenbach1925} article discussed above. In particular, it was argued that it is impossible to compare two magnitudes belonging to \emph{different} theories. However, Reichenbach has no difficulty to retort that we actually do this all the time. We may say, for instance, that a highly compressed gas behaves differently than \emph{it would have} according to Mariotte's law for ideal gases—that is, an ideal gas occupies a smaller volume than a real gas would have occupied in the same conditions. The comparison is possible because the Euclidean rigid rods used to measure the gas volume serve as a \emph{tertium comparationis} \todo{reference}. Also in special relativity, Reichenbach pointed out, we have a \emph{tertium comparationis}: light-geometrical definitions supply a standard to which the rods of the different theories can be compared. The difference between the classical theory and the Lorentz-Einstein theory is then an objective fact. The problem is to grasp the difference between Lorentz and Einstein theories:

\begin{enumerate}[label=(a)]
\item  \cop{the rest-length of the moving rod is shorter from the rest-length of a moving rod rod which that satisfies the classical theory} (Lorentz-Einstein) 
\item  \cop{the rest-length of the moving rod is shorter from the rest-length of the rod at rest} (Lorentz). 
\item the rest-length of the moving rod is equal to the rest-length of the rod at rest (Einstein)
\end{enumerate}
%
Statement (a) is correct for all theories, thus Lorentz contraction is not \emph{ad hoc}. From (a), Lorentz infers (b), that it must be shorter than its true classical length; thus he requires a dynamical explanation to account for the deformation of the from the default classical rigid rods. Einstein redefines the concept of length declares by convention that (a) is the default behavior of rigid rods (c). This move is important, since it removes Lorentz’s crypto-Kantian prejudice that classical behavior must be \apr true in some sense. Einstein could wash his hands and declared a dynamical explanation of the contraction unnecessary. However, Reichenbach felt that this trick was ultimately unsatisfactory. 

According to Reichenbach, it is correct to claim that Einstein contraction \emph{does not} require any explanation. Indeed, this contraction is introduced precisely in order to redefine relativistic rods as rigid. However, Lorentz contraction \emph{does} cry out for such an explanation. The negative result of the Michelson experiment implies that rods of all materials \emph{happen to} invariably behave in agreement with distances measured by round-trip light signals. How can such a \emph{coincidence} be explained? \Mink's alleged \emph{geometrical explanation}, in Reichenbach's view is no explanation at all. In reality, it is merely a graphical representation of Einstein's move. Lorentz's \emph{dynamical explanation} was unsatisfying since it presupposes the classical behavior of rods as natural. Reichenbach concluded that the way out of this conundrum came from a proper analysis of what counts as a \emph{dynamical explanation}. 

The word \s{contraction} has unfortunately led \cop{a mistaken application of the principle of causality} \rzl{**}{**}. Lorentz treated the classical theory as the default and sought an explanation for the \emph{deviation} from it. Einstein tried to avoid the problem of explanation altogether declaring the relativistic behavior as natural. However, why the special relativistic behavior is natural, not a different one? According to Reichenbach, the search for a dynamical explanation is legitimate, only \q{to be posed in a different form} \rzl{**}{**}. Once again, Reichenbach resorts to Weyl's expression \emph{adjustment} to outline the contour of kind of explanation he was searching for, but he provides no details as to what it would look like: \qt{The answer can of course be given only by a detailed theory of matter, of which we have not the least idea}{Die Antwort kann nat\"{u}rlich nur eine ausgef\"{u}hrte Theorie der Materie geben, von der wir noch nicht die leiseste Vorstellung besitzen} \crz{233}{201\tm}.


%\section{Dynamica vs.\  Geometrical Explanation in General Relativity}

%% !TEX root = reichenbach_explanation.tex

\subsection{Reichenbach Dynamical Interpretation of General Relativity}

That Reichenabch was ultaly, amrginal. Hwove,r that consdierht ethe case. To this purpuse some analitycal elemth should be tinodee. It is important to emphasize that this is not a marginal aspect of Reichenbach's philosophy. According to Reichenbach the very same problem emerges in \gr when the non-Euclidean nature of the continuum is taken into account. In this case, too, he resorts to the expression \scare{adjustment} and he refers his readers to the very same \S31 of the book. The insistence on the problem of explanation must bot be taken as. Indeed, it becomes the same role of \gr. 


As we have seen, because, Einstein redefined the length of simultaneity that it became advantageous to combine space-time into a single geometrical structure, that is a four-dimensional manifold, that \Mink called the \scare{World} or \st (\S24). This simply means that it takes four numbers, the so-called \emph{coordinates}, $x_\nu$ (where $\nu=1,2,3,4$) to identify a \wpo, namely three numbers $x_1,x_2,x_3$ for its spatial location and one for time $x_4$. The set of points whose coordinates are defined by $x_\nu(s)$ where $s$ is an arbitrary parameter is called a \wl{}. At this stage, coordinates $x_\nu$ are nothing but identification numbers, that is, in Reichenbach's parlance, they have only a \scare{topological} function; they determine the order of the \emph{between} relation. In order to define distances between points, one needs to introduce an expression that assigns a number $ds$ to the coordinate differences $dx_\nu$ between two close \wpo{}s, the so-called \scare{fundamental metrical form}. In the case of \Mink \spti, it is always possible to choose the coordinate numbers \xn so that any distance satisfies the relation.
%
That the behavior of small \rac and satisfy the following relations:
%
\begin{equation}\label{eq:mink}
\diff s^{2} = \diff x_{1}^{2}+\diff x_{2}^{2}+\diff x_{3}^{2}-\diff x_{4}^{2}\,.
\end{equation}
%%
This metrical (that is \scare{measurement}) formula allows to calculate the distance $ds$ between two nearby points from their coordinate differences $dx_\nu$. A metric that has positive as well as negative signs is called \scare{indefinite} \rzlp{**}{188--189}. The physical realization of the negative $\diff s^2$ is a physical object that satisfies the relations of congruence defined by the hyperbolas of quadrants \rom{1} and \rom{2}. The realization of the positive $\diff s^2$ is a physical object that satisfies the relations of congruence defined by the hyperbolas of quadrants \rom{3} and \rom{4}. The first is called a time-like interval $\diff s^2 = -1$ and is realized by the proper time of a clock. $\diff s^2 = 1$ is the space-like interval and is realized by the proper length of a rod. Light rays realize $\diff s^2 = 0$, the limiting velocity, which cannot be reached but only approached arbitrarily closely. Otherwise rods and clocks behave by following the hyperbolic contour lines.

As is well-konw pecial realtivity, that Lorentsz recatug coordiantes, invariance of \cref{eq:mink} that  itself in the existence of a group of transformations which leave invariant the four-dimensional distance or interval between two points. In \Mink, space-time one can of also introduce a non-rectangular coordinate system, for example polar, cylindrical coordinates\etc, or arbitrary curvilinear coordinates. If the topological relations remain unchanged, the information about the reciprocal distances among any pair of points is lost. In order to recover the distance between two points in the new coordinate system, one needs to know the \scare{generalized fundamental metrical form}. In Einstein's notation (where summation over repeated indices is implied), it reads:

\begin{equation}\label{eq:lineelement}
ds^2=\gmn dx_\mu dx_\nu\,.
\end{equation}
%
In Reichenbach's interpretation \cref{eq:lineelement} leaves the \scare{topological} function of numbering to the coordinate system, and assigns the \scare{metrical} function of measuring spatial lengths and time intervals to the metrical coefficients \gmn \rzlp{**}{**}. The \gmn are numbers by which coordinate differences $dx_\nu$ have to be multiplied so that, over larger region of \st, identical rods measure $ds=+ 1$ in every position and in every orientation, identical clocks $ds=-1$, and light rays $ds =0$. For example, a unit rod laid along the $x_1$-axis ($dx_2=0,dx_3=0,dx_4=0$) will measure $-1=\sqrt{g_{11}}dx_1$; a unit clock at rest will measure $1=\sqrt{g_{44}}dx_4$. By transporting our \rac, we can determine $g_{11}= ds^2/dx_1^2$, $g_{44}= ds^2/dx_4^2$ and in general all values of the \gmn in a given coordinate system. 

This measurement procedure is meaningful under the condition that the ratio $n_1/n_2$ is fixed once and for all, so that identical rods and clocks always measure the same $ds^2$ wherever they are placed. This empirical fact. However, which is a convetion which to be rigid is  The geometry of space-time is \Mink if it is possible so that $\gmn$ are constant, the diagonal $\gmnbar=1,1,1,-1$. However, in the general case, this is not possible, as $\gmn$ can be functions of the coordinates, $\gmn \neq \gmnbar$, and space-time is non-\Mink{}ian over larger regions. This indeed, the case in the case of the presence of a real gravitational field. The \gmn are the potentials of the gravitational field. The statement of Reichenbach's famous conventionalism is in this formalism. In principle, Reichenbach pointed out, one can always reintroduce such a difference, by a universal force, that rod is not a unit rod and clock is not unit clock, by they shortened and by a field $\gamma\mn$.

In other terms one can set $\gmn=\gmnbar+\gamma_{\mu \nu}$, where $\gmnbar$ are the normal orthogonal values of the $\gmn$, and refer to the $\gmnbar$ as the \Mink geometry measured by \rac and light rays, and only to the $\gamma_{\mu \nu}$ as the some gravitational potential field determined by the path of free-falling test particles \rzlp{**}{237}. If this separation into the geometry and gravitational field is introduced, then \rac and light rays measure the $\gmnbar$, but free-falling particles follow the geodesics of the $\gamma_{\mu \nu}$ \rzlp{**}{237}. Free-falling particles will determine a \scare{measurable but distorted} geometry, which would differ from the \scare{true but hidden} \Mink geometry. However, since the $\gmnbar$ are in principle inaccessible to measurement, the splitting of $\gmnbar$ and $\gamma_{\mu \nu}$ is \q{hardly appropriate} \rzlp{**}{237}. In \gr, the gravitational field affects equally rods and clocks, light rays, and the behavior of test particles; as a consequence, all these instruments agree on the \emph{same} geometry, that is, on the same Riemannian geometry with the same curvature.

Once again this agreement is the reason \gr is to have established a new link between physical geometry and gravitation. Traditionally, \spti geometry has been associated with the behavior of ideal measuring instruments rods (geometry) and clocks (chronometry) and \scare{free} test-particles (inertial structure). In \pgrc theories, including \sr, non-accelerating \rac, and light rays determine the geometry of space and time (that is flat \Mink geometry of special relativity) $\gmnbar$, whereas the motion of test particles determine field properties. For example, charged test particles deviate from their straight \wl in \Mink \st under the influence of the electromagnetic field. The equivalence principle makes this separation impossible in the case of the electromagnetic field. Gravitation is a \emph{universal force} that affects all bodies in the same manner. \q{This is the physical significance of the equality of gravitational and inertial mass} \rzlp{**}{257}: \q{If gravitational and inertial mass were not equal, we would not be able to look upon the paths of freely falling mass points as (four-dimensional) geodesics \textelp{}, \emph{since different geometries would result for the various materials of the mass points}} \rzlp{**}{257}. It be bacuse that are equal that that  \rac and light rays and free-falling particles define a \emph{single geometry} \rzlp{**}{256}. The distinction between the fixed geometrical background and the changing gravitatianl field disappears.

As Reichenbach points out, because of this non-separability, \q{it has occasionally been said that this conception deprives gravitation of its physical character and \emph{that gravitation, therefore, becomes geometry}} \rzlp{**}{256}. However, this conclusion, in Reichenbach's view, was not justified. Two aspects have to be distinguished:

\begin{itemize}
\item The universal effect of gravitation on all kinds of measuring instruments defines a \emph{single geometry}. \q{In this respect we may say that gravitation is \emph{geometrized}} \rzlp{**}{256}. We do not speak of deformation of our measuring instruments \q{produced by the gravitational field}, but we regard \q{the measuring instruments as \scare{free from deforming forces} in spite of the gravitational effects} \rzlp{**}{256}. what theu disagree, from \Mink space-time.

\item Even if \q{we do not introduce a force to explain the \emph{deviation} of a measuring instrument from some normal geometry}, we must still invoke a force as a \emph{cause} for the fact that \q{there is a general correspondence of all measuring instruments} \rzlp{**}{256}. One may still consider the gravitational field as the \emph{cause} of the fact that all measuring instruments happen to \textit{agree} on the same geometry.
\end{itemize}

Thus, Reichenbach points out that there two different philosophical issues at stake that should not be confused:

\begin{itemize}
\item Riemann, Helmholtz, and Poincar\'e introduced the problem of the \emph{coordinative definition} in the philosophy of geometry \rzlp{**}{257}. \Gr continued in this tradition. Once one defines $ds\pm =1$, $ds=0$ in terms of \rac and light rays, the geometry of \st can be ascertained empirically; it is a branch of physics that can be true or false.

%Conceptual definition of $ds\pm =1$, $ds=0$ in integrated by by light rays, there is a coordinative definition that \rac.

\item Einstein introduced the problem of a \emph{scientific explanation} of physical geometry, which finds its mathematical solution in the field equations. The gravitational field has an effect on \rac and light rays, that is comparable to that of any other field of force, if not for the fact that it is a universal effect \rzlp{**}{256}.

\end{itemize}
%
The issue of \s{coordination} is well-known to Reichenbach's scholars, the second one has been hardly emphasized in the literature. The conventionalist move is a stepping stone. Once, again it serve to breake the crypto-kantian prejudice that natural geoemty is Euclidean. However, once the is broken Reichenbach once again to reastiblich of geomery. Indeed, the removal of serve that only one convetional choice is legitimate, and the geometry is empirical task. 

%Still However, it has not been appreciated that 

%Reichenbach insisted on the importance of the problem of \scare{explanation} on several occasions, including in his writings on \sr. 

%Lorentz provided a \emph{dynamical explanation} for the fact that \rac do not behave according to Newtonian \spti, but they are distorted by their motion through the ether. By contrast, Einstein-\Mink provided a \emph{geometrical illustration} of the same behavior by claiming that \rac behave as they do because \spti is \Mink{}an.  In \gr, an analogous distinction can be made. However, no explation is proved, of wy theu do conferom to this geomety and not another. In \gr, an analogous distinction can be made. 

Rods and clocks, light rays, and free-falling particles all agree on the same non-Euclidean non-flat \spti geometry. Two types of explanations of this empirical fact are possible: \begin{inparaenum}[(1)] \item a \emph{dynamical explanation}: there is force, gravitation, that distorts all our measuring instruments, \rac, light rays, and free-falling particles in the same way \item a \emph{geometrical explanation}: free-falling particles light rays, \rac behave as they do the geometry of \st is non-Euclidean, and \end{inparaenum}. Like in the case of \sr, Reichenbach considered the first dynamical explanation unsuitable since it introduced an otherwise inaccessible background geometry. However, he considered the geometrical explanation, simply the renunciation of any explanation. It is simply the restament of the agrement between, light particle  and matter goemtry, happen to agree.

%It simply the repression of a physical of fact in geometrical terms, that gravitational field \textit{does not} affect the behavior of \rac. 

Thus, Reichenbach insists that also in \gr, it was necessary to provide a dynamical explanation of the observed behavior of \rac, although a dynamical explanation of a new kind. As we have seen, traditionally, we introduce a dynamical explanation to account for the fact that \rac, light rays, and test particles \emph{diverge} from an alleged standard behavior defined by a flat background geometry. However, according to Reichenbach, one needs to explain why \rac, light rays, and test particles \emph{converge} towards a \scare{single geometry}, which is general non-flat and depends on the matter distribution. In order to characterize this kind of explanation, Reichenbach resorts once again to Weyl's notion of \scare{\emph{adjustment}} (\german{Einstellung}) to contrast to that of \emph{deviation} \origg{Abweichung}: \q{The word \scare{adjustment,}, which was first used by Weyl in this connection, characterizes the problem very well} \rzlp{**}{201}. \Gr requires the introduction of a similar explanation. It cannot be an accident that, say, around the sun, measuring instruments of all kinds, light rays, free falling ecc. all agree on the same non-flat \spti geometry.

%Weyl had used the expressions to account for the non-trivial Riemannian behavior or \rac. It cannot be an accident that two measuring rods that have the same length at one place always have the same length when brought to a different place along different paths. \q{It must be explained as an adjustment to the field in which the measuring rods are embedded like test-bodies} \rzlp{**}{201}. \Gr requires the introduction of a similar explanation. It cannot be an accident that, say, around the sun, measuring instruments of all kinds, light rays, free falling ecc. all agree on the same non-flat \spti geometry. The geometrical is only statemnet of this fact, that that this fact must be explaiedn: In Reichenbach's view, \q{the word \scare{adjustment} \textelp{} poses a problem rather than supplies a solution}. As Reichenbach quite surprisingly pointed out, \q{\textins{t}he answer can, of course, be given only by a detailed theory of matter, which has not yet been elaborated} \rzlp{**}{201}.

The combination of gravitation and geometry, therefore, does not force us to forgo \emph{dynamical explanations} but teaches us that this kind of explanation is applicable even to those cases in which it was customary to resort to a \emph{geometrical explanation}.  Indeed, according to Einstein's theory, \gr teaches us that we may consider the \q{effect of gravitational fields on measuring instruments to be of the same type as all known effects of forces} \rzlp{**}{257}. What is characteristic of gravitation is that this force has an effect on the measuring instruments also used in geometry, \rac and light rays, that is on those measuring instruments that in previous theories could be insulated from the effects of physical fields \rzlp{**}{13}. The universal influence of the gravitational field on all our measuring instruments, however, does not imply that it is \q{\emph{the theory of gravitation that becomes geometry}}; on the contrary, that it implies that \q{\emph{it is geometry that becomes an expression of the gravitational field}} \rzlp{**}{256}. 

According to Reichenbach, in \gr, too, only a theory of matter can explain this peculiar behavior of the space-time measuring instruments. In the presence of a real gravitational field it is impossible to arrange rods and clocks in a rectangular grid, just like it is impossible to \scare{develop} a flat piece of paper around a sphere. That the geometry of space-time is a flat network of \rac, cannot explain the behavior of bodies—that explains why gravitational fields follow the geodesic path, and not \q{We know that a more detailed investigation would reveal the presence of molecular force-fields, which affect the molecules on the surface of the sphere and thus force it into a definite}{wir wissen, daß eine genauere Betrachtung das Vorhandensein molekularer Kraftfelder lehren W\"{u}rde, die die an der Oberfl\"ache der Kugel liegenden Molek\"{u}le angreifen und in bestimmte Bahnen zwingen} \rzl{295}{258} congruence relationship. Once again, only a dynamical expansion could complete the theory. That the Earth is spherical, however, clearly shows that we should explain how the surface acts to resist or react. This would be the proper explanation—in the case of space or space-time, only a theory of matter can provide it.



%For example, one can often use them to get a rough preliminary idea of the answer. But one should beware of trying to use them for everything, for their utility is limited. Analytic or algebraic arguments are generally much more powerful.


%This paper hope to have made the case that Reichenbach defened a dynamical interpretation of special relativity


\section{Conclusion}

The main point of Reichenbach's axiomatization of special relativity is the separation between the light axioms and the matter axioms. While the distinction between classical and relativistic light geometry is the result of \textit{convention}, the agreement between matter geometry and relativistic light geometry requires \textit{explanation}: actual rods behave relativistically like $l$ and not classically like $L$. \begin{inparaenum}[(a)] \item Lorentz offered a \emph{poor dynamical explanation} of this fact, since it considers the behavior $L$ as normal and seeks to explain the deviation. \item Einstein rejected that prejudice and declared the behavior $l$ as the natural one, thus providing \emph{no dynamical explanation}. \item \Mink provided a graphical representation of, but \emph{no geometrical explanation}. \end{inparaenum} Reichenbach concluded that \sr is still explanatory deficient. Why do rods behave like $l$, and not classically like $L$? A \emph{good dynamical explanation} is necessary to explain why rods of all kinds precisely adjust to the relativistic light geometry and not to a different one 

The paper concludes that Reichenbach's interpretation of relativity \begin{inparaenum}[(a)] \item \textit{historically} stands out as the most fitting forerunner of the dynamical approach to special relativity\footnote{It would be possible to show that Reichenbach, in closing his 1928 book, attempted to provide a dynamical interpretation of general relativity. This goes beyond the scope of this paper}, without advocate for a return to the ether theory;  \item \textit{systematically} is a better version of the dynamical approach compared to the one advocated today. \end{inparaenum}.  In particular, Reichenbach offers a better articulation of the different feature of what counts as \s{good} dynamical explanation in this context:

\begin{itemize}

\item \latin{explanans}: Reichenac' that ligth axiosm and mater axiosm. Thus, Lorentz's appeal to a theory of matter is inadequate, and Einstein's claim that such a search is unnecessary unsatisfactory. Minkowski’s space-time formalism that the two geometries agreesd, but they do not explain why. Reichenbach thus concludes that the only way forward is to seek an \emph{explanans} in a yet unknown \s{theory of matter}, complementing the existing Maxwell \textit{theory of radiation}.

%The word \s{contraction} is, however, a misnomer; what we should explain is not the effect of motion through the ether on the rod, but why the rod is shorter than it would have been in the old theory of the round-trip time of light is used as a standard of length;\todo{improve}

 \item \latin{explanation}: In asking for an explanation, Reichenbach does not seek to account for a \s{deviation} \origg{Abweichung} from some presumed classical standard, as there is no reason to regard the classical behavior of \rac as natural. What requires explanation is the \s{adjustment} \origg{Einstellung} between the behavior of light rays and \rac, which all converge on the same \Mink geometry.
 
 \item \latin{explanandum}: Reichenbach clarifies that what must be explained is not \s{Einstein contraction}, but \s{Lorentz contraction}. This removes the common objection that the dynamical approach overlooks the perspectival nature of length contraction in \sr. Einstein contraction, being frame-dependent, requires no explanation; Lorentz contraction, by contrast, expresses an objective difference.
\todo{improve}

 
 
\end{itemize}
%
In this way, Reichenbach offers important contributions to \s{steel man} the dynamical approach framework, neutralizing some of the usual objections raised against it. However, I argue that precisely for this reason, he unwittingly exposes what is, in my view, that framework's fault line—one that defenders of the geometrical approach have overlooked. Reichenbach, like the modern dynamicists, claims that a future, yet unknown, theory of matter is need to finally complete \sr. However, what should such a \sr{theory of matter} actually be like?  

This is the point where, I think, Einstein misses the gist of his own approach. Einstein’s theory, unlike Lorentz’s, does not need to find a \textit{particular} theory of matter or radiation; special relativity accounts for the failure of ether drift experiments by requiring that \textit{any} theory of matter or radiation must be Lorentz invariant. \Mink introduced a four-dimensional vector calculus that allows to check directly whether whether available laws comply with this requirement, but does not provide any further explanatory contribution. Thus, neither the dynamical nor the dynamical approach to this purpose cast the contribution of \sr relativity in explanatory terms. Tho this purpose is more appropriate to refer to it as an \emph{explanation by constraint}, following the suggestion of Marc \citet{Lange2016}. The misunderstanding of this point also accounts for the different attitudes of Einstein and Reichenbach toward Miller's alleged positive result from an ether drift experiment. For Reichenbach, if Miller turned out to be right, \q{nothing would change}. Indeed, the fact that both the theory of radiation and the available theory of matter \textit{happen to be} Lorentz invariant is a mere \textit{coincidence}. However, Einstein transformed this coincidence into the constraint that \textit{all} laws of nature \textit{must} satisfy if motion through the ether is to remain undetectable. Thus, if Miller's result were to be confirmed, the \sr \s{would go down like a house of cards}.



%For Reichenbach, the fact that the laws governing the field and those governing matter agree is a \textit{coincidence} that requires an \textit{explanation}.  Einstein's strategy was to transform this \textit{coincidence}, revealed by the failure of all ether-drift experiments, into a \textit{requirement} that all laws of nature must satisfy. This is the sense of \citets{Einstein1919} famous comparison between relativity and \th. Classical \th asks: How must the laws of nature be constructed in order to rule out the possibility of bringing about perpetual motion of the first kind? The laws of nature must satisfy energy conservation. \Rt asks: How must the laws of nature be constructed in order to rule out the possibility of bringing about the ether wind? The laws of nature must be Lorentz invariant. 



%


%\Sr does not need to \emph{find} an \emph{actual} constructive Lorentz invariant theory of matter describing the behavior of rods and clocks; it is sufficient to \emph{require} that all \emph{possible} constructive theory, whatever they might be, are Lorentz invariant.

%However, such a theory would be an \emph{instantiation} of Einstein's new kinematics, not an \emph{explanation} of the latter.


%This is the sense of \citets{Einstein1919} famous comparison between relativity and \th. Classical \th asks  should the laws of nature be laid of the first kind must be impossible: the laws of anture sbou satisfy the energy conserati. The special relativity laws of nature be laid if ether drift should be impossible; the laws of nature should be Loretnz invariant.



%This also explains the negative result of the Michelson-type experiments: that the laws governing matter hold would make the requirement meaningful, to find a pererla motion ac. Reicenac simply accept that it is merely a coincidence, that he would s\todo{improve}



%\begin{itemize}
%\item Lorentz maintained the old kinematics, thus, need to provide a \emph{dynamical explanation} for why the arms of an interferometer contract by the right amount to compensate for the difference in the two-way velocity of electromagnetic radiation. Since the laws governing radiation—\ME—\emph{happen to be} Lorentz invariant, a \emph{particular} electromagnetic theory of matter could explain why the material arm of the interferometer contracts exactly in the same way as the electromagnetic field.
%
%\item Einstein, by contrast, doubted the validity of \ME from the outset, and derived a new kinematics based on the Lorentz transformations independently of any actual theory of matter or radiation. Einstein theory contains only the requirement that \emph{all} possible laws of matter and radiations, whatever they may be, \emph{must} be Lorentz invariant. According to Einstein, Minkowski’s contribution was to have developed a vector calculus that helps to check whether a theory is actually Lorentz invariant, without the need to carry out the transformations by sheer calculation. 
%\end{itemize}
%%
%Einstein neither embraced \Mink's claim to have provided a \emph{geometrical explanation}. However, Einstein did not feel the need for a \emph{dynamical explanation}; indeed, he more or less explicitly emphasized that this was the advantage of his approach over Lorentz’s. 

%Lorentz’s theory stands or falls with his \emph{particular} Lorentz-invariant theory of matter and radiation. Thus, it needed to hypothesize about the structure of matter, the nature of molecular forces, etc. Einstein’s theory, by contrast, requires that \emph{any} theory of matter or radiation must be Lorentz invariant. Thus, no such detailed assumptions were needed. Certainly, setting up a new kinematics is not sufficient. Ultimately, one must complete the theory by searching for the particular theory of matter and radiation that happens to hold in nature; the equations of such theory might have solutions corresponding to material systems that we can use as \rac, thereby closing the circle. However, such a theory would be an \emph{instantiation} of Einstein's new kinematics, not an \emph{explanation} of the latter. 



%Were, that fact tha ..how be lod if is post. Nad How s that be imps the reqiret of Lroetz invariece. 



%Reichenbach, so to speak, \s{puts asunder what God hath joined together}.



 
%Reichenbach concluded that while Lorentz provided a bad dynamical explanation, Einstein provided no explanation at all, and thus his theory was still incomplete. 

%He argued that also in special relativity we needed a future \emph{dynamical explanation}, that is, to find a specific theory of matter that explains why \rac exhibit relativistic behavior. However, this indeed misses the point. 



%


%In both cases, the demand for an explanation and a detailed theory adds nothing further to the explanation itself. This characteristic can also be extended to general relativity. General relativity imposes fundamental constraints on any possible theory of matter and fields: (1) the laws must be generally covariant (\gc); (2) second derivatives of \gmn must not enter into the field equations (minimal coupling). If one asks why the gravitational field is a universal force—that is, why it affects \rac of any material in the same way—the explanation is that the components of the Riemann tensor do not explicitly enter into the formulation of the laws governing the material systems we use as rods and clocks. In this sense, they \s{adjust} to the gravitational field. Also in this case, we have an \emph{explanation by constraint}. Finding a particular theory of matter that satisfies this requirement does not add further explanatory value.











%\appendix


%\section{Linearity}
%
%% !TEX root = reichenbach_explanation.tex

%If bodies did not behave like $l$ in the relativistic sense, but like $L$ in the classical sense, there would be ne the Einstein contraction that $l$ is a rigid rod.

%


%Thus, the relativistic length of a moving rod from the rest frame is $\lK{}{'}$; the relativistic length of a stationary rod measured in the moving system is $\lK{'}{}$:
%\todo{The notation serves to highlight the relational nature of the notion of length}

%\begin{equation*}
%\lK{\notateol{'}{0.5}{\texts{measured in the moving frame}}}{\notateul{\textcolor{white}{'}}{0.5}{\texts{at rest in the rest frame}}} \quad \quad \LK{\notateor{\textcolor{white}{'}}{0.5}{\texts{measured in the rest frame}}}{\notateur{'}{0.5}{\texts{at rest in the moving frame}}} 
%\end{equation*}
%


In order to provide a proof of this claim, Reichenbach resorts to the somewhat idiosyncratic notation introduced in his \citeyear{Reichenbach1924} monograph. He labels $l$ a rod following the Lorentz–Einstein theory, and $L$ one with classical behavior. $K$ and $K'$ are resp.\ the stationary and moving systems. Then he uses index notation where the upper index $^{K}$ marks the frame in which the rod is measured, the lower the frame in which the rod is at rest $_{K}$. In both the classical and Lorentz–Einstein theories, the lengths of unit rods in the rest frame $K$ are equal, or $\lK{}{} = \LK{}{} = 1$. The difference emerges when one considers the length of the rod in the system in motion $K'$:

\begin{itemize}
\item \emph{classical theory}:  \hide{the rest length of the rod as measured from the moving system is equal to the rest length in the rest frame $\LK{'}{'}=\LK{}{}$
%
%\footnoteh{In classical mechanics $x' = x - vt$, then at the time $t = 0$, when $K$ and $K'$ coincide, we have $vt = 0$, and thus $x = x'$}
%
,} the rod has a unique length, regardless of motion:

\begin{equation}\label{eq:CT}
\frac{\LK{}{}}{\LK{'}{'}} = \frac{1}{1} \,\,\ \text{no contraction}
\end{equation}
%
\item \emph{Lorentz theory}: the length of the moving rod in the moving system $K'$, is \textit{contracted} \hide{$\lK{'}{'}<\LK{'}{'}$ by a factor \kappafactor} with respect to the length that it would have in the classical theory in the same frame $K'$:

\begin{equation}\label{eq:LT}
\frac{\lK{'}{'}}{\LK{'}{'}}= \frac{\kappafactor}{1} \,\,\ \text{Lorentz contraction}
\end{equation}

\item \emph{Einstein theory}:  the length of the moving rod measured in the rest system $K$ is contracted \hide{$\lK{}{'}<\lK{'}{'}$, by a factor \kappafactor} with respect to the length of the moving rod measured in the moving system system $K'$:

\begin{equation}\label{eq:ET}
\frac{\lK{}{'}}{\lK{'}{'}} =\frac{\kappafactor}{1}  \,\,\ \text{Einstein contraction}
\end{equation}
%
\end{itemize}
%
and viceversa\footnoteh{The length of the rest rod measured in the moving system $K'$ is contracted $\lK{'}{}<\lK{}{}$, by a factor \kappafactor with respect to the length of the moving rod measured in the moving system system $K'$: $\frac{\lK{'}{}}{\lK{}{}} =\frac{\kappafactor}{1}$.}. Reichenbach aims to prove that the fact that Lorentz contraction and Einstein contraction amounts to the same factor \kappafactor, is the consequence of the linearity of the Lorentz transformations. His proof runs as follows. According to classical theory we have $\LK{'}{}=\LK{}{}$, and since $\lK{}{}=\LK{}{}$ we obtain $\LK{}{'}=\lK{}{}$. Relation \cref{eq:ET} therefore becomes

\begin{equation}\label{eq:ETCT}
\frac{\LK{}{'}}{\lK{}{'}} = \kappafactor
\end{equation}
%
Because of the linearity of the transformation ratio \cref{eq:ETCT} is the same as ratio \cref{eq:LT}, which means that \cref{eq:ET} is also the same as ratio \cref{eq:LT}: the ratio is the same in all case and in particular is equal to \kappafactor.  However, Reichenbach insists that there is the deep \emph{conceptual difference} between the two contractions despite the their coincidental \emph{numerical equality} \rc{45f.}{189f.}.




%Indeed, shows that one can imagine cases in which in which there is no Lorentz contraction but an Einstein contraction\footnote{With not Lorentz contraction; the Einstein contraction from $K'$ to $K$ would also disappear. However, in the classical theory \cop{we might define the simultaneity in $K^{\prime}$ according to Einstein's convention, setting $\epsilon=\frac{1}{2}$, the inverse comparison from $K$ to $K^{\prime}$ will show the Einstein contraction}. The magnitude of which is the square of that of the Lorentz-Einstein contraction\todo{why?}.} and viceversa\todo{improve}.






%\printshorthands
\printbibliography
%
\end{document}