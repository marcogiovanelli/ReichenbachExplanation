
%\footnote{This claim is actually problematic \cites[cf., e.g.,][]{Stachel1982}{Dongen2009-07}}


\footnoteh{The length of the rest rod measured in the moving system $K'$ is contracted $\lK{'}{}<\lK{}{}$, by a factor \kappafactor with respect to the length of the moving rod measured in the moving system system $K'$: $\frac{\lK{'}{}}{\lK{}{}} =\frac{\kappafactor}{1}$.}
%by Reichenbach's claim to be able to construct a pure \s{light metric}, forgoing not only rods but even clocks ,

% \qt{Die Möglichkeit einer reinen Lichtmetrik ist in der Tat höchst überraschend. Dass man ohne Maßstäbe auskommt, hat mich zwar nicht so erstaunt, dass aber ausserdem noch auch die Uhren entbehrt werden können, ist überaus bemerkenswert}

%He was willing to admit with Schlick that Poincaré would probably acknowledge this point. Nevertheless, Reichenbach still insists that t

%\begin{inparaenum}[(a)]
%\item The classical theory entails neither contraction;
%\item Lorentz's theory entails the Lorentz contraction without the Einstein contraction;
%\item Einstein's theory entails both the Einstein and Lorentz contractions;
%\item a hypothetical example of a theory entailing the Einstein contraction without the Lorentz contraction can be constructed by appropriately defining simultaneity :
%\end{inparaenum}

%\begin{inparaenum}[(a)] As we have seen, \item the classical theory entails neither type of contraction: \emph{rigid} rods behave like $L$. The theory is refuted by \MME. \item Lorentz theory includes only the Lorentz contraction: \emph{distorted} rods behave like $l$. \item Einstein’s theory includes both contractions: \emph{rigid} rods behave like $l$ \end{inparaenum}. 



%either with $\epsilon = \sfrac{1}{2}$ or with $\epsilon \neq \sfrac{1}{2}$
%As we have seen, \begin{inparaenum}[(a)]\item the classical theory entails neither type of contraction: \emph{rigid} rods behave like $L$. The theory is refuted by \MME. \item Lorentz theory includes only the Lorentz contraction: \emph{distorted} rods behave like $l$. \item Einstein’s theory includes both contractions: \emph{rigid} rods behave like $l$. \end{inparaenum} \Mink's geometrica representation illustrates the difference between these three cases but does not explain why rods behave like $l$ and not like $L$:

%(a) the rest-length of the moving rod is different from the rest-length
%of the rod at rest.
%(b) the rest-length of the moving rod is different from the rest-
%length of another rod which moves with it but satisfies the
%classical theory.

%Rods and clocks at rest in a relatively moving system measure the same speed of light as those at rest in the original frame. Even when lengths and times are projected differently due to relative motion, the ratio of distance to time—used to measure the speed of light—remains \( c \)
 
%Rods and clocks at rest in a relatively moving system measure the same speed of light, since their length is projected differently on the differetn coordinate systems.
 


%As Reichenbach once put it, after Einstein, even if there were absolute time, it would not be absolute \citep[18/110]{Reichenbach1922}.

%However, an entry in Rudolf Carnap’s diary recording a meeting with Reichenbach near Berlin on \datef{2}{9}{1926} indicates that Reichenbach continued to stress the importance of the distinction between the two types of contraction: \qt{He explained me the difference between Lorentz and Einstein contractions}{Er erkl\"art mir den Unterschied zwischen Lorentz- und Einstein-Verk\"{u}rzung} \citep[303]{Carnap2022}. 


%Let $K$ be a stationary frame with synchronized clocks. A bar $AB$ moves along the $x$-axis of $K$. Imagine a device that marks two points, $A'$ and $B'$, on the $x$-axis, when $A$ and $B$ coincide with them on the $x$-axis at the same time in $K$. $A'B'$ is what Reichenbach calls the \emph{projection} of $AB$ in $K$

%Einstein contraction compares the same rod \emph{with respect to different reference frames within the same theory}. Lorentz contraction, by contrast, compares the same rod \emph{in the same frame with respect to different theories} using the light-geometrical definition of length as \emph{tertium comparationis} \rzl{228}{196f.}.  

 %A magnetic needle has no intrinsic orientation; however, the fact that it always points north means that it \s{adjusts} every time to the Earth's magnetic field. Similar rods have no intrinsic length; they \s{adjust} themselves to light geometry \citep[66]{Reichenbach1927}. 





%are also the  and the difference between the Lorentz and Galilei transformations is \emph{matter of convention} about the definition of simultaneity}.  is \emph{matter of fact}.


% 


%

%

%and distinguished from the Galilei transformations only definitionally; 

%\new{The Lorentz transformations can be derived in light-geometrical terms are distinguished from the Galilei transformations only by a convention; the new factual prediction made by Einstein is that  the Lorentz transformations are also the transformations for \rac}. 

%The Lorentz transformation can be derived from this

% still stresses the continuity of his \q{axiomatic efforts} with his earlier neo-Kantian stance.


%The weakness of the dynamical approach is ultimately encapsulated in the separation between the light and matter axioms at the core of Reichenbach's axiomatization. %Einstein did not restate this coincidence in the form of a \emph{convention}, as Reichenbach argues. Instead, he elevated it to a \emph{constraint} on all possible theories of matter and radiation. For this reason,



%In his reading, Einstein simply transformed this coincidence into a \textit{convention}, while \Mink only provided a geometrical \textit{interpretation} of such conventional choice. Reichenbach found this \s{trick} unsatisfying. He concluded, like modern dynamicists, that some \emph{detailed} theory of matter describing the behavior of \rac is required to explain the failure of \ED experiments. 



%What explains the failure of all \ED experiments is that \emph{all} laws of nature are Lorentz invariant, not a the \emph{particular} laws of matter that happen to hold are Lorentz invariant.

% The behavior of radiation always agreed to that of matter, since they are both Lorentz invariant.

%\todo{It is not a coince it is a requiremnt that we impose if the impossiblity of decetive the erath should be uphod. } \todo{Reichenbach first separet the matter and light and that their agreemnt as a coince. Einstein is to ahve coincede as constraint}. The behavor adhst, sicne the laws fovering both processes are both Lorentz invariant. The difference emerges clearly in Einstein's and Reichenbach's opposite responses to Miller's alleged positive result from an ether drift experiment. 



%Reichenbach has enconded the misundestading of this point the very structure of axiomatic the separation of light and matter axioms. That laws govenring radaitio and governing matter coince that requre of eqpat. The gola to have tranformed this coinceice into a constraitn

%From this perspective the fact that the available theory of radiation and the theory of matter both happen to be Lorentz invariant is a \textit{coincidence}. Einstein transforms this coincidence into a \textit{constraint} that all theories of matter and radiation must satisfy. 

%The explanation of the negative result of \ED experiments is not that a \textit{particular} Lorentz invariant theory of matter holds, as Reichenbach argues, but the requirement \textit{all} possible theory of matter are Lorentz invariant.  

%The comparison with \th illuminates the difference. If a \pemo machine could be constructed using a certain physical process, it would not simply mean that the particular laws of nature governing the process violate energy conservation—it would mean that the principle of energy conservation must be abandoned.


%Thus, Einstein appears to reject both the dynamical and geometrical explanations. 

%Thus, Einstein appears to reject both the dynamical and geometrical explanations. If one wishes to cast Einstein's approach in explanatory terms, it is more appropriate, following the suggestion of Marc \citet{Lange2016}, to refer to it as an \emph{explanation by constraint}. Indeed, in this suggestion best captures the meaning of Einstein's comparison with \th. Both \th and \rt transform what appears to be a mere \emph{coincidence}—the repeated negative result of an experiment—into a \emph{constraint}. \Th claims that, if the construction of a \pemo is to be impossible, then \emph{all} laws of nature \emph{must} satisfy the requirement of energy conservation. \Sr claims that, if the construction of an ether-wind detecting device is to be impossible, \emph{all} laws of nature \emph{must} satisfy the requirement of Lorentz invariance. The words \emph{all} and \emph{must} are key to the explanatory role in both theories. This, I think, is what the dynamical approach misunderstands. 

%\s{What therefore God hath joined together, let not man put asunder.}

%For the dynamical approach, like in Lorentz theory, the failure of \ED experiments experiment is explained by the \emph{coincidence} that the \emph{specific} theories of radiations and matter that hold in nature \emph{happen to} be Lorentz invariant. On the contrary, in Einstein's theory, the failure of \ED experiments is explained by the \emph{requirement} that \emph{any} physical theories  of matter and radiation \emph{must} be Lorentz invariant. 


%Reichenbach encoded this misunderstanding in his own axiomatic system by separating the light and matter axioms and declaring their agreement a coincidence. In Reichenbach's assessment, Einstein simply transformed this coincidence into a convention, forgoing any explanation in terms of a particular matter theory. However, Einstein's move was different: he transformed the coincidence into a constraint that all matter theories must satisfy. 



%The word \s{contraction} is, however, a misnomer; what we should explain is not the effect of motion through the ether on the rod, but why the rod is shorter than it would have been in the old theory of the round-trip time of light is used as a standard of length;\todo{improve}


%The main point of Reichenbach's axiomatization of special relativity is the \emph{separation} between the light axioms and the matter axioms. While the distinction between classical and relativistic light geometry is the result of a \emph{convention}, the agreement between matter geometry and relativistic light geometry is a fact that requires \emph{explanation}: actual rods behave relativistically like $l$ and not classically like $L$. \begin{inparaenum}[(a)] \item Lorentz offered a \emph{poor dynamical explanation} of this fact, since it considers the behavior $L$ as normal and seeks to explain the deviation. \item Einstein rejected that prejudice and declared the behavior $l$ as the natural one, claiming that \emph{no dynamical explanation} was necessary. \item \Mink provided a graphical representation of, but \emph{no geometrical explanation}. \end{inparaenum} Reichenbach concluded that \sr is still explanatory deficient. Why do rods behave precisely like $l$, and not classically like $L$? A \emph{good dynamical explanation} is necessary to explain why material rods of all kinds happen to adjust to precisely to the behavior of electromagnetic radiation.


 %Einstein simply declared that the moving rod retains its length in its own frame, despite appearing shorter when measured in the stationary frame. Einstein achieved this result by introducing \emph{by convention} a new definition of length as a frame-dependent quantity: 


%Reichenbach concluded his reconstruction by arguing against not better specified objections he had received since the publication of his \citeyear{Reichenbach1925} article discussed above (\cref{twocontractions}). In particular, it was argued that it is impossible to compare two magnitudes belonging to \emph{different} theories. However, Reichenbach has no difficulty to retort that we actually do this all the time. We may say, for instance, that a highly compressed gas behaves differently than \emph{it would have} according to Mariotte's law for ideal gases—that is, an ideal gas occupies a smaller volume than a real gas would have occupied in the same conditions \rzl{**}{**}. The comparison is possible because the Euclidean rigid rods used to measure the gas volume serve as a \emph{tertium comparationis} \rzl{**}{**}. Also in special relativity, Reichenbach pointed out, we have a \emph{tertium comparationis}: light-geometrical definitions supply a standard to which the rods of the different theories can be compared \rzl{**}{**}. The difference between the classical theory and the Lorentz-Einstein theory is then an objective fact. 

%The problem is to grasp the difference between Lorentz and Einstein theories:
%
%\begin{enumerate}[label=(a)]
%\item  \cop{the rest-length $l$ of the moving rod is shorter than that of a rod moving according to the classical theory $L$} (Lorentz–Einstein)  
%\item \cop{the length of the moving rod $l$ is shorter than that the rest-length of the rod rest $L$} (Lorentz)\todo{check}.
%\item the rest-length of the moving rod $l$ is equal to the rest-length of the rod at rest $l$ (Einstein)
%\end{enumerate}
%%
%Statement (a) is correct for both theories, thus Lorentz contraction is not \emph{ad hoc}. From (a), Lorentz infers (b), that the moving rod behavior is a distortion of the true rest length. Einstein redefines the concept of length and declares (c), that is that (a) is the default behavior of rigid rods. This move is important, since it removes Lorentz’s crypto-Kantian prejudice that classical behavior must be \apr true in some sense. Einstein could wash his hands and declared a dynamical explanation of the contraction unnecessary. However, Reichenbach felt that this trick was ultimately unsatisfactory. 

%According to Reichenbach, it is correct to claim that Einstein contraction \emph{does not} require any explanation. This contraction is introduced precisely to redefine relativistic rods as rigid. Lorentz contraction, however, \emph{does} call for an explanation. The negative result of the Michelson experiment implies that rods of all materials \emph{happen to} behave in agreement with distances measured by round-trip light signals. How can such a \emph{coincidence} be explained? \Mink's supposed \emph{geometrical explanation}, in Reichenbach's view, is no explanation at all—it is merely a graphical representation of Einstein's move. Lorentz's \emph{dynamical explanation} is unsatisfying, as it assumes classical rod behavior as natural. Reichenbach concluded that the solution lies in a proper analysis of what counts as a \emph{dynamical explanation}.


%$OQ$ is the time axis and represents a clock at rest, while the horizontal axis $OS$ represents a rod at rest in $K$. The right hyperbola represents the locus of points from which a light round-trip yields the same spatial separation; the upper hyperbola represents the locus of points from which a light round-trip takes the same total time on the origin clock. Einstein's relativity of simultaneity is shown diagrammatically by the fact that both the time and space axes can be rotated. The rotation of $OQ$ represents a clock in motion with respect to $K$, while the rotation of $OS$ represents a rod in motion with respect to $K$. The agreement between light geometry and matter geometry is symbolized by the fact that $Q$ and $S$ follow the contours of the hyperbolas.


%The control of natural phenomena is achieved by means of mathematical concepts. These concepts arc defined by implicit definitions and are not dependent on a unique and specific kind of visualization.

%\subsection{Dynamical vs.\  Geometrical Explanation Special Relativity}


%Einstein had shown that determining the simultaneity of distant events requires knowing the speed of light, but measuring the speed of light presupposes knowledge of simultaneity. This circularity implies that simultaneity of distant events cannot be known in principle. 


%
%Unlike Schlick, however, Reichenbach explicitly rejects the idea that a confirmation of Miller's result would entail a revival of the ether theory. 
%s
%If the aether-drift experiments yielded a positive outcome only the material axiom \rom{8} would be challenged—namely, the fact that rods behave according to the relativistic light geometry. The distinction between classical and relativistic light geometries would continue to depend on the conventional definition of simultaneity. Only the claim that rods behave in accordance with the relativistic light geometry would be refuted. 
%
%The choice between them is thus one of simplicity, not truth \citep{Schlick1915}. 


%Einstein was firmly convinced that Miller’s findings were likely flawed, and he even declared himself willing to stake money on that judgment, as he plainly put it \citep{Einstein1926a}. Yet he also made it clear that if Miller’s results were validated, the principle of relativity would have to be entirely relinquished \citep{Hentschel1992a}.

%As we have seen, Reichenbach did not consider Lorentz contraction to be \latin{ad hoc}. Indeed, both Lorentz's and Einstein's theories assert the agreement between matter geometry and relativistic light geometry. 

%Evidence is provided to justify
%claims that something is the case. Explanations are provided to answer
%questions why something is the case.


%As we have seen, for Reichenbach, Lorentz contraction is not \latin{ad hoc} at all, since both Lorentz and Einstein theory presuppose Lorentz contraction, that is the agreement between relativistic light and matter geometry. The key novelty of the theory was the introduction of Einstein contraction, which, however, has nothing to to with such agreement, it serves only as to Einstein could Lorentz and nevertheless consider rods as rigid thus avoiding getting bogged down in the shallows of Lorentz's dynamic explanations of the contraction. Schlick was indeed, right that the difference between their approaches emerge with particular clarity from Reichenbach's reaction to Miller's experiment. If the experiment were to be confirmed, Schlick argued, the universal conspiracy of nature that hide the aether from detection would be broken and we had to return to the ether theory, that is the Lorentz's theory without the contraction hypothesis. On the contrary, according to Reichenbach, no such return was required. In this sense, Schlick argued, Reichenbach's axiomatic method has no \s{physical consequences}:
%

%Hans Reichenbach an Edgar Zilsel, 7. 5. 1925 HR-016-24-07. Tvposkript-Durchschlag (3 Seiten) Edgar Zilsel an Hans Reichenbach, 22.5.1925  HR-016-24-06, Typoskript (2 Seiten) 


%Lorentz accounts for the negative result of Michelson–Morley via compensatory contractions and dilations, while Einstein’s theory avoids such auxiliary hypotheses, making the contraction a consequence of relativity of simultaneity . 


%Indeed, it is hard to deny that, in criticizing an unidentified mainstream interpretation of \sr, Reichenbach appears to have aimed at Schlick's own position, that he had recently defended at the Leipzig meeting \GDNA, in which Reichenbach also participated \citep[60\psq]{Schlick1923}\todo{check}.

%Axiomatiken durch Zilsel in einem Brief an diesen vom 7. Mai 1925 nachdrücklich zu, vermerkte aber, daß trotz unterschiedlichster Motivation einzelne Axiome "fast wörtlich identisch" seien.

%, the Lorentz vs.\  Einstein debate could be framed as a classical case of theoretical underdetermination. 



%and was later picked up by \emph{Science} \citep{Slosson1925}.

%\citep{Hentschel1992a} 
%

%\qt{Mr. Reichenbach}{Herr Reichenbach,}  he wrote \qt{has recently published a paper \citetitle{Reichenbach1925} in the \citejournal{Reichenbach1925}}{hat vor kurzem in der \citejournal{Reichenbach1925}  34, S. 32 eine Arbeit \citetitle{Reichenbach1925} publiziert}; Schlick



%Two days before Reichenbach's paper was submitted to the \citejournal{Reichenbach1925}, on \datef{26}{6}{1925}, Edwin E.~Slosson, the first director of the Science Service and editor of the popular magazine Science News-Letter, asked Einstein for a comment.

%Two days before Reichenbach's paper was submitted to the \citejournal{Reichenbach1925}, on \datef{26}{6}{1925}, Edwin E.~Slosson, the first director of the Science Service and editor of the popular magazine Science News-Letter, asked Einstein for a comment.


\hide{the rest length of the rod as measured from the moving system is equal to the rest length in the rest frame $\LK{'}{'}=\LK{}{}$

\hide{$\lK{'}{'}<\LK{'}{'}$ by a factor \kappafactor}
 \hide{$\lK{}{'}<\lK{'}{'}$, by a factor \kappafactor} 
%
%\footnoteh{In classical mechanics $x' = x - vt$, then at the time $t = 0$, when $K$ and $K'$ coincide, we have $vt = 0$, and thus $x = x'$}
%
,}


%Which among the Riemannian geometry happens to hold in reality  is on the contrary a matter of \emph{convention}.

%Einstein (a) defines simultaneity by synchronizing clocks using the assumption of equal round-trip light speed; (b) defines the length of a moving object as the distance between its endpoints measured simultaneously in a given frame; (c) thus, it is not surprising that equal distances correspond to equal round-trip travel times of light in within each inertial frame. 

%In Reichenbach's perspective the light axioms have a higher degrees of certainty since they contain only statements that are already accepted by classical optics, plus the definition of simultaneity. On the contrary, the matter axioms make \emph{new} statements about very complicated material structures, and are only partially verified. Einstein was right in challenging Lorentz's implicit view that the classical light geometry is the natural one, thus declared relativistic rods as rigid. However, one still needs an explanation about why rigid rods do happen to behave according to a relativistic light geometry and not with another one. 

%Just a few months before Reichenbach's paper appeared, Miller had published the results of his Mount Wilson experiments \citep{Miller1925}.



%\section{Michelson and Miller's Experiments}

%\section{Two Kinds of Contractions: Einstein Contraction vs.\  Lorentz Contraction} 





%Classical aether theory could account for it by postulating that the aether is dragged along with the Earth in its orbit. However, this stratagem contradicts the phenomenon of stellar aberration \crw{**}{**}. If instead one preserves the existence of a stationary aether, with respect to which light travels at a fixed speed $c$, it becomes puzzling how the equality of the travel times is maintained, despite the fact that the apparatus is moving through the ether. As a result, 19th-century optics found itself in a seemingly irresolvable dilemma. The subsequent unfolding, Reichenbach argues, was usually considered uncontroversial.

%How can this negative result be explained? At the turn of the century, \q{Lorentz [\citeyear{Lorentz1895}] in Leyden presented his \myemph{explanation}} of this equality \qt{that assumed that all rigid bodies moving in opposition to the ether undergo a contraction}{H. A. Lorentz in Leyden eine Erkl\"arung gab, die eine Verk\"{u}rzung aller starren K\"orper bei Bewegung gegen den Aether annahm} \crw{325}{197\me}. When moving uniformly through the ether with velocity $v$ relative to the ether frame $K$ (where light travels rectilinearly at speed $c$), the arm $\mathrm{OS}_2$ contracts in the direction of motion by a factor \kappafactor—like a metal bar exposed to cold—compensating for the slower light speed and explaining the experiment's null result. 


%In \citeyear{Einstein1905}, Reichenbach continued, \qt{a more basic explanation was proposed by A.~Einstein in which these contractions occur as a result of a universal principle, the principle of relativity}{Noch tiefer ging die 1905 von A. Einstein aufgestellte Relativit\"atstheorie, welche diese Verk\"{u}rzung als Ausfluß eines universellen Prinzips betrachtet, des Relativit\"atsprinzips;} \crw{326}{197}. Instead of introducing a \emph{deformable} rod, Einstein proposed a new definition of what counts as a \emph{rigid} rod: spatial measurements depend on temporal simultaneity, which in turn is frame-dependent. From this perspective, the horizontal arm $OS_2$ of the Michelson interferometer is \emph{not} shortened with respect to the coordinate system $K'$ co-moving with the Earth; however, it is shortened by a factor \kappafactor with respect to a coordinate system $K$ at rest relative to the Sun. 
%
%Reichenbach concurs with the standard semi-historical account on one point: Lorentz's and Einstein's theories \emph{agree} on the same empirical fact, stated in axiom \rom{8} of his axiom system: \s{Two space intervals which are equal when measured by rigid rods, are also light-geometrically equal}, meaning they are equal when measured by the round-trip time of a light signal. To avoid invoking simultaneity at a distance, mirrors \origg{Spiegel} at $S_{1}$ and $S_{2}$ reflect the signal back to the same point $O$, and the round-trip time determines the distance. The Michelson experiment tests this: if the arms $OS_1$ and $OS_2$ (\cref{mv}) are equal in terms of round-trip light time ($\overline{OS_1O} = \overline{OS_2O}$), they are also equal when measured by rigid rods ($S_1 = S_2$). 

%The null result of the \MME confirmed this equality and was widely accepted. Further support came from Rudolf~\citet{Tomaschek1924}, a critic of relativity influenced by Philipp Lenard, who repeated the interferometer experiment using starlight \rc{39}{180}.


 
% The experiment was meant to test whether the rods satisfy the light-geometrical definition of length in every inertial system:  $OS_{1}=OS_{2}$ when  $\overline{O S_{1} O}=\overline{O S_{2} O}$. According to classical theory, the equality is satisfied only in ether system}


%However, in \datemy{5}{10}{1922}, Weyl informed him of a significant flaw in his axiomatization \citep{Rynasiewicz2005}: it was not possible to distinguish the class of inertial frames solely based on light rays as Reichenbach as claimed \letterhrp{Weyl}{Reichenbach}{5}{10}{1922}[015-68-02]. The book, 


%\begin{inparaenum}[(a)] \item one can abandon the \s{apodeictic} character of \apr without giving up its \s{constitutive} meaning; \item this shift replaces Kant's \s{analysis of reason} with \s{scientific analysis}, the study of the logical structure of scientific theories \citep[**]{Reichenbach1922}. \end{inparaenum} Philosophy, Reichenbach argues, should leave theory construction to science (without aiming to constrain it \apr) and instead clarify science’s implicit presuppositions, as his axiomatic claims to do. By limiting itself to logical structure, philosophy avoids empirical refutation but loses its grip on reality. Reichenbach ends by paraphrasing Einstein’s famous \citeyear{Einstein1921} dictum: \s{If the principles of epistemology refer to reality, they are not certain; and if they are certain, they do not refer to reality} \citep[**]{Reichenbach1922


%In the book, Reichenbach shared with \citet{Schlick1918} the idea that physical knowledge is, ultimately \s{coordination} (\german{Zuordnung}), the process of relating an axiomatically defined mathematical structure to concrete empirical reality \citep{Padovani2009}. However, Reichenbach attempted to give this insight a \scare{Kantian} twist. According to Reichenbach, in a physical theory, besides the \scare{axioms of connections} (\german{Verknüpfungsaxiome}) encoding the mathematical structure of a theory, one needs a special class of physical principles, the \scare{axioms of coordination} (\german{Zuordnungsaxiome}), to ensure the univocal coordination of that structure to reality. 

%In the book, \citet[chap.~IV]{Reichenbach1920a} shared with \citet{Schlick1918} the idea that physical knowledge is, ultimately \s{coordination} (\german{Zuordnung}), the process of relating an axiomatically defined mathematical structure to concrete empirical reality \citep{Padovani2009}. However, he attempted to give this insight a \scare{Kantian} twist, by searching for the conditions \apr for that make this coordination univocal \citep[chap.~VII]{Reichenbach1920a}. For the young Reichenbach, these coordinating principles are \apr because they are \scare{constitutive} of the object of a physical theory. However, they are not apodeictic or valid for all time \citep[chap.~V]{Reichenbach1920a}. 

%\citep[see][for a balanced account of the debate]{Acuna2016}. 



%the debate about the role of \rac in relativity theory has undergone a fundamental transformation, shifting from the problem of \emph{confirmation} (conventional vs. empirical) as raised by logical empiricism to that of \emph{explanation} (geometrical vs. dynamical).

% Contrary to the supporters of the geometrical approach 

%Indeed, Reichenbach argues one can artificially construct cases in which in which there is no Lorentz contraction but an Einstein contraction \rzl{**}{**} and viceversa \rzl{**}{**}.\todo{check is movin donw}


%The theory predict that $A'B'<AB$.  How the length of this projection. This also on the shape of a moving object indeed, Since the shape is determined by the simultaneous projection of all the points.This result is the basis of Einstein's principle of the constancy of the velocity of light; the motion of light can be considered as a spherical wave for any uniformly moving system. Our presentation enables us to visualize Einstein's result. The shape of the surface of the light wave is not uniquely determined but depends on the definition of simultaneity. Its so-called shape is always


%For a long time, the debate—rooted in logical empiricism—framed the role of \rac in \rt in terms of \textit{confirmation}, as a dispute between empiricism and conventionalism. Since Brown's work, the center of gravity has shifted toward the problem of \textit{explanation} of the behavior of \rac, recasting the discussion as a conflict between dynamical and geometrical approaches.

%Reichenbach denied the explanatory power to \Mink \spti, and argued that \sr requires the behavior of \rac to be explained by a specific, though still unknown, theory of matter. His proposal fell flat and has been largely forgotten \citep[see, however][]{Gruenbaum1955,Gruenbaum1963a}; indeed, most of his readers have focused on his well-known conventional $\epsilon$-definition of simultaneity. 



%For Reichenbach, the fact that the laws governing the field and those governing matter agree is a \textit{coincidence} that requires an \textit{explanation}.  Einstein's strategy was to transform this \textit{coincidence}, revealed by the failure of all ether-drift experiments, into a \textit{requirement} that all laws of nature must satisfy. This is the sense of \citets{Einstein1919} famous comparison between relativity and \th. Classical \th asks: How must the laws of nature be constructed in order to rule out the possibility of bringing about perpetual motion of the first kind? The laws of nature must satisfy energy conservation. \Rt asks: How must the laws of nature be constructed in order to rule out the possibility of bringing about the ether wind? The laws of nature must be Lorentz invariant. 



%


%\Sr does not need to \emph{find} an \emph{actual} constructive Lorentz invariant theory of matter describing the behavior of rods and clocks; it is sufficient to \emph{require} that all \emph{possible} constructive theory, whatever they might be, are Lorentz invariant.

%However, such a theory would be an \emph{instantiation} of Einstein's new kinematics, not an \emph{explanation} of the latter.


%This is the sense of \citets{Einstein1919} famous comparison between relativity and \th. Classical \th asks  should the laws of nature be laid of the first kind must be impossible: the laws of anture sbou satisfy the energy conserati. The special relativity laws of nature be laid if ether drift should be impossible; the laws of nature should be Loretnz invariant.



%This also explains the negative result of the Michelson-type experiments: that the laws governing matter hold would make the requirement meaningful, to find a pererla motion ac. Reicenac simply accept that it is merely a coincidence, that he would s\todo{improve}



%\begin{itemize}
%\item Lorentz maintained the old kinematics, thus, need to provide a \emph{dynamical explanation} for why the arms of an interferometer contract by the right amount to compensate for the difference in the two-way velocity of electromagnetic radiation. Since the laws governing radiation—\ME—\emph{happen to be} Lorentz invariant, a \emph{particular} electromagnetic theory of matter could explain why the material arm of the interferometer contracts exactly in the same way as the electromagnetic field.
%
%\item Einstein, by contrast, doubted the validity of \ME from the outset, and derived a new kinematics based on the Lorentz transformations independently of any actual theory of matter or radiation. Einstein theory contains only the requirement that \emph{all} possible laws of matter and radiations, whatever they may be, \emph{must} be Lorentz invariant. According to Einstein, Minkowski’s contribution was to have developed a vector calculus that helps to check whether a theory is actually Lorentz invariant, without the need to carry out the transformations by sheer calculation. 
%\end{itemize}
%%
%Einstein neither embraced \Mink's claim to have provided a \emph{geometrical explanation}. However, Einstein did not feel the need for a \emph{dynamical explanation}; indeed, he more or less explicitly emphasized that this was the advantage of his approach over Lorentz’s. 

%Lorentz’s theory stands or falls with his \emph{particular} Lorentz-invariant theory of matter and radiation. Thus, it needed to hypothesize about the structure of matter, the nature of molecular forces, etc. Einstein’s theory, by contrast, requires that \emph{any} theory of matter or radiation must be Lorentz invariant. Thus, no such detailed assumptions were needed. Certainly, setting up a new kinematics is not sufficient. Ultimately, one must complete the theory by searching for the particular theory of matter and radiation that happens to hold in nature; the equations of such theory might have solutions corresponding to material systems that we can use as \rac, thereby closing the circle. However, such a theory would be an \emph{instantiation} of Einstein's new kinematics, not an \emph{explanation} of the latter. 



%Were, that fact tha ..how be lod if is post. Nad How s that be imps the reqiret of Lroetz invariece. 



%Reichenbach, so to speak, \s{puts asunder what God hath joined together}.



 
%Reichenbach concluded that while Lorentz provided a bad dynamical explanation, Einstein provided no explanation at all, and thus his theory was still incomplete. 

%He argued that also in special relativity we needed a future \emph{dynamical explanation}, that is, to find a specific theory of matter that explains why \rac exhibit relativistic behavior. However, this indeed misses the point. 



%


%In both cases, the demand for an explanation and a detailed theory adds nothing further to the explanation itself. This characteristic can also be extended to general relativity. General relativity imposes fundamental constraints on any possible theory of matter and fields: (1) the laws must be generally covariant (\gc); (2) second derivatives of \gmn must not enter into the field equations (minimal coupling). If one asks why the gravitational field is a universal force—that is, why it affects \rac of any material in the same way—the explanation is that the components of the Riemann tensor do not explicitly enter into the formulation of the laws governing the material systems we use as rods and clocks. In this sense, they \s{adjust} to the gravitational field. Also in this case, we have an \emph{explanation by constraint}. Finding a particular theory of matter that satisfies this requirement does not add further explanatory value.











%\appendix


%\section{Linearity}
%
%% !TEX root = reichenbach_explanation.tex

%If bodies did not behave like $l$ in the relativistic sense, but like $L$ in the classical sense, there would be ne the Einstein contraction that $l$ is a rigid rod.

%


%Thus, the relativistic length of a moving rod from the rest frame is $\lK{}{'}$; the relativistic length of a stationary rod measured in the moving system is $\lK{'}{}$:
%\todo{The notation serves to highlight the relational nature of the notion of length}

%\begin{equation*}
%\lK{\notateol{'}{0.5}{\texts{measured in the moving frame}}}{\notateul{\textcolor{white}{'}}{0.5}{\texts{at rest in the rest frame}}} \quad \quad \LK{\notateor{\textcolor{white}{'}}{0.5}{\texts{measured in the rest frame}}}{\notateur{'}{0.5}{\texts{at rest in the moving frame}}} 
%\end{equation*}
%


In order to provide a proof of this claim, Reichenbach resorts to the somewhat idiosyncratic notation introduced in his \citeyear{Reichenbach1924} monograph. He labels $l$ a rod following the Lorentz–Einstein theory, and $L$ one with classical behavior. $K$ and $K'$ are resp.\ the stationary and moving systems. Then he uses index notation where the upper index $^{K}$ marks the frame in which the rod is measured, the lower the frame in which the rod is at rest $_{K}$. In both the classical and Lorentz–Einstein theories, the lengths of unit rods in the rest frame $K$ are equal, or $\lK{}{} = \LK{}{} = 1$. The difference emerges when one considers the length of the rod in the system in motion $K'$:

\begin{itemize}
\item \emph{classical theory}:  \hide{the rest length of the rod as measured from the moving system is equal to the rest length in the rest frame $\LK{'}{'}=\LK{}{}$
%
%\footnoteh{In classical mechanics $x' = x - vt$, then at the time $t = 0$, when $K$ and $K'$ coincide, we have $vt = 0$, and thus $x = x'$}
%
,} the rod has a unique length, regardless of motion:

\begin{equation}\label{eq:CT}
\frac{\LK{}{}}{\LK{'}{'}} = \frac{1}{1} \,\,\ \text{no contraction}
\end{equation}
%
\item \emph{Lorentz theory}: the length of the moving rod in the moving system $K'$, is \textit{contracted} \hide{$\lK{'}{'}<\LK{'}{'}$ by a factor \kappafactor} with respect to the length that it would have in the classical theory in the same frame $K'$:

\begin{equation}\label{eq:LT}
\frac{\lK{'}{'}}{\LK{'}{'}}= \frac{\kappafactor}{1} \,\,\ \text{Lorentz contraction}
\end{equation}

\item \emph{Einstein theory}:  the length of the moving rod measured in the rest system $K$ is contracted \hide{$\lK{}{'}<\lK{'}{'}$, by a factor \kappafactor} with respect to the length of the moving rod measured in the moving system system $K'$:

\begin{equation}\label{eq:ET}
\frac{\lK{}{'}}{\lK{'}{'}} =\frac{\kappafactor}{1}  \,\,\ \text{Einstein contraction}
\end{equation}
%
\end{itemize}
%
and viceversa\footnoteh{The length of the rest rod measured in the moving system $K'$ is contracted $\lK{'}{}<\lK{}{}$, by a factor \kappafactor with respect to the length of the moving rod measured in the moving system system $K'$: $\frac{\lK{'}{}}{\lK{}{}} =\frac{\kappafactor}{1}$.}. Reichenbach aims to prove that the fact that Lorentz contraction and Einstein contraction amounts to the same factor \kappafactor, is the consequence of the linearity of the Lorentz transformations. His proof runs as follows. According to classical theory we have $\LK{'}{}=\LK{}{}$, and since $\lK{}{}=\LK{}{}$ we obtain $\LK{}{'}=\lK{}{}$. Relation \cref{eq:ET} therefore becomes

\begin{equation}\label{eq:ETCT}
\frac{\LK{}{'}}{\lK{}{'}} = \kappafactor
\end{equation}
%
Because of the linearity of the transformation ratio \cref{eq:ETCT} is the same as ratio \cref{eq:LT}, which means that \cref{eq:ET} is also the same as ratio \cref{eq:LT}: the ratio is the same in all case and in particular is equal to \kappafactor.  However, Reichenbach insists that there is the deep \emph{conceptual difference} between the two contractions despite the their coincidental \emph{numerical equality} \rc{45f.}{189f.}.




%Indeed, shows that one can imagine cases in which in which there is no Lorentz contraction but an Einstein contraction\footnote{With not Lorentz contraction; the Einstein contraction from $K'$ to $K$ would also disappear. However, in the classical theory \cop{we might define the simultaneity in $K^{\prime}$ according to Einstein's convention, setting $\epsilon=\frac{1}{2}$, the inverse comparison from $K$ to $K^{\prime}$ will show the Einstein contraction}. The magnitude of which is the square of that of the Lorentz-Einstein contraction\todo{why?}.} and viceversa\todo{improve}.






%For example, one can often use them to get a rough preliminary idea of the answer. But one should beware of trying to use them for everything, for their utility is limited. Analytic or algebraic arguments are generally much more powerful.


%This paper hope to have made the case that Reichenbach defened a dynamical interpretation of special relativity



%the length of a moving rod is the distance between its endpoints measured simultaneously in a particular frame (in which, of course, simultaneity is defined by clock synchronization using light signals)\todo{improve}:

%It was one of Einstein's philosophical insights to realize that we are free to call the length of the moving rod in the moving coordinate system one if we wish, despite its having a coordinate difference of only $\sqrt{ }\left(1-v^2-c^2-\right)$. If we make this decision, then while we

%It was one of 

%

%In discussing the Lorentz contraction we did not talk about the length of the rod in the moving system, but merely about its coordinates in a particular coordinate system. It was one of Einstein's philosophical insights to realize that we are free to call the length of the moving rod in the moving coordinate system one if we wish, despite its having a coordinate difference of only / (1 - 02/c?). If we make this decision, then while we


%The difference between the two contractions is clear: Lorentz contraction means that the world-strip of the relativistic length is shorter than that of the classical length in the same system $K'$. On this both, theories agrees. Einstein’s contraction refers to the fact that the same shorter strip can be in a different way by the plane of simultaneity, that is by horizontal space axis of \Mink diagram if the rod $K'$ or $K$: 

%Einstein could achieve this result, by introducing \emph{by convetion} a new definition of length, which requires measurement at the same time. In this setting, identical rigid rods, measured the distance between different points: with respect to the co-moving frame $K'$  the end points $O$ and $S^{\prime}$ of a moving bar are simultaneous; with respect to rest frame $K$, $O$ and $S_{1}$ are:

%Indeed, with the same rod, we measure different things. This is a perspectival difference between the length of the moving rod in the moving frame $K'$ and of its projection in the rest frame $K$.  
%
%\cop{It has been objected to previous remarks of mine on this subject ${ }^1$ that it is impossible to compare two magnitudes belonging to different theories. This objection is incorrect. By reference to a third body, we are able to establish a comparison, if we calculate how the two magnitudes under consideration compare with that third body. Furthermore, this mode of expression is frequently used in physics. We may say, for instance, that a highly-compressed gas behaves differently than it would according to Mariotte's law. This means nothing but that the real gas $g$, when compressed to a certain degree, occupies a larger volume than a gas $G$ which satisfies the MariotteBoyle relations. The third body used in this comparison is the rigid}. \cop{Measuring rod which measures the volume. The third body is not always explicitly mentioned, and the abbreviated formulation is often preferred, because it clearly suggests a difference in actual behavior. The Lorentz contraction must indeed be considered a real difference in this sense. In this case, the tertium comparationis is light, which in terms of light-geometrical definitions supplies a standard to which the rods of the different theories can be compared. }. 
%
%In this case, the fertium comparationis is light, which in terms of light-geometrical definitions supplies a standard to which the rods of the different theories can be compared. It would be an

%It would be an incorrect mode of speech, however, to say that the Einstein contraction is an apparent difference. This contraction has nothing to do with the difference between the real and the apparent, but results from a difference in the conditions of measurement. We shall speak of a metrogenic difference because this difference originates in the nature of the measurement. Since we are specifically concerned with kinematic


%

%A lightlike line in a Minkowski diagram has equal projections onto the space and time axes when using units where \( c = 1 \). Rods and clocks at rest in a moving system measure the same speed of light as those at rest in the original frame. Even when lengths and times are projected differently due to relative motion, the ratio of distance to time—used to measure the speed of light—remains \( c \).

%\footnote{Needless to say, a moving system $K^{\prime\prime}$, being slower, would correspond to a line that is less tilted, while a faster one $K^{\prime\prime\prime}$ would be more tilted than $K'$ in the given frame. }. \todo{improve}

%The effect is perspectival: $OS_{1}$ is the projection of $OS^{\prime}$ onto the horizontal axis \rzl{220}{190}. If one place a rod along $OS^{\prime}$ and the the \emph{same} rod $OS_{1}$, one discover $OS_{1} < OS^{\prime}$.

%\cop{locate the right-hand boundary , parallel to $OQ^{\prime}$ of the rod along the tangent to the hyperbola that passes through $S^{\prime}$; classical theory asserts that the boundary must pass through $S$}.

%The effect is reciprocal: the rest length $OS = \lK{}{}$, as seen from the moving frame, appears as the projection $O^{1}_{3} = \lK{'}{}$, so $O^{1}_{3} < OS$. For co-moving observers, both effects vanish: $OS = OS^{\prime}$ when at relative rest.
%
%
%; the length as measured by a rod at rest in the rest system $K$ ($\lK{\prime}{}$) is \q{symbolized} by
%
%}\footnote{That is, \sr makes the specific prediction that $OS_{1} < OS$, whereas the classical theory would predict that they have equal length; indeed, that if one cacept Einstein defin $\epsilon$ definiton of   simultaneity this a specific prrection can be true or flase.}


%That is if measured if $OS = OS_{1}$ if they are compared at relative rest: $\lK{\prime}{\prime} = \lK{}{}$.


%It is because of this analogy that we can \emph{graphically represent} $a$ via $b$, if the latter are easier to visualize. E.g. the same Euclidean geometry $A$—that is, an $n$-dimensional $ds^{2} = \sum_{\nu} dx_{\nu}^{2}$—can represent: ($a$) rigid rods measuring $ds$, ($b$) thermodynamic relations, ($d$) electric pipes, etc. The graphical representation of, say, thermodynamic relationships $b$ in a $PV$ Euclidean diagram $c$ is based on the fact that both are realizations of the same system $A$ of Euclidean geometry. In other temerm while usally of a vertical coordiantes, Eucldae with \rac (a); the graphcial rpreatiat is a horizioanl coordiantion.\todo{improve}

%by adopting Einstein's conventional definition of simultaneity one could still claim that the motion of light is a spherical wave for any uniformly moving system. Also the assertion that that the velocity of light as a \emph{limiting} velocity could also be maintained. Indeed, this empirical fact is confirmed by measurements on fast traveling electrons ($\beta$ rays), whose kinetic energy approaches infinity as their velocity nears the speed of light. The assertion the velocity of light has always a \emph{numerical value} $c$, if measured by \rac on the contrary would be refuted.


\hide{We have now succeeded in distinguishing between the physical assertions of the relativistic theory of space and time and its epistemological foundation. This epistemological foundation is supplied by the discovery that coordinative definitions are needed far more ... The physical core of the theory, however, consists of the hypothesis that natural measuring instruments follow coordinative definitions different from those assumed in the classical theory. This statement is, of course, empirical. On its truth depends only the p11ysic1tl theory of relativity. However, the philosophical theory of relativity, i.e., the discovery of the definitional character of the metric in all its details, holds independently of experience. Although it was developed in connection with physical experiments, it constitutes a philosophical result not subject to the criticism of the individual sciences.}
%
%\begin{figure} \begin{center} \includegraphics[scale=0.33, trim = 0mm 0mm 0mm 0mm, clip]{graph/mr2} \caption{Realization of the indefinite metric by means of light rays, clocks and measuring rods; from \cite[215]{Reichenbach1928}; the label $S_2$ has been added} \label{mr} \end{center} \end{figure}

%A few months later, Reichenbach submitted a popular paper on the Miller's experiment entitled \citefulltitle{Reichenbach1926a} which appeared in the weekly magazine \citejournal{Reichenbach1926a} in \datemy{24}{4}{1926}. By that time, Reichenbach was fully aware of what \qt{Einstein himself has recently said in the newspapers,}{Einstein selbst hat k\"{u}rzlich in den Tageszeitungen ausgesprochen}; however, he saw no reason to abandon his \qt{less radical opinion}{eine weniger radikale Ansicht entwickelt} \crw{327}{202}, namely that \qt{\myemph{Miller's result in no way affects the philosophical consequences of the theory of relativity}}{Auf keinen fall werden jedoch die philosophischen Konsequenzen der Relativit\"atstheorie von den Millerschen Versuchen betroffen} \crw{328}{203\me}. It would only imply a change in our knowledge of the physical mechanism governing rods and clocks \citep[308]{Hentschel1982}.


%How can this negative result be explained? Classical aether theory attempted to account for it by postulating that the aether is dragged along with the Earth in its orbit. However, this stratagem contradicts the phenomenon of stellar aberration \crw{**}{**}. As a result, 19th-century optics found itself in a seemingly irresolvable dilemma. According to Reichenbach, the standard view in physics was to preserve the existence of a stationary aether by introducing auxiliary hypotheses—such as length contraction—to explain why the experiment yielded no observable effect.

%According to Reichenbach, the received view of how physics attempted to resolve this puzzle was as follows: the equality holds in all frames.

%This principle must now be formulated in a more exact manner. The velocity of light is identical in all directions in a uniformly moving frame of reference, provided simultaneity is correspondingly defined. This additional


%When moving uniformly through the ether with velocity $v$ with respect to the ether frame $K$ (in which light propagates rectilinearly with velocity $c$), the arm $\mathrm{OS}_2$ shrinks in the direction of its motion, just as a metal bar contracts when exposed to cold, and this compensates for the slower speed of light rays in that direction—accounting for the null result of the experiment. The deformation of the rod cannot be ascertained by measurement—for instance, by rotating the non-contracted $\mathrm{OS}_1$ around $O$ to check whether $S_{1}$ can be brought into coincidence with $S_{2}$. Indeed, in this case, $\mathrm{OS}_1$ would also be contracted by the same amount as $\mathrm{OS}_2$. Thus, Lorentz justified the contraction hypothesis on the grounds that the molecular forces which hold material bodies together are electromagnetic in nature and are affected by translational motion. Lorentz could then conclude that there exists an aether wind over the surface of the Earth, but it is not detectable.


%Whereas in Einstein's original theory of relativity light served merely to determine simultaneity, it became clear in later developments of the theory that light can be used for all measurements of time—for defining the unit of time, and even for the measurement of space. One may construct a geometry of light in which light determines the comparison of spatial distances. Thus, light comes to function as the ordering framework of physics, gathering within the mesh of its rays all the events of the world and arranging them in numerical order.

%According to this interpretation the measurement of space can be reduced to the measurement of time; spatial distance can be measured by the time needed for a light signal to travel a certain segment.  Definition of the equality of spatial distances in terms of measurements of time intervals.




%Although his attempt fell flat and has been mostly forgotten \citep[see, however][]{Gruenbaum1955,Gruenbaum1963a}, Reichenbach should be considered the foremost forerunner of the \emph{dynamical} interpretation of relativity. According to the usual interpretation, Reichenbach provided a conventionalist reading of special relativity. He addressed the issue of \emph{coordination} between the abstract mathematical structure of relativity and the behavior of material systems like light rays and \rac, arguing that the definition of simultaneity is conventional. This interpretation is not incorrect but partial. Reichenbach was the first to engage explicitly with the problem of \emph{explanation} for the fact that the relativistic behavior of material systems like \rac. According to Reichenbach, Lorentz theory is inadequate as it provides a \emph{bad explanation} of this behavior in terms of the deviation from a standard classical behavior of \rac. However, Einstein theory also falls short by offering \emph{no explanation} of this and simply by asserting by decree that theory relativistic behavior is of \rac the standard behavior. Reichenbach argues that also in \sr, a \textit{proper explanation} is required. A geometrical explanation, he argued, was no more than a graphical illustration; thus a proper explanation must be a dynamical rooted in a yet unknown theory of matter.

%According to Reichenbach, Lorentz theory is inadequate as it provides a \emph{bad explanation} of this behavior in terms of the deviation from a standard classical behavior of \rac. However, Einstein theory also falls short by offering \emph{no explanation} of this and simply by asserting by decree that theory relativistic behavior is of \rac the standard behavior. Reichenbach argues that also in \sr, a \textit{proper explanation} is required. A geometrical explanation, he argued, was no more than a graphical illustration; thus a proper explanation must be a dynamical rooted in a yet unknown theory of matter. 

%\epsilon-definition of simultaneity


%Ether theory predicted such a shift due to Earth's motion through the ether, as the beam toward $M_2$ would return slightly later—thus revealing absolute motion.

%\textcites{Reichenbach1924}{Reichenbach1926a} account of the \MEE is quite conventional. It is, however, important to briefly reconstruct it, since his goal is precisely to show that the conventional presentation is incorrect. As is well known, the essential idea of the experiment was to test whether the speed of light is the same in all directions relative to the Earth's motion through the ether. The reader would find \textcites{Reichenbach1924}{Reichenbach1926a} offer a quite conventional account; however, this is important since Reichenbach's goal is precisely to show that the conventional presentation is wrong. In the setup, a beam of light is split at point $O$ by a semi-transparent mirror, sending two coherent beams along two perpendicular arms of equal length, each terminating at mirrors $M_1$ and $M_2$. The beams are reflected back to $O$ and recombined to produce an interference pattern. The experiment tests whether the round-trip travel time depends on the apparatus's orientation relative to the Earth's motion through the ether. If light speed varied with direction, rotating the device would shift the interference fringes \rw{325}{195f.}. According to ether theory, the beams return together only if the apparatus is at rest with respect to the ether. Since the Earth moves through the ether, the theory predicts a \s{deviation}: the beam toward $M_2$ returns slightly later, shifting the fringes. Thus, the interference pattern could reveal the Earth's absolute motion through the ether.


% According to classical theory, (1) is satisfied in only one of the inertial systems, namely, the system at rest relative to the ether. In all other systems the rest-length of one of the branches of the coordinate axes will no longer satisfy (1). Since the Michelson experiment has been confirmed to a very high degree, we could consider this matter closed, because it has no bearing upon epistemological considerations, if it had not been given certain erroneous interpretations in the usual discussions on relativity.

%Michelson experiment. 1 

%is thereby considered well con-firmed. Further, the result has received long-awaited confirmationby the experiments of Tomaschek using the light emitted by stars, which were inspired by the work of Lenard.5 Recently, doubts have been raised by Dayton C. Miller,6 who obtained a positiveresult on Mount Wilson;

%Axioms $IX$ and $X$ can be tested by transverse Doppler effect

%\section{Two Types of Explanati}
 
 
%\cop{The relativity of simultaneity has nothing to do with the contraction in Michelson's experiment, and Einstein's theory explains the experiment as little as does that of Lorentz.}. \cop{The above opinion is incorrect because the contraction of the arm of the apparatus occurs for the moving system relative to which the apparatus is at rest. The Einstein contraction would explain a contraction of the arm only if it were measured from a different system and is therefore not sufficient to explain the Michelson experiment. This experiment proves that a rod which lies along the direction of the motion is shorter than it should be according to classical theory, if it is measured relative to the rest-system. In other words: the comparison}. \cop{there were a special inertial system $I$ that could be regarded as an absolute rest-system, and if we had in this system two equally long rigid rods, one of which behaved according to classical theory and the other according to Einstein's theory, the two rods would cease to be equally long if they were brought into any other inertial system $S$, provided that they lay along the direction of the motion of $S$. The Einstein rod would be shorter. The difference could be measured in $S$ as the difference between their rest-lengths, and in any other system as the difference between the lengths of the moving rods. Einstein's theory as well as Lorentz's theory therefore assumes the behavior of rigid rods to be measurably different from their behavior according to classical theory; but the difference has nothing to do with the definition of simultaneity}.

%Michelson experiment. 1 This experiment proves that the rods satisfy the light-geometrical definition of congruence (cf. Fig. 29, page 170) where
%
%$$
%A B=A C \text { when } \overline{A B A}=\overline{A C A}
%$$
%
%in every inertial system, for any orientation of the coordinate axes. According to classical theory, (1) is satisfied in only one of the inertial systems, namely, the system at rest relative to the ether. In all other systems the rest-length of one of the branches of the coordinate axes will no longer satisfy (1). Since the Michelson experiment has been confirmed to a very high degree, we could consider this matter closed, because it has no bearing upon epistemological considerations, if it had not been given certain erroneous interpretations in the usual discussions on relativity.


%\subsection{Dynamical and Geometrical Explanation in Special Relativity}


%That this strucure, a particuarl relatiosn that by a definite but an indefinte metric %This assertion has the consequence that not only the time axis can be rotated—this can also be done in the classical space-time theory, since it merely means that any moving system system may be chosen as the one at rest—but even the space axis can be rotated. This statement means that simultaneity is arbitrary within a certain angular interval







%By asserting that measuring rods, clocks, and light rays behave according to the relations of congruence of the indefinite metric, Minkowski space-time provides a \emph{graphical representation} of the agreement between relativistic light geometry. However, \Mink like Einstein but provide an explanation of such \emph{agreement}.



%How can \textit{identical} rods and clocks in relative motion both measure the same speed of light $c$? By introducing the relativity of simultaneity, Einstein showed that observers in relative motion assign different time intervals and spatial distances to the \textit{same} pair of events—ensuring that both still measure the speed of light as $c$.

%The role of hyperbolas as representing loci of events with the same spacetime interval from the origin $O$ 

%How can the same rod and same clock appear in different inertial frames — where the speed of light $c$ is constant in all frames — and yet preserve their length or proper time, despite transformations? o the proper length of the rod and the proper time of the clock are frame-invariant quantities — meaning they are not the same as the length and time measured in other frames.



%he rod has a fixed spatial length at each universal time slice.

%Gray hyperbolas: Surfaces of constant timelike interval ($ct^2 - x^2 = (text{const}$) • Light gray hyperbolas: Surfaces of constant spacelike interval ($x^2 - ct^2 = (text{const}$)

%According to classical theory, the moving rod is not represented by the strip between the world-lines $O Q^{\prime}$ and $S_1 S^{\prime}$, but by the wider strip bounded by $O Q^{\prime}$ at the left and $S S^{\prime} 2$ at the right. This follows, according to classical theory. because ..

%Indee,d there is only one plane of simultaneity, is by the ... that indee.. 


%Such \s{Minkowski diagrams} can be extremely helpful and illuminating in certain types of relativistic problems. 




% !TEX root = reichenbach_explanation.tex


%In the booklet, Reichenbach, like \citet{Schlick1918} conceives physical knowledge as (\emph{Zuordnung}), the process of relating an axiomatically defined mathematical structure to concrete empirical reality \citep{Padovani2009}. However, Reichenbach attempted to give this insight a \scare{Kantian} twist. According to Reichenbach, in a physical theory, besides the \scare{axioms of connections} (\german{Verknüpfungsaxiome}) encoding the mathematical structure of a theory, one needs a special class of physical principles, the \scare{axioms of coordination} (\german{Zuordnungsaxiome}), to ensure the univocal coordination of that structure to reality. For young Reichenbach, the latter axioms are \apr because they are \scare{constitutive} of the object of a physical theory. However, they are not apodeictic or valid for all time \citep{Padovani2009}.

%19On the session on relativity theory at the soth convention of the scientine society Gesellschaft Deutscher Naturforscher und Arzte at the spa town of Bad Nauheim on Sep. 20, 1920 see, e.g., Hermann Weyl's eyewitness account published in Umschau 24 (1920], pp. 609-611 as well as Gehrcke and Weyl, ibid. 25 (1921], pp. 99, 123, 227, Pincussen (1920], and Weyland (1920)c on the supposed 'stifling' of Einstein's opponents; see also Goenner [1993), pp. 123-127 and references there.


%\todo{1927 tood spaceitme}
%
%\subsection{Minkowski Space-Time as a Graphical Representation}
%
%
%We do not know whether Reichenbach ever became aware of Schlick's negative opinion of his interpretation of \sr. However he clearly did not made any conciliatory steps.  Reichenbach's line of argument can be found again in the \citetitle{Reichenbach1928} \citep{Reichenbach1928}, which, as a letter from Reichenbach to Schlick reveals, was already finished at the end of 1926 (\letter{Reichenbach}{Schlick}{6}{12}{1926}[][SN]), though Reichenbach was only able to find a publisher months later. When the book was finally published that the beginning of 1928 the importance of Miller's experiment was fading away especially in Germany. However, Reichenbach's stance is of course not changed. There is no difference between Einstein and Loretnz theory. Lorentz provided a bad explanato Einstein did not provide any explanation. 
%
%
%
%%The distinctive role of light in the theory of relativity can be expressed in another way. While in Einstein's original formulation of relativity, light was used primarily to define simultaneity, the later refinement of the theory revealed that light could serve as the basis for all measurements of time and the measurement of space. One could construct a \s{light geometry}. The physical tools for measuring time and space play only a secondary role. The Lorentz-Einstein theory claims that, magnetic needle aligning itself with the magnetic field, they adapt to the geometry of light, \rac adapt to the relativistic light geometry. The difference between Lorentz and Einstein is that where that a deflection from the true classical light geometry, Einstein from the declare the relativistic light geometry to be are rigid. By that of time and depend length interval matter on the same definition of simultaneity as light geometry: \q{This idea can be expressed mathematically by bringing together space and time into a four-dimensional structure, into a space-time manifold} 112.  
%
%
%
%One might argue that not Einstein but \Mink provied such explanation, dynamical but a geometrical explaation. The name of \Mink is surprisingly never mentioned in his 1924 book. The most improta how the content of the light- and matter-axioms can be \q{visualized geometrically} by the world- geometry of \Mink. Reichenbach views \Mink-\spti  as nothing but a \scare{graphical representation}, an expression he borrowed from Arthur Stanley \citet{Eddington1925a}. Reichenbach defines \scare{graphical representations} as structural analogies between different physical systems (e.g., compressed gases, electrical phenomena, mechanical forces, rigid bodies and light rays, etc.), which are realizations of the same conceptual system (e.g., the axioms of Euclidean geometry) \rzl{123ff}{101ff}. In the case of Minkowski space-time, if \qt{we speak of a geometrization of physical events, this phrase should not be understood in some mysterious sense; it refers to the identity of types of \emph{structure} and not to the \emph{identity of the coordinated physical elements}}{Wenn man von einer Geometrisierung des Weltgeschehens gesprochen hat, so darf dies auf keinen Fall in irgendeinem geheimnisvollen Sinn aufgefaßt werden; es besagt nur die Identitat von Strukturtypen, nicht der zugeordneten dinglichen Elemente.} \rzl{220}{190}. By asserting that measuring rods, clocks, and light rays behave according to the relations of congruence of the indefinite metric, Minkowski space-time provides the geometrical representation of the light and matter axioms. When in \cref{mr} we \scare{symbolize}, say, the motion of rod $OS$ with its rotation of the segment around $O$, \qt{we only g\textins{ive} a graphical representation, which means that the logical structure \origins{Beziehungsgef\"{u}ge} exhibited by the rods \textelp can also be realized by the space-time manifold}{so wird sie nur graphisch dargestellt; es wird damit behauptet, da§ das Beziehungsgef\"{u}ge, welches r\"aumliche St\"abe von der Art des vorangehenden \S~28 enthalten, zugleich auch; von der Raum-Zeit-Mannigfaltigkeit realisiert wird.} \rzl{220}{190}.
%
%%Such \s{Minkowski diagrams} can be extremely helpful and illuminating in certain types of relativistic problems. For example, one can often use them to get a rough preliminary idea of the answer. But one should beware of trying to use them for everything, for their utility is limited. Analytic or algebraic arguments are generally much more powerful.
%
%
%
%The advance of this logical structure, Einstein only because, he redefined the length of simultaneity that it became advantageous to combine space-time into a single geometrical structure, that is a four-dimensional manifold, that \Mink called the \scare{World} or \st (\S24). This simply means that it takes four numbers, the so-called \emph{coordinates}, $x_\nu$ (where $\nu=1,2,3,4$) to identify a \wpo, namely three numbers $x_1,x_2,x_3$ for its spatial location and one for time $x_4$. The set of points whose coordinates are defined by $x_\nu(s)$ where $s$ is an arbitrary parameter is called a \wl{}. At this stage, coordinates $x_\nu$ are nothing but identification numbers, that is, in Reichenbach's parlance, they have only a \scare{topological} function; they determine the order of the \emph{between} relation. In order to define distances between points, one needs to introduce an expression that assigns a number $ds$ to the coordinate differences $dx_\nu$ between two close \wpo{}s, the so-called \scare{fundamental metrical form}. In the case of \Mink \spti, it is always possible to choose the coordinate numbers \xn so that any distance satisfies the relation.
%
%That the behavior of small \rac and satisfy the following relations:
%
%\begin{figure} \begin{center} \includegraphics[scale=0.33, trim = 0mm 0mm 0mm 0mm, clip]{graph/mr2} \caption{Realization of the indefinite metric by means of light rays, clocks and measuring rods; from \cite[215]{Reichenbach1928}; the label $S_2$ has been added} \label{mr} \end{center} \end{figure}
%
%\begin{equation}\label{eq:mink}
%\diff s^{2} = \diff x_{1}^{2}+\diff x_{2}^{2}+\diff x_{3}^{2}-\diff x_{4}^{2}\,.
%\end{equation}
%%
%This metrical (that is \scare{measurement}) formula allows to calculate the distance $ds$ between two nearby points from their coordinate differences $dx_\nu$. A metric that has positive as well as negative signs is called \scare{indefinite} \rzlp{**}{188--189}. The physical realization of the negative $\diff s^2$ is a physical object that satisfies the relations of congruence defined by the hyperbolas of quadrants \rom{1} and \rom{2}. The realization of the positive $\diff s^2$ is a physical object that satisfies the relations of congruence defined by the hyperbolas of quadrants \rom{3} and \rom{4}. The first is called a time-like interval $\diff s^2 = -1$ and is realized by the proper time of a clock. The rotation of the interval $OQ$ into the position $OQ'$ represents a moving clock. $\diff s^2 = 1$ is the space-like interval and is realized by the proper length of a rod. The rotation of interval $OS$ into $OS'$ sets the rod into motion. Light rays realirze $\diff s^2 = 0$, the limiting velocity, which cannot be reached but only approached arbitrarily closely. Otherwise rods and clocks behave by following the hyperbolic contour lines.
%
%%As Reichenbach rightly notices, there is a deep disanalogy between clocks and rods. Clocks are intrinsically four-dimensional measuring instruments, since they measure distances between two events. Measuring rods, on the other hand, are three-dimensional measuring instruments; they can be treated as four-dimensional instruments, if events are produced at their endpoints according to the appropriate definition of simultaneity \rzl{217}{187}. It is from this difference that all difficulties arise concerning the behavior of rods.
%
%In Minkowski space-time the history of a uniformly moving unit rod is represented by a world-strip bounded by the parallel world-lines of the rod's endpoints. In the Lorentz-Einstein theory---keeping in mind that the points on the hyperbolas are at distance $1$ from $O$---the moving rod is represented by the narrower strip between the world-lines $OQ'$ and $S_1S'$; according to the classical theory it is represented by the wider strip between $OQ'$ and $SS'_2$. The Einstein contraction maintains that the moving rod $OS'=1$ looks shorter from the perspective of the rest frame ($OS_1$) than the proper length of the rod $OS=1$ (\lK{}{'}<\lK{}{}). The Lorentz contraction refers to the fact that the classical length $OS'_2$ would be longer than the relativistic proper length $OS'=1$, if both were measured in the same moving frame (\lK{'}{'}<\LK{'}{'}) \rzl{225}{195}.
%
%
%%In geometrical terms, the Einstein contraction compares the width of different three-dimensional simultaneity cross-sections of the same relativistic world-strip (it is a perspectival difference); the Lorentz contraction compares the width of same three-dimensional cross-section of different world-strips (it is a real difference). \Cref{mr} reflects the fact that in the classical theory there is neither a Einstein nor a Lorentz contraction ($OS=OS_2=OS'_2$). In the Lorentz-Einstein theory the Einstein contraction is  Lorentz contraction ($OS'_1<OS'_2 \rightarrow OS_1<OS$). However, the two contractions are not identical. (a) the length of the moving rod $\lK{}{'}$ measured from the rest frame is different from its proper length $\lK{}{}$. The rest-length of the moving rod $\lK{'}{'}$ is different from the rest-length of another rod $\LK{'}{'}$ which moves with it but satisfies the classical theory. 
%%
%%In Reichenbach's view, (b) can either be \scare{true} or \scare{false} in both Lorentz and Einstein's theories, depending on whether one accepts, say, Michelson or Miller's experimental results. In the geometrical representation, it is indicated by the difference between the distances $\lK{'}{'}=OS'$ and $\LK{'}{'}=OS'^2$ (\cref{mr}). On the contrary, (a) is neither \scare{true} nor \scare{false}; it depends on the definition of simultaneity adopted (one can choose the standard Einsteinian definition or an alternative one). In the geometrical representation, statement (a) is equivalent to the comparison of $\lK{}{}=OS$ and $\lK{}{'}=OS'$. 
%%
%%Lorentz believed that lengths mentioned in (a) are different because the lengths mentioned in (b) are: the lengths of the arms of the interferometer are equally long only in the ether frame. On the contrary, Einstein declared the lengths mentioned in (a) are equal if measured at relative rest: the proper lengths of the arms of the interferometer are the same in all inertial frames \rzl{229--230}{198--199}. Reichenbach, however, made a further statement embodying the peculiarity of his approach: \qt{\myemph{It is sometimes overlooked by proponents of the theory of relativity that statement (b) is nevertheless true}}{dagegen wird von relativistischer Seite gewohnlich ?bersehen, da§ trotzdem die Behauptung b gilt} \rzl{230}{198} in Einstein's theory as well.
%%
%%To sjow, Reichenbach repeats the proof of this statement, as it appeared in his 1925 article \rzl{230f.}{199f.}; however, in order to emphasize the difference between the two contractions, he constructs a counterexample in which an \scare{Einstein contraction} appears but there is no \scare{Lorentz contraction}. The example is based on the possibility of using Einstein's definition of simultaneity or an alternative one ($\epsilon \neq \nicefrac{1}{2}$) in the classical theory \rzl{231}{200}. Reichenbach's conclusion can be understood without entering into too much detail:
%
%\qrz{The example \textelp makes it particularly clear that the Einstein contraction is a metrogenic phenomenon. In the geometrical representation this means that we may choose as the length of the rod differently directed sections through the world-strip of the rod. On the other hand, the geometrical representation of [\cref{mr}] shows very clearly that through the difference in the width of the strip, the Lorentz contraction indicates a difference in the actual behavior of the rod. These considerations also explain how it is possible to compare rods $I$ and $L$, although only one of them is physically realized. $OS$ is the same in both theories; the classical theory claims that the right-hand boundary of the strip parallel to $OQ'$ must be drawn through $S$, whereas the new theory places the boundary along the tangent to the hyperbola which passes through $S'$}{Gerade dieses Beispiel \textelp macht es besonders deutlich, daß die EinsteinVerk\"{u}rzung eine metrogene Erscheinung ist; es kommt in der geometrischen Darstellung darauf' hinaus, daß man als L\"ange des Stabes verschieden gerichtete Schnitte durch den Weltstreifen des Stabes ausw\"ahlt. Andrerseits zeigt die geometrische Darstellung der Fig. 32 (S. 215) deutlieh, daß die LorentzVerk\"{u}rzung mit dem Unterschied der Streifenbreiten einen Unterschied des realen Verhaltens betrifft. Auch erkennt man hier, wie es \"{u}berhaupt m?glich ist, die St\"abe $l$ und $L$ zu vergleichen, obgleich nur der eine von ihnen realisiert ist: die Strecke OS ist f\"{u}r beide Theorien dieselbe; die alte Theorie behauptet, daß die rechte, zu $OQ'$ parallele Begrenzung des Streifens durch S gezogen werden muß, w\"ahrend die neue Theorie behauptet, daß diese Begrenzung als Tangente an die durch $S$ gehende Hyperbel gezogen werden muß}[232][200]
%
%Thus it is correct to claim that the Einstein contraction does not require any physical explanation; it is a metrogenic difference between proper and coordinate length. However, Lorentz's contraction \emph{does} cry out for such an explanation. The negative result of the Michelson experiment implies that rods of all materials invariably behave in agreement with distances measured by light rays. How can such a coincidence be explained? As we have seen, for Reichenbach, Weyl's expression \scare{adjustment} aptly expresses the need for an explanation, but it provides no details as to what it would look like. \qt{The answer can of course be given only by a detailed theory of matter, of which we have not the least idea}{Die Antwort kann nat\"{u}rlich nur eine ausgef\"{u}hrte Theorie der Materie geben, von der wir noch nicht die leiseste Vorstellung besitzen} \crz{233}{201\tm}. It is important to emphasize that this is not a marginal aspect of Reichenbach's philosophy. According to Reichenbach the very same problem emerges in \gr when the non-Euclidean nature of the continuum is taken into account. In this case, too, he resorts to the expression \scare{adjustment} and he refers his readers to the very same \S31 of the book.  Here the adjustment means is grapjicall that rod follows the contour of the ellipsi, instand of that of the parlall lines. On this fact both EInstein and Lorentz theory agree.
%
%\subsection{General Relativity}
%
%$x_i=f_i\left(x_1^{\prime} \ldots x_4^{\prime}\right) \quad i=1 \ldots 4$




%The advance of this logical structure, Einstein only because, he redefined the length of simultaneity that it became advantageous to combine space-time into a single geometrical structure, that is a four-dimensional manifold, that \Mink called the \scare{World} or \st (\S24). This simply means that it takes four numbers, the so-called \emph{coordinates}, $x_\nu$ (where $\nu=1,2,3,4$) to identify a \wpo, namely three numbers $x_1,x_2,x_3$ for its spatial location and one for time $x_4$. The set of points whose coordinates are defined by $x_\nu(s)$ where $s$ is an arbitrary parameter is called a \wl{}.

%%Such \s{Minkowski diagrams} can be extremely helpful and illuminating in certain types of relativistic problems. For example, one can often use them to get a rough preliminary idea of the answer. But one should beware of trying to use them for everything, for their utility is limited. Analytic or algebraic arguments are generally much more powerful.


%%Neither a dynamical expert, the geometrical explanation does explanation. The theory explains by requiremetn that \textit{all} the there, by claim, that must be Lorentz invariant. The geometrical transalti, is easy to fidn wjehte this reuqirement has been fullfiled. 
%
%After all a a not difefrent poit. In genreal that all cab be expressed in the so that evall ,, afreea all is mianca couple. This principle does ... hoeve,r all theory of amter would do, at ... the detailed are irerelav Is ehursitic tooo to find ... that are to be rejceted.

%\todo{Princeton}

%On the contrary, changed the kinematics and requires that all possible laws of nature must be Lorentz invariant.  That Lorentz invariance independent of Maxwell equations, deeper than Maxwell equations, independent of any theory of matter and radiation. Indeed, however, \textit{any} Lorentz invariant theory of matter and radiation would do. Ineed, at end we must peak one,  this would have some solution, that corresponds to \rac at rest in an inertial, that will be in equilibrium also in the moving frame.


%Sincee, the The advance are Lorentz invariant, that all laws of anture are laws governing material structuure of \rac should be Lorentz invariant. That some, e.g. a multime of would do the, descreed solutons, e.g. corresponding toe lemetnarly paricle. Will have some solution correspondn with \rac, the circle wull be closd only the theooy oas a whole. Howver, that the detail are fundamentally irrelvant. Any Lorentz-invariant theory of matter would do of only that have solution dynamical laws of nature, some of this laws that we can use as \rac, if a rod is rigid in one, it would be rigid on all cordiantes syst, Was provisioanl compromize, in the  If they are stable then we are xact to find again of the theory. rac ofr an explation \rac serve for \s{testing} the new kinmatics. It is only the requiremnt that laws of physiocs are Lorentz ivnariant. The use of \rac was because more secure than Maxwewll equatin. 


%That is also the real is a requiremt of the formualtion of laws of nature, so that simply putting delived the graivtiaontl field. Newith that as theory of princoples. Ideed, Lorentz theory was however, legitaibel approach, howeve, that the theory on a prticular theory of matter and radiation, that Einstein thought was probably wrong. The advanvet was inded, that this was not the case, the Lorentz were devloepd indepedn from Maxwell eqations (probably), and the entier theory requirement, a rquire that indeed whell newyond the electrodyanics,

%\todo{The coordinate difference of rod in motion i}

%The smae reasonging of \ger. Reichenac to thta Helmhotz and Poincaré problem of coordination between geoemtry and physics: brought this language into the special and general conventionalist camp, so to speak, with two $G+P$ that Reichenbach called \s{coordination}. Reichenbach considered the universal force to be a fact, asserting that Riemannian geometry is conventional; still, the elimination of universal forces declared it to be rigid. Hwver, this is not explanation. that while adpted that they alays onform to beav lighr rays, on the smae eomtry. Theory of matter? That \rac are not mdoifed by the gravitational field; That eh adatmpat Riennain geometry; that is princple of minialca cpupleing. That laws foverojng that do not entain \gmn. ANy otehr detail is irrelevant to the equation. That indeed, the simpoly in general coaiant direclty direcluer garivationaal laws. 


%In this sense Galilei and Lorentz transformations are neither true nor false, since both agree on the light axioms and are different because of definition of simultaneity.





%\hide{Natural measuring instruments adhere to coordinative definitions distinct from those assumed in classical theory. This assertion is, of course, empirical. The validity of the physical theory of relativity depends solely on this empirical claim. However, the philosophical theory of relativity—namely, the recognition of the definitional nature of the metric in all its details—stands independently of empirical evidence. While it emerged in the context of physical experiments, it represents a philosophical conclusion that is not subject to critique by individual scientific disciplines.}


%\hide{cop{If physics leaves room for a certain arbitrariness in  the determination of times, it must be that every determination  is justified in the same way and pertnits the construction of a  system of natura] laws. From the viewpoint of epistemology, this is  the meaning of th.e statement: \s{there is no absolute time}}.  } 



%The epistemological relativity of simultaneity refers only to the fact that numerous distinct light-geometries (e.g., Galilean vs. Lorentzian) are consistent with the light axioms. The semaratoo betwn and the matter axioms, and tha 


%An entry in Rudolf Carnap diaries, that he met Reichenbach (near Berlin) on \datef{2}{9}{1926}: \qt{He explained me the difference between Lorentz and Einstein contractions}{Er erkl\"art mir den Unterschied zwischen Lorentz- und Einstein-Verk\"{u}rzung} (RC 025-72-05). Carnap had just joined the Schlick circle in Vienna. The philosophical disagreement between the Schlick-Circle and Reichenbach was deepening and went beyond the specific issue of the philosophical interpretation of \sr. Reichenbach was convinced that the philosophy of nature should tackle metaphysical questions, such as that of the reality of the external world or of the human freedom \citep{Reichenbach1926b}. By contrast, Schlick, influenced by the young Carnap and Ludwig Wittgenstein deemed all such metaphysical questions as non-sensical \citep{Schlick1926}.

%Yet it should not be believed that that the relativistic concept of time would be false. As I have shown in the former report, this time can be defined purely in terms of the light geometry without using natural clocks and rigid measuring rods. Since definitions are arbitrary, the two times are equally justified. Independent of all empirical verification it can therefore be said: if there were an absolute time, it would no longer be absolute.


%\section{Dynamical vs. Geometrical Explanation int the \PRZL}

%In the meantime, Reichenbach thanks Max Planck's and and Max von Laue's support, had obtained the chair for natural philosophy in Berlin (\letter{Reichenbach}{Schlick}{2}{7}{1926}; \letter{Schlick}{Reichenbach}{5}{7}{1926}). Reichenbach clearly continued to consider the difference of this two types of contractions important. 




% \q{Einstein's assertion can be expressed by saying that material things adjust, not to the classical, but to the relativistic light-geometry}, which in truend on the convetion. 

%In is constrained by the requirement of \s{univocality} and \textit{simplicity} that depends upon the matter of fact validity of the axioms \citep{Reichenbach1921d}. 



%Any event at $A$ occurring at a time $t_1+\varepsilon\left(t_2-t_1\right)$, where $0<\varepsilon<1$, could be simultaneous with $B$. Einstein choice of $\epsilon=\frac{1}{2}$ implise that  Reichenbach famous emphasize, the freedom to choose which events are simultaneous in a given inertial frame depending on the value of the parameter $\epsilon$\todo{improva}.

%also neben der Gleichzeitigkeit auch die Gleichtormigkeit und also neben Streckengleichheit 1). Und zwar gelingt dies ohne Benutzung


%\cop{We see that the locus of the philosophical difference between Lorentz and Einstein has been misplaced by the proponents of the \textit{ad hoc} charge against Lorentz}. %The same could be made by clocks, indeed, he clock is with respect that the classical clock. Is the same ideal clock meaure that coordinate time of the moving clock in the  

%In this way, Einstein's theory could mantain in spherical ways, whereas Lorentz that are not sherical we do not notice it. nad that, and simoluc declare the relativis as rigid. Lorentz provided bad explatioan; howeever, Einstein did provide any kind of explanation. 

%In order, to assure that identical rods  similarly for the periods of material clocks, theuy seem to read $c$. That is that progrpagaion f light is spehrical in all inertial frames. That is are equal in spite of the fact, that their coordinate length is shorted by a idelcal rods read $c$. 

%The differencs the type of explanation. \cop{The deviation from the classically expected behavior exhibited by rods and clocks must have a cause in the sense of being due to a perturbational influence}. It is easily shown that such a theory leads to exactly the same observable predictions for the kinematics and geometrical optics in any inertial frame as does Einstein's special theory of relativity. Both theories have the same 



%\cop{We see that the locus of the philosophical difference between Lorentz and Einstein has been misplaced by the proponents of the ad hoc charge against Lorentz. To understand Einstein’s philosophical innovation, we must take cognizance of the fact that the Lorentz-Fitzgerald contraction hypothesis was not the only addition to the ether theory made by Lorentz in order to account for the available body of experimental data. Since the horizontal arm of the Michelson-Morley experiment is shorter than a rod lying alongside it but conforming to the expectations of classical light geometry, we must infer that when a unit rod in the ether system is transported to a moving system, it can no longer be a true unit rod but becomes shorter than unity in the moving system, and similarly for clocks. The deviation from the classically expected behavior exhibited by rods and clocks must have a cause in the sense of being due to a perturbational influence.}

%Einstein realized that the length of this rod in the moving system can be legitimately decreed by definition, and similarly for the periods of material clocks, then it would have been clear to him that the ground is cut from under his distinction between \s{true} and \s{local} (i.e., spurious) lengths and times and thereby from his idea that the horizontal arms. the \s{natural} behavior of things is indeed different from what the theory in question has been supposing it to be and that (b) deviations from the assumedly natural behavior transpire without perturbational causes. Thus Einstein declared as rigid the conform to the relativistic light geometry. To this purpose it need, the \s{Einstein contraction}, that is that definition of simultaneity that that the \s{moving length} in the moving system is shorter than the \s{rest length}. The effect disappears, if the comoving observer. However, the effect is real, in the sense, that at the same time, the measuring one would indeed, shorted. That the natural behavior rods.









%  \cop{In particular, Weyl's class of inertial frames cannot be singled out in this fashion, but only a wider class of frames related by conformal transformations, preserving angles but not necessarily distances.} \citep{Rynasiewicz2005}. \cop{Therefore, when operating with the entire infinite world, Minkowskian geometry can be completely built upon the single concept of the null element; the straight line becomes dispensable},  adds a single “point at infinity,” 

%One whoudl add light-cones onto light cones (conformal structure), also the straint lines (projective struute). Alternatively, one could also use \rac to assure that have the same units are transfoproatable in idfferent inerila frame.


%If one renounces the demand for the Kleinian perspective, and thus understands by mapping onto the numerical space a bijective continuous mapping of the entire infinite world onto the complete numerical space, then the first condition alone—that every straight line must be transformed into a straight line—defines the world not only as a projective but also as an affine space. For the only projective mappings that transform finitely distant points into finitely distant points and infinitely distant points into infinitely distant points are the affine mappings. Conversely, the only spherical transformations that leave all points in the finite domain within the finite domain are the similarity transformations. Therefore, when operating with the entire infinite world, Minkowskian geometry can be completely built upon the single concept of the null element; the straight line becomes dispensable. This was carried out several years ago by the English mathematician Robb in a very cumbersome and artificial system of axioms. I refer to his book "Theory of Time and Space," published by Cambridge University Press, and the shorter presentation, also published there, "The Absolute Relations of Time and Space."

%Wenn man also nicht von einer vorgefassten Meinung ¨ uber die Welt als Ganzes ausgehen will, bedarf man weiterer Naturgesetze, um die gleichf¨ ormigen Translationen aus der Klasse aller einheitlichen Hyperbel- bewegungen herauszuheben. Als solches eignet sich, wie wir sahen, das Galileische Tr¨ agheits - gesetz. Aber man kann sich auch, wenn man das vorzieht, auf das Verhalten der starren Masst¨ abe oder Uhren st¨ utzen


%Diejenigen Abbildungen, welche Nullkegel in Nullkegel ¨ uberf¨ uhren, sind die M¨ obius’schen Kugelverwandtschaften. Beim Operieren im beschr¨ ankten Ge- biet (was doch allein physikalisch vern¨ unftig ist) gen¨ ugen die Nullkegel also nicht zur Charakterisierung der Geometrie, wohl aber Nullkegel und Gerade.




%%\cop{Using only the behavior of light rays, he defines separate
%temporal and spatial metrics for individual frames, which he calls the “light-geometry” [Lichtgeometrie], and then derives the Lorentz transformations as the isometry boosts for the light-geometry. (A defect of this procedure, and one certainly not lost on Weyl, is that the class of inertial frames cannot be
%singled out in this fashion, but only a wider class of frames related by conformal transformations. Reichenbach had sensed the problem, but probably at a stage too late for him to make major revisions.}. So far no mention is made of rigid rods and material clocks. \cop{These notions are introduced only after the development of the light-geometry, at which point
%Reichenbach sets out a series of “matter axioms” which collectively assert that these material structures behave in accordance with the light-geometry and thus the Lorentz transformations.} \cop{The upshot of the entire work is that, apart from the prohibition on super-luminal causal propagation, it is only these latter assertions that separate relativity theory from classical space-time physics. The physical axioms governing the behavior of light do not depend on a principle of relative motion and are consistent with both the (relativistic) light-geometry and the geometry [or rather, kinematics] of classical optics.}

%that wheare the are arbitary, the combination of axioms and definition is empirically testabls. \cop{A footnote directs us to a third option, the use of test masses to determine inertial motion, as in the appendix to (Weyl, 1921a)}. Reichenbach will later introduce the term of art, \s{constructive} axiomatizion, for this manner of proceeding (Reichenbach 1924). Its virtue (emphasized in Reichenbach’s subsequent writings, but not in the “Bericht”) is that it permits the separation of the factual content from the conventional components of the theory.


%Concerning the production of the work, the
%preface, dated March 1924, mentions delays in publishing due to the difficult eco-
%nomic situation. It also tells us that the investigation was begun in the fall of 1920
%and completed “in essesence” in March 1923. It is not inconsistent with this that these
%passages represent emendations at the stage of correcting the galley proofs.

%\cop{Wenn Poincaré nicht ausdrücklich betont hat, daß Konventionen nicht voneinander unabhängig, sondern immer nur gruppenweise möglich sind, so würde man ihm natürlich doch sehr unrecht tun, wollte man glauben, er sei sich dieses Umstandes nicht bewußt gewesen. Selbstverständlich war dies der Fall, und den Unfug, den z.B. Dingler}, against which have written at about that time \citep{Reichenbach1921a}. 

%Schlickl was suoprise, that Poincaré and belived that Poincaré stance could be easily extend to the case of Riemannian geometry. It is only a question of \s{simplicity} that the choice among differnt goemtry. Weyl and Hilebrt were against, this stance, but Schlick could Einstein's opiniion this was indeed, the case. \q{Ich zweifle nicht, daß Sie sich in dieser Frage auf unsere Seite stellen.}.



%‘Die Einsteinsche Raumlehre’, Die Umschau, 24: 402–405. Engl. transl. in Reichenbach (2006), 21–29.

%Rezension von Moritz Schlick, Allgemeine Erkenntnislehre’, Zeitschrift für angewandte Psychologie, 16: 341–343. Engl. transl. in Reichenbach (2006), 15 19.

%In the previously mentioned report, the axioms are divided into light axioms, those that assert only the properties of light, and matter axioms, which assert the properties of rigid measuring rods and natural clocks. It is clear that axiom B can only enter into a contradiction with the matter axioms, since it asserts nothing about light. We will omit two of the matter axioms, specifically, VIll and X, and begin by showing that axiom B is compatible with the remaining axioms.

%Through his attendance at Einstein's lectures in 1917/1918 and 1918/1919, he had acquired a detailed technical knowledge of the new theory, surpassing that of most philosophers of his time. The success of the eclipse expedition was announced in November 1919 \citep{Dyson1920}, Einstein was turned into an international celebrity. debate As a trained physicist with a doctorate in philosophy \citep{Reichenbach1916}, Reichenbach was uniquely positioned to engage with the philosophical implications of the theory. In February or March 1920, shortly after his move to Stuttgart, Reichenbach decided to make this topic the subject of his habilitation. According to his later recollections\footnote{These autobiographical notes \cite[044-06-23]{HR} were written in 1927}, in the preceding months, he had further worked on the theory. The Kapp-Pusch coup in \datemy{13}{3}{1920} rovided Reichenbach with a few days off from his job at the Huth radio industry  \citep[044-06-23]{HR}. This gave him the opportunity to work uninterrupted, and in just ten days, he completed an early draft of his habilitation. The manuscript was then typed and shown to Einstein and others. Thanks to the intervention of Arnold Berliner, the influential editor of \jt{Naturwisseschaften}, Reichenbach secured a publishing agreement with Springer \citep[044-06-23]{HR}.

%For the young Reichenbach, one of the main philosophical merits of the theory of relativity was the revelation of the physical character of geometry
%Reichenbach borrowed from \citet{Schlick1918} the idea that physical knowledge is, ultimately (\emph{Zuordnung}), the process of relating an axiomatically defined mathematical structure to concrete empirical reality \citep{Padovani2009}. However, Reichenbach attempted to give this insight a \scare{Kantian} twist. According to Reichenbach, in a physical theory, besides the \scare{axioms of connections} (\german{Verknüpfungsaxiome}) encoding the mathematical structure of a theory, one needs a special class of physical principles, the \scare{axioms of coordination} (\german{Zuordnungsaxiome}), to ensure the univocal coordination of that structure to reality. For the young Reichenbach, the latter axioms are \apr because they are \scare{constitutive} of the object of a physical theory. However, they are not apodeictic or valid for all time. As is well known, Reichenbach would soon abandon the project of a constitutive but relativized \apr. However, he would firmly maintain the separation between the mathematical framework of a theory (the \scare{defined side}) and the way it relates to empirical reality (the \scare{undefined side}) as an essential feature of his philosophy \rhp{40}{42} as an essential feature of his philosophy


%\footnote{\letterhr{Schlick}{Reichenbach}{26}{11}{1920}[015-63-22]; \letterhr{Schlick}{Reichenbach}{11}{12}{1920}[015-63-19]; \lettersa{Reichenbach}{Schlick}{29}{11}{1920}; \lettersa{Reichenbach}{Schlick}{10}{9}{1920}}

%What should be explained is not the Einstein contriction. this a purly perspectiva phenemna, that judge at the smae time. Is the Lorentz contraction, that is dynamical that rac behhave differently then from classical kinematics. 
%
%Reichenbach is a convetionamis ... that is ... that cohoce, that ... agree wiht the rather that ...
%
%coordination of the ... the of explpantion that is hyel in particular \rac livght raugs, agreed on the same geometry. The second apsec thas eebe as far as I cann see conpllt. Negeltec. Indeed. adjustitmnat. ... hat goemetricla expalan in snot expalantioan. However, that reaciton to Miller's experi, why dos not work. The theory is only necessary it ... explantiona by constraitn. by
%
%as a conventionalist. However, this characterisiat poorly the progfres. that Actually, was of mgeoemtrical empiriciscist, since that rejection of metrical forces; Reichnebach was cmmmeter explanation
%

%That a modell in which explanation, (2) that not Einstien contracit, but Lorentz contaction (3) t


%Why the \rac geometry 



%The story is well after the discoery of non-Eculdiean, th equestio which was the true geometry of the work. Helmholtz that  empriics, ... conveitonalis, that decision. In recenet years the question rods has become. Not why the geometry. The rods behave in a Ueclidean way because the geomtry is Eculdiean, that geometry is Eculdiean because all physocal system behave. Of special relativity. Indeed, in this contedst since The what should be explaiend. Indeed, molecular; tis that support ... on the opposite, sicne.
%
%To this purpuse between and adhust. Weyl, had used with a very wht to behave in a Riemannian way? Indeed, it cannot be a coindece 
%
%Reichenbach with specific fact. TO exned to very Euclodean and Euclida geometry. The present paper has the goal of Reichennach, hoev Reichenach ad quite interesting. The explanation is the deviation from background geometry. Whereas the is the convergecen on the same non-trivial geoemtry. Whey to all rods and clocks, light rays egc. converve on the same geometry. Is not an exlanai. 
%
%Introuced in special relativity, 1924 paper, it made systemmatic use of the distiacntion. One year later started to writne in general relativity. Taht Reocemhacn. Indeed, the image of REichenach is suprosingly ... REichenahch was a gometicla empricisst the confirmation. But this not all Reichenacbh was that ultimat explanatoo would have beeb. The argument geoemtrizaiton of physical field, but physicalisation of gehetry.

%The aspect of Reichenbach's interpretation of \sr that has attracted the most attention is probably his famous conventionality of simultaneity---the freedom to choose which events are simultaneous in a given inertial frame depending on the value of the parameter $\epsilon$. In a classical paper, David \citet{Malament1977} has famously shown that one cannot allow such a freedom if one believes---as Reichenbach did---in the causal theory of time. Malament's paper, the epitome of an influential work, has generated an enormous amount of discussion \citep[cf.][sec. 4 for an overview]{Janis2014}. 
%
%More recently, however, Robert \citet{Rynasiewicz2003} has, as far I can see, rightly challenged this view, showing that the non-standard definitions of simultaneity, though possible, play a \q{disappointingly little \textins{role}}[][125][Rynasiewicz2003] in Reichenbach's formal construction \citep[see also][]{Rynasiewicz2012}. The present paper intends to show that, whatever stance one might take about this matter, there is another aspect of Reichenbach's interpretation of \sr that has gone unnoticed in recent literature, which is possibly more characteristic of his approach than the $\epsilon$-definition of simultaneity: Reichenbach's distinction between two types of rod contractions occurring in \sr, the Lorentz and Einstein contractions.
%
%Reichenbach introduced this distinction introduced this distinction in his discussion of Miller's experiment. In 1925 the American experimentalist Dayton C.~\citet{Miller1925} published the results of a series of Michelson-type experiments that he conducted at the top of Mount Wilson, in Southern California. Miller's \scare{scandalous} detection of an ether-drift sparked, as one would expect, considerable debate that well beyond the physics community about a possible refutation of \sr. The role of Miller experiment in the history of this experiment has been Einstein's attitude towards this experiment has been carefully investigated by Klaus \citet{Hentschel1992a}. The debate that ensued in the physics community, in particular in the American one has been nicely presented by \citep{Lalli2012-04}. However, the philosophical reception of the experiment among \scare{philosophers} is still to be written. This paper will like to close this gap in the literature.
%
%Reichenbach was the first to join the debate by publishing a paper the same year \citep{Reichenbach1925}, intending to demonstrate the physical implications of his axiomatization of \sr. The peculiarity of Reichenbach's approach can be inferred from remarks that Moritz Schlick sent to Einstein at the end of 1925 (\letter{Schlick}{Einstein}{27}{12}{1925}[21-591][EA]; cf. \cite[361f.]{Hentschel1990}).  To his surprise, Schlick found conclusions in Reichenbach's paper that could not have been further from his own. Reichenbach did not believe that the Lorentz contraction was an \latin{ad hoc} hypothesis and that Einstein had dispelled \citep[see][on this issue]{Janssen2002}; he believed the contraction needed a dynamical explanation based on a theory of matter in \sr, just like in Lorentz's theory. Thus Schlick realized that for Reichenbach there were no differences between Lorentz and Einstein's theories, and therefore no conventional choice between them. Moreover, the acceptance of Miller's results would not have implied a return to the old ether theory. By reading Schlick's letter, one is immediately disabused of the conviction that Reichenbach's axiomatization was a mathematically sophisticated version of Schlick's conventionalist reading of \sr.
%
%Reichenbach's line of argument was based on the above mentioned distinction between \scare{Lorentz contraction} and \scare{Einstein contraction}, which would resurface in his major 1928 monograph \citep{Reichenbach1928}. In a slightly updated parlance, one can say that the Lorentz contraction compares the proper length of a moving rod in \sr with the length that the rod would have had in the classical theory; the Einstein contraction compares the proper length of a relativistic rod with its coordinate length in a moving frame. The first contraction implies a real difference and requires a molecular-atomistic explanation in both Einstein and Lorentz's theories. The second contraction is a perspectival difference that depends on the definition of simultaneity. Reichenbach holds the controversial opinion that only the first contraction is necessary to explain the Michelson experiment. 
%
%The aim of this paper to show how it was Reichenbach's reaction to Miller's experiment that prompted him to introduce this distinction. \Cref{ax} of this paper presents a sketch of Reichenbach's axiomatization of \sr, which is indispensable to understanding his line of argument. \Cref{lcec} introduces the distinction between Lorentz and Einstein contractions. \Cref{me} shows how Reichenbach uses the distinction to provide a philosophical account of Miller's experiment. \Cref{mc} follows Reichenbach's \scare{geometrization} of the opposition between Lorentz/Einstein-contraction in terms of Minkowski diagrams, as presented in his 1928 book. The paper concluded by showing, that Reichenbach's distinction was resurrected by Adolf Gr\"unbaum in the 1950s, that use it against Karl Popper's claim that the Lorentz contraction was not falsifiable. 
%
%
%Nevertheless, it was Reichenbach who must be credited for having attempted to show that the Lorentz-Einstein relationship should not be understood along the lines of the ad hoc/non ad-hoc distinction. According Reichenbach both theories attempt to account for an odd coincidence, that matter and fields contract in the same way (that is in Reichenbach's parlance the light geometry and matter geometry agree). Lorentz explained the difference as a deviation from a standard behavior for rods and clocks. Einstein, by contrast refused to give an explanation and simply declared that the relativistic behavior to be the natural behavior of rods and clocks. According to Reichenbach, however, an explanation is needed in Einstein theory as well. Surprisingly, in Reichenbach's view the explanation is not to be searched in the geometrical structure of space-time \citep{Janssen2009}, but in a theory of the material structure of rods and clocks. In this sense, Reichenbach approach appears as a curious anticipation of a neo-Lorentzian approach to \sr \citet{Brown2005}.
%
%%Nevertheless, it was Reichenbach who must be credited for having attempted to to show that the the Lorentz-Einstein relationship should bot be understood along the lines of the ad hoc/non ad-hoc distinction. According Reichenbach both theory attempt to account for and odd coincidence, that matter and fields contract in the same way (that is in Reichenbach's parlance the light geometry and matter geometry agree). Lorentz \emph{explained} the difference as a deviation from a standard behavior for rods and clocks. Einstein, by contrast refused to give an \emph{explanation} and simply declared that the relativistic behavior to be the natural behavior of rods and clocks. According to Reichenbach, however, an explanation is needed. Surprisingly, Reichenbach this explanation is to searched in the geometrical structure of space-time, but in a theory of the material structure of rods and clocks. In this sense, Reichenbach approach appears as a a curious anticipation of a neo-Lorentzian approach to \sr \citet{Brown2005}**.
%
%%%Suprisingly this have to be sear
%%%
%%%What have to be explained is however 
%%%
%%Reichenbach made a unacepted, in the geometrical structure of space-time \citep{Janssen2002}{Janssen2002a}, but in the future theory of the material structure of rods and clocks. In this sense the so-called neo-Lorentzian approach to \sr that has come to the fore in the last decade, thanks mainly to the work of Harvey \citet{Brown2005}. 
%
%
% but a non-trivial one or rods and clocks. What is surprising respect to the current debate. In their common origin, that forces of all types must transform alike because they inhabit the same space and time \citep{Janssen2002}{Janssen2002a}. 

%After the correspondence with Schlick and his consequent “conventionalist shift”, around the end of 1920, Reichenbach no longer talks about constitutive principles. The first development of this discussion is apparent in a report that he deliv- ered at the Deutsche Physikertag in Jena, in September 1921, on his plan to ax- iomatise relativity theory.45 The fundamental methodological innovation of this short paper is to present an axiomatisation in which the starting point consists of making use of axioms that can ideally make direct contact with empirical (or ex- perimentally testable) facts, and of complementing them by introducing a number of coordinative definitions for the construction of the conceptual content of the theory.


%Reichenbach this nothing more than a trick delcalre that \Mink space time of coinceidnece between mater and light geometry, is only a vizializaiton of the agreement between, simply deaclret as reigi,d that two geometruy agre, \Mink of this agreemt.


%Schlick was indeed, better convetion; howver, Reichenbach a dynamical explantion, of a differnet deflection from deflection from a classical geometry, which is true ;;; that that adjustimeant to the same egeomru; This a palce-hodele, a theory of matter. Reichenbach is indeed, fora. That the laws of field are Loretnz ivanriatn, in particualr that light; the by the laws of field fovering matter. Thus, that are Lorentz invariant. However, that the delai the only releavan tis requiremtn that all laws nature. Explanation by constarint. 


%Howver, what his coincicen amouts two? That lwas field and laws foveriong matter are Lorentz ivnaraint happens to be Lorentz ivnariant. One that \Mink does not add aytjing to this since it only codofied more ... that must be Lprentz ionvairant. THis is reeason why woud cosndier as house of cards. Is a theorericoa rigidi that save by a differnet choice of a conveiton. is an explation by constraint Reichennach hat is modeled to \th, this was heurist, and the core of the theory. 

%%The type of explanation Reichenbach refers to is geometrical rather than dynamical. The Lorentz contraction behaves according to dynamical principles, while the Einstein contraction follows the geometrical framework of light geometry.  The geometrical explanation is not violated by experimental results, as it ensures that optical and tactile measurements agree—a coincidence that Bridgman refers to as fundamental. Reichenbach explains that this agreement aligns with Minkowski space-time, providing a unified explanation of matter consistent with the theory of relativity. However, what Lorentz's explanation misses, according to Reichenbach, is the requirement for invariance under the principles of relativity. Lorentz's approach introduces constraints that lead to contradictions with the first postulate of relativity, which is grounded in experimental results. Without Lorentz invariance, the theoretical framework would collapse like a deck of cards. 
%
%%\q{It cannot be an accident that two measuring rods which have the same length at one p'lace always have the same length when brought to a different place along different paths. It must be explained as an adjustment to the field in which the measuring rods are embedded like test-bodies. just as the compass needle adjusts its direction to the magnetic field of its immediate environment, so measuring rods and clocks adjust their unit length to the metrical field. All metrical relations between physical structures must be explained in this fashion, including the Michelson experiment, according to which rigid rods adjust themselves in a definite way to the motion of light. The answer can of course
%%be given only by a detailed theory of matter, which has not yet been elaborated. It must be explained why the accumulations of density in certain parts of the field, electrons and similar particles, provide a simple indication for the metric of the surrounding field. The word \s{adjustment} therefore poses a problem rather than supplies solution. The existing situation is formulated rigorously by the
%%matter-axioms without the use of the word “adjustment.” If this
%%theory of matter were exactly formulated, we would be able to explain
%%the metrical behavior of physical structures. For the time being,
%%however, we can speak of an explanation by Einstein's theory as little
%%as we can speak of an explanation by Lorentz’s theory or by the
%%classical theory.}
%
%Decal, to be of the smae lenght. THis indeed, where that Einstien contraciton comees inti play, that latter that indeed, the same of linlitaity Indeed, Reichenbach critiques both approaches, emphasizing that any explanation must remain consistent with the Lorentz invariance inherent to relativity theory.
%
%That the light geometry correspond to the matter geometry is indeed a coincidece, if we proceed from a sort of operational point of view. This Bridgeman points. The sursping fact that agree, and this agreemedn an explanation. One might that ezplanation the is geometrical. However, that is probably that that this is sompu ropeating te coicnicne ... why indeed light geoemtry behave correspond. Thus, we can in formal terms by a  \s{graphucal represtnation} ... geometrucal explaiantion since, geoemtrical interpreatin. However, that the laws of field are Loretnz invariant just as laws governing matter. Howeve, that presentation in a different language, that alla agree.
%
%
%%%Is a convetion, however, this is not hte ase. That the is pars destruten. Are not onvus, Ohwer, wonce the new are set buto they are either ture or flase. Indeed, that .... notice, that indee,d is compatible Ritz theory. What is the real diferen is tranverse toppbe efect. 
%
%However, that the laws of field are Lorentz invariant just as laws governing matter. Lorentz a particular theory of matter, electromagneic this is bi; however, the noveltiy that he did not trust Mawell eqaiton \th. That all laws of anture, must Lorentz invariant. A detailed of matter is not necessary; indeed, has not particl to do with \rac. That to test the new kinematic ... howcer, howver, that was not convetonal. Lorentz transflaron are true or false.  Can be tested by uisng \rac. Of course Einstien was at least proviisaon. Ar complicated, material system, that somw laws that theese, are innde, Loretnz invariant, and vbvusi cirucaltiy. Howefer, that provisiall. The problem that thest of the toery .. that delived ... laws that ...  Neither a dynamical is necesary, since any Lorentsz ivnariant would do; the geometruical expalatio ... that of this areement. Is a explation by constraint, which is indeed similar \th.  Thus, a concre, that requirement that the theory, is Loretnz invariant. This indeed, the peculiarity of specual. The rethoric that some ... however, it is true or flace. avoiding 
%
%What that matter explaatin. That we set in general covariant form and we expet that will also in ghe gravitaitonal field Waht is that the Riemann-Christogell tensor does not enter of the field equations. That that new requirement, indeed, special relativity, by and assume ... here, that .... indeed, hwoevr, also in this case. The deails are irreletnabv. that the laws, to depend only on the first defrivative of the \gmn
%
%1. dynmcial expation; that agree, that behave the laws fovering the fidsl agreed with laws foernign matter, that they are boti Lorentz covanraitn, 
%
%2. geoemtrical epation: not an exlati; it is graphical condof this agreement; declare that \rac are equal lirhg determin, 
%
%Howver, that all are Lorentz invariant; indeed expalation by constraint; that must be Lorentz invariant.
%
%It represents a general assumption about all
%natural phenomena. This equivalence is supposed to hold
%not only for the mechanical, but also for the electrical,
%optical and other phenomena; in all these cases, no differ
%ence is supposed to result, whether one speaks of an ac
%celerated motion of the box or of a gravitational field.



%general realtivity ... thta is principle of minibal copling, that does not appear. This howeva, a cosntriant that we impose of the laws nature, rather than a speciifc law.



%That coincixe, indeeed, that the beahviro of matter is a mre coincice, .. that thermo... is a reguirematn, that lla laws. That alre Lorentz ivnariant, thus, nay theory of matter or radiation would woht the jobe as long .. Howeve, what ... happens to be, must be.  Thus, is not necesary, is can cut true the deati, once are Lorentz invariant, nothhing eelse ie reuqre. What tries to accompli, Lorentz contrac, and not the Einstein contraction that elim a mere metrogentic pheone. adjusticemt is indeed, for .. defivation, but the coincirace a the same geometry. ... An explanation by constraint, the model of themroduankcs However, it is also that the spiri is not coince, but at must be Loretnz invariant. of faor. Whule the con wudl the tneire suste, The great that \s{theoretical rigidity}. That only rigid but fagile. The fact is that doe snot clapls, is the sing that it is true.


%The answer can of course be given only by a detailed theory of matter, which has not yet been elaborated.  the coincicen that matter conract exactly to with field. However, this inded, taht laws of matter are Lorentz ivnariant just like the alws of fields, that is Maxell. 




%\end{comment}
%\begin{comment}

%The part $B$ of the last chapter of \PRZL is dedicated to \gr. Reichenbach introduced the problem of gravitation from a physical point of as the passage from Newton's scalar theory of gravity based on a the single potential $\varphi$ to Einstein's theory of gravitation based on the multi-component object $\gmn$ (\S\S34--38). To clarify the nature of the latter mathematical object, in \S39 Reichenbach offered a didactic presentation of \q{the analytic treatment of the problem of space that was introduced by Riemann and Gauss} \rzlp{**}{242}. Reichenbach's starts from the two-dimensional case and then extends his results to four-dimensions. To slightly shorten our presentation, however, I will directly start to latter case. 
%
%As is well known, \rt has forced us to combine space and time into a four-dimensional manifold, that \Mink called the\scare{World} or \st (\S24). This simply means that it takes four numbers, the so-called \emph{coordinates}, $x_\nu$ (where $\nu=1,2,3,4$) to identify a \wpo, namely three numbers $x_1,x_2,x_3$ for its spatial location and one for time $x_4$. The set of points whose coordinates are defined by $x_\nu(s)$ where $s$ is an arbitrary parameter is called a \wl{}. At this stage, coordinates $x_\nu$ are nothing but identification numbers, that is, in Reichenbach's parlance, have only a \scare{topological} function, they determine the order of the \emph{between} relation. In order to define distances between points, one needs to introduce an expression that assigns a number $ds$ to the coordinate differences $dx_\nu$ between two close \wpo{}s, the so-called the \scare{fundamental metrical form}. In the case of \Mink \spti, it is always possible to choose the coordinate numbers \xn so that any distance satisfies the relation:


%According to Reichenbach, however, the choice \emph{among} Riemannian geometries (Euclidean and non-Euclidean), remains a matter of convention. Let's say that, by using \rac, one has found a certain set of \gmn in a certain set of polar coordinates \polart, say, $\gmn=g^S\mn$ as in \cref{eq:polarscharzschild}. \rite does not vanish, and, therefore, the geometry of \spti is non-Euclidean. However, to avoid this consequence, we could still claim that the rod along the radial $r$ is not rigid, but it is shorted by $\sqrt{g_{11}}$, whereas a clock is slowed down by the factor $\sqrt{g_{44}}$ by some force that affects our measuring instruments. Thus, given a certain non-Euclidean \gmn, we can always reduce them to normal values \gmnbar by introducing suitable correction factors $C\mn$:
%
%\begin{equation*}
%\gmn =\gmnbar + C_{\mu\nu}\,.
%\end{equation*}
%%
%Thus, one can always introduce the distinction between a \scare{measurable but distorted} non-Euclidean \gmn  where $\rite \neq 0$ and the \scare{true but hidden} \Mink \gmnbar, where $\rite=0$.  $C\mn$ is a tensor field that can be interpreted as a real force field that, contrary to usual forces, affect all bodies in the same manner, that is, it is a universal force. Since this force, in principle, undetectable, we can \emph{decide} to set universal forces as equal to zero by definition\footnote{However, the idea of taking $C\mn$ as a real force is somewhat problematic. The force will deform our rods and clocks in a different way in different coordinate systems since the coefficients $\gmn$ are different. As a consequence, the force field will make the $ds$ not invariant. This objection was raised by \citet{Weyl1922c}. See also \cite{Norton1994}}. The metric of \st becomes an empirical fact only after the postulate of the disappearance of universal forces is introduced. Once a definition of congruence is given, the choice of the geometry is no longer in our hands; rather, the geometry is now an \emph{empirical fact}. In this sense, Reichenbach's philosophy of geometry is sometimes called a neo-conventionalism \citep[see also][ch.\ 8]{Reichenbach1953a}.


%Our reflection shows us that space-measurement and time on simultaneity. that is taht simultaneisy poi. To determine the rate at which a moving clock ticks, you compare readings on that clock to the readings on stationary clocks synchronized in the observer’s frame. And now immediately Einstein points out the following "peculiar consequence", namely, that a clock moved uniformly between two stationary synchronized clocks A and B lags behind B when it arrives there. And then, by extension. This idea can be expressed mathematically by bringing together space and time into a four-dimensional structure, into a space-time manifold- ness.  The proper time interval measured by the moving clock. The corresponding time interval measured in the observer’s reference frame to cmpare They determine the time interval $\Delta t$ between these two events using their synchronized clocks, which rely on Einstein’s simultaneity convention, the sarting point at the same time ... The observer must synchronize their own clocks to establish when the moving clock passes the starting point of the measurement.

%The relativity of simultaneity has a peculiar conse quence, as far as the measurement of space is concerned.



% of the motion and accordingly set the clock properly. But the theory of relativity maintains much more; it maintains, namely, that any running mechanism, regardless of kind, would manifest a similar retardation.


%One is not entitled to infer with Lorentz that a unit rod in the aether system will no longer be unity; that  For the relational conception allows us to call that same rod unity in the moving system by definition, and similarly for the units of time on clocks at rest in a moving system. That to change the geometry of space-time ...   That the two arms have the same lenght; this while we measured the lneght of the arm at the same time. Then  \cop{If Lorentz had realized that the length of this rod in the moving system can be legitimately decreed by definition, and similarly for the periods of material clocks, then it would have been clear to him that the ground is cut from under his distinction between "true" ("real") and '10cal" (i.e., spurious or apparent) lengths and times and thereby from his idea that the horizontal arm in the Michelson-Morley experiment is actuaJly shorter than the vertical arm.}. That two are equal, that is that is non-spherical, but since one would not notice it; that is spherical that since the ... does not explaom experime, and not; the is that the geometry space-time is the relativistic light geoemtry, that to reclear that equal are equal are equal 1.  \cop{Einstein was able to see that the unexpected results of the Michelson-Morley experiment do not require any perturbational causes of the kind envisioned in the aether theory because they are integral to the \s{natural} behavior of things}. HOweve, this was just as bad as 

%However, this is not an explation, it is a encode that happens to be.

%Every physical theory tells us what particular behavior of physical entities or systems it regards as "natural" in the absence of the kinds of perturbational influences which it envisions. Concurrently, it specifies the influences or causes which it regards as responsible for any deviations from the assumedly "natural" behavior. But when such deviations are observed and a theory cannot designate the perturbations to which it proposes to attribute them, its assumptions concerning the character of the "natural" or unperturbed behavior become subject to doubt. For the reliability of our conceptions as to what pattern of occurrences is "natural" is no greater than the scope of the evidence on which they rest. And a theory's failure to designate the perturbing causes of the nonfulfllment of its expectations therefore demands the envisionment of the possibility that: first, the "natural" behavior of things is indeed different from what the theory in question has been supposing it to be and that, second, deviations from the assumedly natural behavior transpire without perturbational causes of the kind previously envisioned by the theory.

%(a) the \emph{Einstein contraction}, which results from the relativity of simultaneity and compares the length of the moving rod with the length of the rod at rest in the same Lorentz-Einstein theory; and (b) the \emph{Lorentz contraction}, which compares the length of the same rod lying in the direction of motion in different theories---classical mechanics and the Lorentz-Einstein theory. In the classical theory the coordinate length of the moving rod is expected to be just as the coordinate length of the rod at rest. In the Lorentz-Einstein theory, by contrast, the coordinate length of a moving rod is always shorter than its proper length which is the same in all inertial frames. Nevertheless, Reichenbach claims, the length of a rod in motion can still be said to be shorter than the length that a classical rod would have if both were measured at relative rest in the moving frame. 

%Lorentz-Fitzgerald contraction by finding that the \s{true} length of the moving arm, which he believes himself to be observing from his vantage, is smaller than the \s{spurious} length measured by the rod of a terrestrial observer. the moving length is shorter thatn  by  af acto \lf,  Einstein realized that the length of this rod in the moving system can be legitimately decreed by definition, as having the same length to realize that we are free to call the length of the moving rod in the moving coordinate system one if we wish, despite its having a moving lenght of only \lf. To indrocuce, that that convetion, as at the smae time; he changed the definition of lenght, between simultaneius points.  If one adomts, taht that length of rest-lenght of the mv



%\cop{The Lorentz contraction is a comparison between the actual length of a moving rod and the length expected on the classical theory, the Einstein contraction involves the measurement of the length of the moving rod from the rest system}. The declear to be equal that are measured. It would result in the same light-geometry as the standard Einstein criterion in the sense that it would yield the same temporal and spatial measures as the standard within each inertial frame. To this purpss, that legnth adn time elabse depend on the definition of simultanienity, 

%The prjeciton of the in moving in rest ystem, where at the same time as defined in threst system. That is to define has having the same lenght that are as defined by light geometry. That this is a convetion. To compensate so that the velocity of light is the same in very direction that the


%That proper lenght has been shorted by a factor 

%A unit rod when placed in motion has by convention the length / (1 - v2/c?) and the velocities of light in the moving system are those given by the classical aether theory, and the rod is contracted;

%A unit rod when placed in motion has by convention the length / (1 - v2/c?) and the velocities of light in the moving system are those given by the classical aether theory, and the rod is contracted;

%That to have the same lenght is the same, to establish the connection between lenght measurement and simulataneity. The coordinate length that this alloed 

%the value obtained by successively marking off the measuring rod along the segment is its length. However, this definition of length is applicable only if the measuring rod is at rest relative to the segment.

%We say that the measuring rod is to he regarded as having the same length whether at rest or in motion

%lu /mgth of a movin~ /int'·St'~tltt:llt is the disltmce bt'lu·,·m simultaneous positions of its em/points.


%EInstien this decision, then while we will say that the rod has a length of one whether at rest or in motion, nevertheless when it is in motion it is shorter than we would expect is to be on the classical theory.

%, it did so by redfining the notion of length, in particular, have the same lenght. By redfinit that is by rest-length adn  that light geometry are of the same length. It can do so, by redefuing the notion of lenght distance between the simultaneous positions of the endpoints. 

%Since the shape is determined by the simultaneous projection of all the points. it will evidently depend on the velocity of the object and on the definition of simultaneity. According to the Lorentz-Einstein assumptions, a moving circle assumes the shape of an ellipse, the minor axis of which lies in the direction of motion

%There is nothing mysterious in this relation, for it is based on the fact that we do not measure the moving rod, but its projection on a system at rest, is shorter that the rest-length. 

%The dependence of spatial measurements on the definition of simultaneity has a peculiar consequence for processes of propagation such as light and sound waves, that travel from one center in all directions.  This result is the basis of Einstein's principle of the constancy of the velocity of light; the motion of light can be considered as a spherical wave for any uniformly moving system The shape of the surface of the light wave is not uniquely determined but depends on the definition of simultaneity. Distances traveled by light in equal times would then in general not be equal, since according to the Galilcan transformation the velocity of light depends on the direction of the moving system.


\todo{Lorentz contraction: Is shorter that it would have been in the previos theory, and shorter the rest by cactor; Einstein the declears that length in motion are of equal length; in order to mantain and that the velocity of light is the same for both observer, is spherical connection between  length and simultanienity;}



%But the very concept of measuring a moving rod while not moving with it requires conceptual innovation.


%More explicitly, the Lorentz-Fitzgerald
%contraction hypothesis asserts a comparison of
%the actual length of the arm, as measured by
%the round-trip time of light, to the greater
%length that the travel-time of light would have
%revealed, if the classical ether theory were true.

%Thus, using light as the standard for effecting
%the comparison, this hypothesis aflirms that in
%the same system and under the same conditions
%of measurement, the metrical properties of the
%arm are different from the ones predicted by
%classical ether theory.

%Unlike the Lorentz-Fitzgerald
%contraction, this “Einstein contraction" is a
%symmetrical relation between the measurements
%made in any two inertial systems and is a
%consequence of the intersystemic relativity of
%simultaneity, because it relates lengths deter-
%mined from diflerent inertial perspectives of
%measurement, instead of contrasting conflicting
%claims concerning the results obtained under the
%same conditions of measurement.

%The philosophical mistake in Lorentz's theory is that he insisted that the moving rod must have the length $\sqrt{ }\left(1-v^2 / c^2\right)$. The assertion that the length of the moving rod is really $\sqrt{ }\left(1-v^2 / c^2\right)$ is the Lorentz-Fitzgerald Contraction Hypothesis. The significant point is that it differs from

%Now according to classical theory one would expect the coordinate difference of the rod in motion to be one, just as the coordinate difference of the rod at rest is one. 

%1. The Lorentz contraction asserts that the coordinate difference of the rod in motion is less than one by the factor $\sqrt{\left(1-v^2 / c^2\right)$ with respect to the length measured by the classical light-geometry. Indeed, that the time taken is different in this ses Therefore, since both STR and Lorentz accepted the result ofthe Michelson-Morley experiment there is in both theories a Lorentz contraction.

%If we make this decision, then while we will say that the rod has a length of one whether at rest or in motion, never-theless when it is in motion it is shorter than we would expect is to be on the classical theory. However, that rod i moving and have the same lenght. 

%The philosophical mistake in Lorentz's theory is that he insisted that the moving rod must have the length $\sqrt{ }\left(1-v^2 / c^2\right)$. The assertion that the length of the moving rod is really $\sqrt{ }\left(1-v^2 / c^2\right)$ is the Lorentz-Fitzgerald Contraction Hypothesis. The significant point is that it differs from. That the correct lenght is that of classical light geometry.

%We now have three elements to Lorentz's theory: there are the two conventions regarding spatial and temporal measurements - a unit rod when placed in motion has the shorter length $\sqrt{ }\left(1-v^2 / c^2\right)$, and the unit indication of the standard clock when placed in motion has the greater length $1 / \sqrt{\left(1-v^2 / c^2\right)}$. With repset to the classical theory, sorted nad with respect;



%For example, if we wanted to measure the length of a bus moving in front of us, we would be inclined to run along side the bus with our measuring rod. But then we would not be at rest, but moving with the bus. 

%system to be $\sqrt{\left(1-v^2 / c^2\right) \text {If we follow Einstein's convention of calling }}$ the length of the rod one whether it is at rest or in motion, then the length of the rod as measured in the moving system is one while its length measured in the rest system is only $\left.\sqrt{\left(1-v^2\right.}-c^2\right)$. It is interesting to note here, however, that if we use Lorentz's convention regarding length, then the unit rod when placed in motion has a length $\left.\sqrt{\left(1-v^2\right.}-c_c^2\right)$. Therefore, when the moving rod is then measured in the rest system using Einstein's technique it will be found to have the same length, that is, no Einstein contraction occurs if we use Lorentz's convention regarding length.

%The length of the moving rod as measured from the rest system is the distance between these two simultaneous events as measured in the rest sytem. 

%To introduce a new concept of length, that is that lenght of the moving system is that time enters in the definition of leght, that is that length of a rod, shorted but the on the moving system. ... 'Einstein contraction' is a symmetrical or reciprocal relation between the measurements made in any two inertial systems. The Einstein contraction,

%Both agree, bowever, this a deflection from the trhue Newotnian-soace rods nad clocks are; hwule, that rolds and clocks are ideal and beheave acroding to \Mink is the true geometry; it doe the difence is in the xplation; as deflection; that does not provide any explanation and simply declear ...

%However, that the classical theory as a strnaded, that have the smae length 1, In irde to appea, that the coordinate difference is shorted $\sqrt{\left(1-v^2 / c^2\right)}$

%%2. sights to realize that we are free to call the length of the moving rod in the moving coordinate system one if we wish, despite its having a coordinate difference of only $\left.\sqrt{\left(1-v^2-\right.} / c^2\right)$. If we make this decision, then while we

%1. rigid rod when placed in motion is shorter than one would expect it to be on the classical theory. Of course light is the if we use the classical theory to measure distance, is then shorted than expected since the length.
%1

%direction of the x axis and also place the same coordinate system in motion along with the rod.



%As in his book, Reichenbach uses the following notation. Let's call $l$ a rod that behaves according to the Lorentz-Einstein theory and $L$ a rod that behaves according to the classical theory. Let's label $K$ the rest system and $K'$ the moving system. The rest lengths of $l$ and $L$ in $K$ are equal, or, as Reichenbach put it, $\lK{}{}=\LK{}{}$. In his notation the upper index refers to the system in which the rod is measured and the lower refers to the one in which the rod is at rest. Thus the rest-length of a moving rod from the perspective of a co-moving frame, in other terms its proper length, is $\lK{'}{'}$; the length of a rest-rod from the perspective of the moving frame $\lK{'}{}$ is its coordinate length. 
%
%The rest length $\lK{}{}:\LK{}{}$ re qual. Now let's consider what happens in the system $K'$ that is in uniform motion with respect to $K$. 
%
%
%\begin{itemize}
%\item The \emph{Lorentz contraction} is concerned with the ratio $\lK{'}{'}:\LK{'}{'}$. That is, the Lorentz contraction compares the behavior of the same rod in the Lorentz-Einstein and classical theories in the same inertial system $K'$. 
%
%\item The \emph{Einstein contraction} whereas the Einstein contraction is concerned with the ratio $\lK{}{'}:\lK{}{}$. compares the behavior of two rods in the Lorentz-Einstein theory in different inertial systems, $K$ and $K'$. 
%\end{itemize}


%One may construct a geometry of light* in which light determines the comparison of spatial distances. Thus light comes to serve as the ordering net of physics, which gathers within the meshes of its rays all the events of the world and puts them in a numerical order. With this idea in mind, one mayfurther represent the content of Einstein s theory of space-time in the follow ing way. Clocks and yardsticks, the material instruments for measuring space and time, have only a subordinate function. They adjust themselves to the geometry of light and obey all the laws which light furnishes for the com-

%parison of magnitudes. One is reminded of a magnetic needle adjusting itself to the field of magnetic forces, but not choosing its direction independently. Clocks and yardsticks, too, have no independent magnitude; rather, they adjust themselves to the metric field of space, the structure of which manifests itself most clearly in the rays of light.


%\footnote{\c{This is consistent with Ilse Rosenthal-Schneider's account of a conversation in ‘Erinnerungen an Gespriche mit Einstein'. typed manuscript, c. 1957. CPAE. 20 295, p, 2: “What would happen, if the report that the American physicist [Miller?] really had established the ‘absolute stationary ether' were correct? In answer to this, he [Einstein] said: ‘Then the entire theory of relativity would just be nonsense}} 

%Whereas Reichenbach was ready keep the light geometry, and drop the matter axioms to save the theory, Einstein wanted to drop the theory entirely.**

%\footnoteh{Einstein was initially forced to take position towards this problem by a discussion on the problem of rigid body in \sr.  Einstein's thought experiment, referred to by Pauli, was prompted by an article in which Vladimir Varicˇak argued that in Einstein's theory the Lorentz contraction was not a physically real effect but “only an apparent subjective phenomenon produced by the manner our clocks are regulated and lengths

%are measured.”correspondence with Vari\'{c}ak \q{Sie sehen, dass die Verk\"urzung nicht einfach durch die Definition der Gleichzeitigkeitigkeit bedingt, d.h. rein konventioneller Natur sei, f\"uge ich bei: es ist unm\"oglich, die Uhren so zu verstellen, dass auch nach der Verstellung der Stab, wenn er mit den Uhren gemessen die Geschwindigkeit $\pm v$ besitzt, stets dieselbe L\"ange $l'$ mit Bezug auf habe}. The consequence if the reality of the Lorentz contraction, followed from the fact that \q{dass eine Drehung ohne elastische Deformation nach der Relativit\"atstheorie ausgeschlossen ist, wenn man hinzunimmt, dass eine transversale Verk\"urzung nicht auftritt} \lettercpaep{Einstein}{Vari\'{c}ak}{24}{2}{1911}[5(11)][255b]}


%This is a quite interesting claim since, as John Stachel has pointed out, is that after 1905 Einstein usually cited the Michelson-Morley experiment in support of the relativity postulate, never in support of the light postulate**. 

%And in Einstein's 1905 paper itself, the light
%postulate says that the velocity of light is independent of the velocity of its source,
%not, as the folklore story wants to have it, that it is independent of the velocity of
%the inertial frame in which it is measured.

%the vacuum speed of light was dependent upon direction

%11.4.1926 Hed Born Raplh Zuar  %D.~C.~Miller, author of a well-known textbook on acoustics, in September 1924
%and March/April 1925, he took the decisive measurement. American generosity, he was able to make 5000 individual measurements in all and detected the same positive result that was
%seen in 1921. In this respect, the negative results is a lot easier to explain than the positive one. Can we then say that the aether-wind hypothesis has been
%proven?

%hat
%natural measuring instruments follow coordinative definitions different
%from those assumed in the classical theory. This statement is, of
%course, empirical. On its truth depends only the physical theory of
%relativity. However, the philosophical theory of relativity, i.e., the
%discovery of the definitional character of the metric in all its details,
%holds independently of experience. Although it was developed in
%connection with physical experiments, it constitutes a philosophical
%result not subject to the criticism of the individual sciences



%Reichenbach logical structure between empirical statements and definitions. That the clear distinction between these types of sentence does not depend on a particula. 

%Responding to one his critics some months later, Reichenbach insists that in his axiomatization, \qt{the term \scare{axiom}} is used to indicate \q{\lse{empirical statements}}{yon mir Axiome genannten Satze Erfahrungssatze sind, ist in meiner Axiomatik deutlich ausgesprochen worden}, statements which lie at the foundations of Einstein's theory and can be confirmed independently of it \citeptr{Reichenbach1926f}[109]{Reichenbach2006}[209]. Definitions that are neither true nor false. That light axioms are empirical making use of the notion of simultaneity that is a definition, and Einstein's definition is only the most simple one. In this sense Lorentz and Galilei's tranformations are neither true nor false. 

%These statements are however only a possible way to fill the \qrw{general framework}{eines allgemeinen Rahmens}[328][**] of the theory. The light axioms are ... whereas the definition of simultaneity is a definition. Both Lorentz and Galilei trasnforation agree only definition of simultaneity. Neither true nor false. Matter axioms are again general stamemts about rods and clocks that can be true or false.  


%\rzl{}{161ff.}

%What is conventialisty of simultaeniety, that forse that Lorentz and Einstein are the same theory. That in both cases rods adjust to the relativistic light geometry (Lorentz transformations) and not to Galilei's transformations. This is a fact, there is no ad hoc. hypothesis as Schlick had claimed.





\begin{itemize}
\item The \emph{light axioms} define the equality of spatial and temporal distances for individual frames (\german{Lichtgeometrie} or light-geometry) using light rays alone. In Reichenbach's view the light axioms in \sr do not differ from those in classical theory except for the assertion that the velocity of light is the velocity's upper limit. The relativistic light-geometry claims that light propagates in spherical waves in any uniformly moving system, whereas in the classical light-geometry light propagates in spherical waves only in the ether system. The difference depends on the choice of $\epsilon$, which is in principle arbitrary. 
%
\item The \emph{matter axioms} postulate that material systems used as rods and clocks behave in accordance with the light geometry (\german{K\"orpergeometrie} or matter-geometry). Space distances and time intervals that are light-geometrically equal will also turn out to be equal if measured, respectively, with rigid rods and ideal clocks. Thus the content of \sr can be expressed by saying that rods and clocks behave according to the relativistic light-geometry and not according to the classical one. In other terms, the Lorentz transformation, which leaves the spherical propagation of light invariant, turns out to be the transformation for measuring rods and clocks. If rods transformed according to the Galileian transformations, distances traveled by light in equal times would in general not be equal if measured by rods.
\end{itemize}

%\section{Reichenbach's Axiomatization and The Problem of 
%Explanation}



%Facts that are considered in pre-relativistic theory remain relevant. The axioms contain the fundamental facts whose existence justifies the theory; in principle, they are empirical assertions capable of experimental verification. In addition to these are the definitions through which the theory's conceptual content is constructed. These, in contrast to the axioms, are arbitrary forms of thought, capable of neither empirical confirmation nor refutation. Their arbitrariness is constrained only by certain well-understood logical demands required for a scientifically appropriate system. They must be univocal; moreover, they must lead to a scientific system characterized by certain properties of simplicity. Whether they fulfill these demands is not solely a matter of form but depends upon the validity of the axioms.

%\todo{transport time or absolute velocity}







%Uhren und Maßstäbe, die materiellen Instrumente der Raum-Zeitmessung, haben nur eine untergeordnete Bedeutung; sie fügen sich der Lichtgeometrie ein, sie befolgen also alle Gesetze, die das Licht für den Vergleich von Größen liefert. Man darf etwa an die Magnetnadel denken, die sich auf das Feld der magnetischen Kraftlinien einstellt, nicht aber ihre Richtung selbständig wählt. So haben auch Uhren und Maßstäbe keine selbständige Größe, sondern sie stellen sich auf das metrische Feld des Raumes ein, dessen Struktur am reinsten in den Strahlen des Lichtes zum Ausdruck kommt.

%One can construct a light geometry in which light also determines the comparison of two spatial distances; light thus appears as the actual ordering framework of physics, organizing all events in the world into the meshes of its rays and thereby determining them numerically.

%While in the form of relativity theory originally developed by Einstein, light serves only to establish simultaneity, in Reichenbach's axiomatization light can be used for all time measurements, including the determination of the measure of time, and even for spatial measurements.

%Reichenbach's \scare{constructive axiomatization} consisted of ten axioms, five concerning the behavior of light (\german{Lichtaxiome} or light axioms, \rom{1}-\rom{4}), and five concerning rods and clocks (\german{K\"orperaxiome} or matter axioms, \rom{6}-\rom{10}). 
%

%
%It will be shown that only through these axioms is it possible to construct a complete theory of space and time. Thereby, points at rest with respect to one another are defined without the use of rigid rods; it is a \s{light rigidity} that is defined. Likewise, uniform time is characterized through the motion of light without the use of material mechanisms. The Lorentz transformation can be derived from this. From the light geometric point of view, therefore, it requires no additional axioms at all. The difference between the Galilean and Lorentz transformations is merely arbitrary. Its relationship to rigid rods and clocks is contained in the matter axioms VI-X.



%While the light axioms are already valid in classical optics, the theory of relativity only adds the proposition that the speed of light is an upper bound for signal velocities; in this way the matter axion1s embody a deviation from the classical theory. They contain the assertion that the Lorentz transformation, which differs from the Galilean transformation only in terms of definitions~ ]sin fact the transfor- mation for rods and docks. \q{By separating axioms from definitions, we are able to distinguish between those propositions concerning the motion of light which speak of its physical characteristics and those arbitrary additions, like the concept of si1nuhaneity and we are in a position to be 8ble to univocaBy describe the behavior of materia~ objects without needing any ambiguous auxiliary concepts like \s{shortening} or \s{stretching.}}.

%The light axioms (I-V in A.) can be considered to be well confirmed. Among them, the only new fact is the limiting nature of the speed of light about which surely there is no longer any serious doubt Amongst the matter axioms, VI and VII only contain the long-held assertions about docks and rigid rods that also follow from the classical theory.  We now turn to the matter axioms that have never been exper- imental1y proven. They are as follows:  ... Since axiom VIII is a formulation of the Michelson experiment, it was considered well confirmed.  ...


%It is this separation that allows Reichenbach to frame in terms of axiomatization; Indeed,  On the michelson experiemnt  the sevond is inded, that as fact, that already p;us adding a convetion; on the contraryl that e.g. that Michelason experiemtn that leng ... \cop{Thus, although the theory of relativity can be understood to be a valid and complete theory insofar as it is founded on the Lichtaxiome, open issues remain as to the behavior of material structures in relativistic space-time. One lacuna Reichenbach mentions in the “Bericht” is confirmation of the transverse Doppler effect using Canal-rays.}; inded, that in terms that clocks dilation, that is time labsed as meaudr by light rays correspoidn to time elabsed ams measuded by clocks.






%\qt{expand upon those consequences which are especially important for physics}{dieienigen Konseqnenzen dieser Untersuchung entwickeln, die gerade fiir die Physik wichtig sind}

%Einstein has shown the way out of this logical circle: we cannot know the simultaneity of distant events at all, but can only define it. Simultaneity is arbitrary; we can lay down whatever definitions we wish concerning it, without giving rise to an error. For if we subsequently make measurements, we will invariably reach the result of the same simultaneity that we inserted by definition in the first place; this process can never lead to a contradiction.


%The first class contains assertions solely about the physical properties of light, without making any reference to material objects. The second class expresses claims about the behavior of rigid rods and natural clocks. According to Reichenbach, it can be shown that a complete "Raum-Zeit-Lehre," can be constructed on the basis of the Lichtaxiome alone, that is, a pure "Lichtgeometrie." On the other hand, as is quite familiar in pre-relativistic physics, the behavior of rigid rods and natural clocks (together with some implicit criterion regarding distant-simultaneity) can also be used to underwrite a "Raum-Zeit-Lehre." The significance of the specific "Materialaxiome" chosen by Reichenbach is that, cumulatively, they entail the identity of the Raum-Zeit-Lehre which they give rise to with that of the "Lichtgeometrie" developed on the basis of the Lichtaxiome. The payoff,

%Reichenbach will later introduce the term of art, “constructive” axiomatizion, for this manner of proceeding (Reichenbach 1924). Its virtue (emphasized in Reichenbach’s subsequent writings, but not in the “Bericht”) is that it permits the separation of the factual content from the conventional components of the theory.



%\subsection{Axiomatizaiton}

%Reichenbach's axiomatization stands apart from the conventional \s{deductive axiomatization} approach in mathematics and physics. In the deductive axiomatization method, a general abstract principle, such as a variational principle, is established as an axiom. In physics, the principle of least action is not directly, but only indirectly, empirically testable through laws. In contrast, Reichenbach proposed a \s{constructive axiomatization,} where empirical assertions, verifiable through experimentation, serve as the axioms. 


%It was one of Einstein's philosophical insights  If we make this decision, then while we will say that the rod has a length of one whether at rest or in motion, nevertheless when it is in motion it is shorter than we would expect is to be on the classical theory.

%But, on this new conception of length, one is not entitled to infer with Lorentz that a unit rod in the ether system will no longer be unity, as a matter of physical fact, once it has been transported to a moving system. For the relational conception allows us to call that same rod unity in the moving system by definition. 




%%
%
%%Es besagt, daß die Raum-messung abhängig ist von der Zeitmessung.
%
%%aß die Zeit etwas noch Tieferliegendes ist als der Raum, daß sie zu- sammenhängt mit dem tiefsten Grundsatz alles Natur-
%%erkennens, mit dem Gesetz von Ursache und Wirkung.
%
%\todo{The Einstein contraction is introduced in order to claim that are of the ``''same length``. Declare of the same lenght. that the geometry is \Mink}. 

%\todo{particular coordinate system. It was one of Einstein's philosophical insights to realize that we are free to call the length of the moving rod in the moving coordinate system one if we wish, despite its having a coordinate difference of only $\sqrt{ }\left(1-v^2-\left(c^2\right)\right.$. If we make this decision, then while we will say that the rod has a length of one whether at rest or in motion, nevertheless when it is in motion it is shorter than we would expect is to be on the classical theory.} \todo{Giannoni!!}

\todo{In discussing the Lorentz contraction we did not talk about the length of the rod in the moving system, but merely about its coordinates in a particular coordinate system, the coordinates of the moving rods in the are shorter, declared to be identical}

\todo{The is only due to the fact that are both declar}


%We wish to show that the Michelson-Morley experiment entails the Lorentz contraction on the assumption that the velocity of light is c in all directions in the 'stationary' system. In the 'stationary' system we have

%Let us assume that the lengths of rods A and B in S were one. We wish to show that the xy, coordinate of rod B is J (1 - u3/c2). Let light leave the

%time, standard clocks in the rest system indicate $1 / \sqrt{\left(1-v^2 / c^2\right)}$ units of time.


%That something similar cam be done for time Einstein dilation and Lorentz dilation
%
%It can be easily seen that the formulas exclude the relativistic time dilation. Let $\tau_K^{K^{\prime}}$ be the time elapsed on a clock at rest in $K$ as measured in $K^{\prime}$ and $\tau_{K^{\prime}}^K$ be the time elapsed on an identical clock at rest in $K^{\prime}$ as measured in $K$ (i.e., the proper times of the compared clocks will be the same), so we have:
%
%$$
%\tau_{K^{\prime}}^K=\tau, \quad \tau_K^{K^{\prime}}=\frac{\tau}{1-\frac{v^2}{c^2}},
%$$
%
%therefore
%
%
%\begin{equation}
%\tau_{K^{\prime}}^K \neq \tau_K^{K^{\prime}} .
%\end{equation}
%
%can only be argued for by analogy. It would therefore be important if the Einsteinian time delay were to be directly proven (transverse Doppler effect). Only then will synchronization by transport be empirically refuted.
%


\todo{linearity transformation}
%\subsection{Explanattion and Miller's}

%In keeping with its brevity, the “Bericht,” as I will henceforth refer to it, sets out only axioms and definitions without indicating, even in outline, proofs of the considerable claims alleged to follow

%Hilfsmittel vor; sie sind immer eliminierbar. Eine Definition ist vor allem die Einsteinsche Gleichzeitigkeit. Sie gibt eine Vorschrift an, wie die an verschiedenen Orten stattfindenden Ereignisse mit einer Zeitzahl zu beziffern sind. Es wäre durchaus irrtümlich zu glauben, da $ß$ die Gleichzeitigkeit der speziellen Relativitätstheorie den Anspruch erhöbe, "richtiger " zu sein als irgendeine andere Gleichzeitigkeit. 

%Relativitätstheorie und absolute Transportzeit’, Zeitschrift für Physik, 9(1/2): 111–117. Engl. transl. in Reichenbach (2006): 77–85.



%Metaphysik und Naturwissenschaft’, Symposion, 1(2): 158–176. Reprinted in Reichenbach (1977a), vol. 9. Engl. transl. ‘Metaphysics and Natural Science’, in Reichenbach (1978), vol. I: 283–297.


%Besonders merkwürdig ist es, dass es möglich war, die starren Massstäbe u. Uhren völlig zu eliminieren. Ich konnte allein durch Benutzung von Lichtsignalen die ganze Metrik definieren. Das ist natürlich auch eine Realdefinition, aber es zeigt sich eben, dass das Licht als Realität genügt, es vermag sogar die Starrheit zu definieren

%Die Axiome zerfallen in 2 Klassen: Lichtaxiome (l—V) und Materialaxiome (Vi—X). Die ersteren sind Behauptungen allein über die physikalischen Eigenschaften des Lichtes, ohne jede Beziehung zu materiellen Gebilden. Es wird gezeigt werden, daß allein auf diese Axiome eine vollständige Raum-Zeit-Lehre aufgebaut werden kann. Die Materialaxiome besagen die Identität der so entwickelten „Lichtgeometrie“ mit der Raum-Zeit-Lehre der starren Maßstäbe und Uhren. Es darf als wichtigstes Resultat dieser Untersuchung aufgefaßt werden, daß diese Trennung möglich ist, daß also auch ohne die Geltung der Materialaxiome, deren empirische Bestätigung noch nicht restlos durchgeführt werden konnte, die Relativitätstheorie eine gültige und vollständige physikalische Theorie ist.

%The axioms are divided into two classes: light axioms (I–V) and material axioms (VI–X). The former are statements solely about the physical properties of light, without any relation to material entities. It will be demonstrated that a complete theory of space and time can be constructed based solely on these axioms. The material axioms state the identity of the "light geometry" thus developed with the space-time theory of rigid measuring rods and clocks. It may be regarded as the most important result of this investigation that such a separation is possible, meaning that even without the validity of the material axioms, whose empirical confirmation has not yet been entirely carried out, the theory of relativity remains a valid and complete physical theory.

%1922 French paper


%This is essential to understand \qrz{\textins{w}hat is the difference between Einstein's and Lorentz's theories}{Worin besteht nun der Unterschied zwischen der Einsteinschen Theorie und der Lorentzschen?}[229][198]. In order to answer this question Reichenbach distinguishes between the following two statements \rzl{229}{198}:
%
%%[(a)] 
%\begin{enumerate}
%\item the length of the moving rod $\lk{}{'}$ measured from the rest frame is different from its proper length $\lk{}{}$. As is well known, the proper length is greater than any coordinate length; the difference disappears only for the co-moving observer.
%
%\item the rest-length of the moving rod $\lk{'}{'}$ is different from the rest-length of another rod $\Lk{'}{'}$ which moves with it but satisfies the classical theory. The relativistic proper length in the moving frame is shorter than the classical length would be as judged from the same frame.
%\end{enumerate}


%As Reichenbach put it, in Minkowski space-time a number $\Delta s$ is coordinated to the coordinate differences $\Delta x_1,\, \dots,\, \Delta x_4$ by means of the fundamental metrical formula $\Delta s^2= \Delta x_1^2+\Delta x_2^2+\Delta x_3^2-\Delta x_4^2$. The minus sign in the rule for computing real distances from coordinate distances is responsible for all the differences between the Minkowski and Euclidean geometry. The lines which are at a constant distance from the origin at $\Delta s$ satisfy the equation $\Delta x^2 = \Delta x_1^2-\Delta x_4^2$ rather than $\Delta x^2 = \Delta x_1^2+\Delta x_4^2$. The contour lines are hyperbolae, and not circles like in Euclidean geometry. The four-hyperbolae in \cref{mr}, like the unit circle in Euclidean geometry, are the set of all points at $\Delta s=1$ distance from $O$. The hyperbola $\Delta x^2 =0$ degenerates into the two asymptotes, which group all other events in space-time into three different classes of intervals characterized by the sign of the quantity $\Delta s^2$. As one might expect, Reichenbach's next step is to find measuring instruments that behave like the indefinite type of metric, just like the behavior of the rods correspond to the definite one. 

%The physical realization of the negative $\Delta s^2$ is a physical object that satisfies the relations of congruence defined by the hyperbolas of quadrants \rom{1} and \rom{2}. The realization of the positive $\Delta s^2$ is a physical object that satisfies the relations of congruence defined by the hyperbolas of quadrants \rom{3} and \rom{4}. The first is called a time-like interval $\Delta s^2 = -1$ and is realized by the proper time of a clock. The rotation of the interval $OQ$ into the position $OQ'$ represents a moving clock. $\Delta s^2 = 1$ is the space-like interval and is realized by the proper length of a rod. The rotation of interval $OS$ into $OS'$ sets the rod into motion. Light rays realize $\Delta s^2 = 0$, the limiting velocity, which cannot be reached but only approached arbitrarily closely. Otherwise rods and clocks behave by following the hyperbolic contour lines.

%As Reichenbach rightly notices, there is a deep disanalogy between clocks and rods. Clocks are intrinsically four-dimensional measuring instruments, since they measure distances between two events. Measuring rods, on the other hand, are three-dimensional measuring instruments; they can be treated as four-dimensional instruments, if events are produced at their endpoints according to the appropriate definition of simultaneity \rzl{217}{187}. It is from this difference that all difficulties arise concerning the behavior of rods.

% \subsection{Lorentz vs. Einstein Contraction in Minkowski Space-Time}



%\qrz{This assertion of the theory of relativity is based mainly on the Michelson experiment}{Diese Behauptung der Relativit\"atstheorie st\"{u}tzt sich vor allem auf den Michelson-Versuch}[226][195]. The Michelson experiment proves that material rods satisfy the light-geometrical definition of congruence in all inertial systems. The matter-geometrical equality of distances happens to coincide with the light-geometrical equality. In other terms, rods set in motion behave according to the hyperbolic contour lines in quadrant \rom{4}. Consider again \cref{mv}. $OM_1$ and $OM_2$ are regarded as equally long if light rays need equal time when they are sent back and forth along $OM_1O$ and $OM_2O$. The negative result of the Michelson-Morley experiment establishes that if $OM_1$ and $OM_2$ are equally long if measured through light signals (in terms of the absence of interference fringes), then small measuring rods that are placed along the arms also mark off an equal number of segments on both arms:
%
%\begin{equation} \label{mic} (OM_1O=OM_2O) \rightarrow (OM_1=OM_2) \end{equation}
%
%According to the classical theory, the implication \eqref{mic} is satisfied only in the ether frame. In all other frames moving through the ether, the rest-length of the rod oriented in the direction of motion will no longer satisfy the implication \eqref{mic}. If $OM_1$ and $OM_2$ are the arms of the Michelson-Morley interferometer, then the fact that the matter-geometrical equality of distances does not coincide with the light-geometrical equality is revealed empirically by the shift in the patterns of light and darkness detected by the apparatus. The Lorentz-Einstein theory claims that the implication \eqref{mic} is satisfied in all frames; the light-geometrical distance always coincides with the matter-geometrical distance in all inertial systems. Light geometry is the same in both cases.
%
%Thus, both Einstein and Lorentz's theories assume that the proper length of the arm of the Michelson apparatus lying in the direction of motion is shorter than it \emph{would be} according to the classical theory. An objection that immediately comes to mind is that it is impossible to compare two magnitudes belonging to different theories, since there is no common standard of comparison. However, according to Reichenbach, his axiomatization provides the common standard: \qrz{In this case, the \latin{tertium comparationis} is light, which in terms of light-geometrical definitions supplies a standard to which the rods of the different theories can be compared}{Das tertium eomparationis ist hier das Licht, welches in der lichtgeometrischen Maßbestimmung ein Maß liefert, an dem die Stabe der verschiedenen Theorien gemessen Werden}[197][228]. The hyperbola in quadrant \rom{4} (\cref{mr}) defines the distance $1$ from $O$. Rods in motion, that is, rotated around $O$, follow the hyperbolas in the Lorentz-Einstein theory, but they do not do so in the classical theory. In this way, in Reichenbach's view, it is possible to compare rods $I$ and $L$, though only one of them has an actual physical existence. 
%
%Thus, the Lorentz contraction is a \emph{real difference}, just as the pressure of gas is really lower according to van der Waals equation than it would be according to the ideal gas equation. This does not mean that Einstein contraction is \scare{apparent}. Reichenbach prefers to speak of a \emph{metrogenic} or (since motion is implied) metrokinematic difference: it depends on the fact that two observers in relative motion measure two different three-dimensional cross-sections of the world-strip of the rod; thus it is a \scare{perspectival difference} \rzl{228--229}{197}. The Michelson experiment implies a real difference between the classical theory and Einstein's theory, as well as between the classical theory and Lorentz's theory. But \emph{there is no difference between Einstein's and Lorentz's theories}: \qrz{The concept of simultaneity does not enter into this problem at all}{In diesem Tatbestand kommt der Gleichzeitigkeitsbegriff \"{u}berhaupt nicht vor}[229][198].
%
%%Thus the difference between Lorentz and Einstein contraction runs as follows: The Lorentz contraction claims that $OS'$ is shorter than $OS'_2$. The Einstein contraction maintains that $OS$ is shorter than $OS'$.



%Thus Einstein's theory, just like Lorentz's, implies a contraction that is independent of the relativity of simultaneity, namely, the Lorentz contraction that implies a comparison of lengths $\lk{'}{'}<\Lk{'}{'}$, i.e., $OS'<OS'^2$ in the same moving frame but in different theories. In addition, however, it contains the Einstein contraction, which compares lengths in the same theory: the proper length and the coordinate length $\lk{}{'} < \lk{}{}$, i.e., $OS < OS_1$. As we have seen, Reichenbach maintained the opinion that the two contractions only happen to amount to the same Lorentz factor as a consequence of the linearity of the Lorentz transformations. The \emph{numerical identity} $OS_1:OS=OS':OS'^2$ is coincidental and it  conceals a deeper \emph{conceptual difference}. 

%As rotations in Euclidean geometry moves points along circular arcs, boosts in relativity move points along hyperbolic arcs. As is only invariant in the sense that any given point lies on only one hyperbola (only one level curve of $x2 - c24$) which has : unique As with respect to the origin.
%
%The quantity is invariant by definition along any single, given hyperbola.
%
%The metrogentic, is the persoectival. However, that with respect to ... indeed, explect htat is equal to the coordinate d...

%\footnotep{Recall the barn-pole paradox. There is a barn that is, say, 4/5 of the length of a pole, so that the pole does not fit into the barn if they are placed next to each other while both at rest. A relativistic pole moving at $3/5$ of the speed of light becomes $\sqrt{1-(3/5)^2}=4/5$ shorter than the pole at rest. The hope that it will fit into the barn, however, is misplaced in \sr: what we have compared is the \emph{coordinate length} of the moving pole with the \emph{proper length} of the barn (Einstein contraction). However, Reichenbach seems to claim that \emph{the relativistic pole could be made to fit into the classical barn} if they were both at rest in the moving system (Lorentz contraction)}

%\q{Als grundsätzliche Hypothese der Einsteinsehen Kinematik läßt sich dann der Satz aufstellen, daß die Lichtgeometrie identisch ist mit der Geometrie der starren Maßstäbe und natürlichen Uhren}. Reichenbach was moved by the conviction that complicated structures such as rods and clocks should be introduced at \q{the \lse{end} of a physical theory, not at its beginning}, since a \qt{knowledge of their mechanisms presupposes a knowledge of the physical laws}{so komplexe Gebilde wie starre K\"orper und Uhren vermieden wird; diese Gebilde geh?ren an das \ls{Ende} der physikalischen Theorie, nicht an den Anfang, da die Kenntnis ihres Mechanismus bereits die Kenntnis aller physikalischen Gesetze voraussetzt.}[]\citeptra[365]{Reichenbach1922}[41]{Reichenbach1978}. 


% Not only did Einstein mentions Schlick's work approvingly, but he also declared that Poincaré was \emph{sub specie aeterni} correct in claiming that so that in principle only geometry plus physics could be compared by experience. In reviewing the book version of Einstein's lecture Schlick could interpret this claim as a confirmation of the form of \s{holistic conventionalism} \citep{Schlick1921a} that he had attributed to Poincaré and, then, extended to Helmholtz in his edition of the latter's epistemological writings \citep{Helmholtz1921}. As Reichenbach recognized in his review, one of the merit of the Schlick-edition was to have shown that \qt{Poincaré did not express conventionalism more clearly}{klarer hat auch Poincaré den Konventionaliomus night ausgesprocaen} than Helmholtz already did \citep{Reichenbach1921b}. He hoped to finally meet Schlick in person in Jena at the meeting of the \GDNA, where he was going to present his project for an axiomatization of \sr that he had developed in the previous months (\letter{Reichenbach}{Schlick}{17}{9}{1921}[][SN]). 

% for the following reasons: (1) it does not express, the important Kantian intuition that the non-empirical principles are \s{constitutive} for the concept of the object; (2) it overemphasizes the arbitrary nature of the principles of knowledge, while downplaying the fact that their combination is no longer arbitrary \citep[**]{Reichenbach1922a}. 

%(\lettercpae{Freundlich}{Einstein}{24}{3}{1922}[13][119])



%From these axioms, he derived the full theory, incorporating additional conceptual elements, such as definitions. These definitions, which Reichenbach termed "coordinative definitions," are arbitrary and lack a truth value. Reichenbach considered Einstein's well-known definition of simultaneity to be a coordinative definition. This definition asserts that when light signals are sent from a source to a mirror at rest and back, the one-way time $\epsilon$ is half of the round-trip time. However, any value $0<\epsilon<1$ could, in principle, be chosen.
