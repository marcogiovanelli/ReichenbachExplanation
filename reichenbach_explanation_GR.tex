% !TEX root = reichenbach_explanation.tex

\subsection{Reichenbach Dynamical Interpretation of General Relativity}

That Reichenabch was ultaly, amrginal. Hwove,r that consdierht ethe case. To this purpuse some analitycal elemth should be tinodee. It is important to emphasize that this is not a marginal aspect of Reichenbach's philosophy. According to Reichenbach the very same problem emerges in \gr when the non-Euclidean nature of the continuum is taken into account. In this case, too, he resorts to the expression \scare{adjustment} and he refers his readers to the very same \S31 of the book. The insistence on the problem of explanation must bot be taken as. Indeed, it becomes the same role of \gr. 


As we have seen, because, Einstein redefined the length of simultaneity that it became advantageous to combine space-time into a single geometrical structure, that is a four-dimensional manifold, that \Mink called the \scare{World} or \st (\S24). This simply means that it takes four numbers, the so-called \emph{coordinates}, $x_\nu$ (where $\nu=1,2,3,4$) to identify a \wpo, namely three numbers $x_1,x_2,x_3$ for its spatial location and one for time $x_4$. The set of points whose coordinates are defined by $x_\nu(s)$ where $s$ is an arbitrary parameter is called a \wl{}. At this stage, coordinates $x_\nu$ are nothing but identification numbers, that is, in Reichenbach's parlance, they have only a \scare{topological} function; they determine the order of the \emph{between} relation. In order to define distances between points, one needs to introduce an expression that assigns a number $ds$ to the coordinate differences $dx_\nu$ between two close \wpo{}s, the so-called \scare{fundamental metrical form}. In the case of \Mink \spti, it is always possible to choose the coordinate numbers \xn so that any distance satisfies the relation.
%
That the behavior of small \rac and satisfy the following relations:
%
\begin{equation}\label{eq:mink}
\diff s^{2} = \diff x_{1}^{2}+\diff x_{2}^{2}+\diff x_{3}^{2}-\diff x_{4}^{2}\,.
\end{equation}
%%
This metrical (that is \scare{measurement}) formula allows to calculate the distance $ds$ between two nearby points from their coordinate differences $dx_\nu$. A metric that has positive as well as negative signs is called \scare{indefinite} \rzlp{**}{188--189}. The physical realization of the negative $\diff s^2$ is a physical object that satisfies the relations of congruence defined by the hyperbolas of quadrants \rom{1} and \rom{2}. The realization of the positive $\diff s^2$ is a physical object that satisfies the relations of congruence defined by the hyperbolas of quadrants \rom{3} and \rom{4}. The first is called a time-like interval $\diff s^2 = -1$ and is realized by the proper time of a clock. $\diff s^2 = 1$ is the space-like interval and is realized by the proper length of a rod. Light rays realize $\diff s^2 = 0$, the limiting velocity, which cannot be reached but only approached arbitrarily closely. Otherwise rods and clocks behave by following the hyperbolic contour lines.

As is well-konw pecial realtivity, that Lorentsz recatug coordiantes, invariance of \cref{eq:mink} that  itself in the existence of a group of transformations which leave invariant the four-dimensional distance or interval between two points. In \Mink, space-time one can of also introduce a non-rectangular coordinate system, for example polar, cylindrical coordinates\etc, or arbitrary curvilinear coordinates. If the topological relations remain unchanged, the information about the reciprocal distances among any pair of points is lost. In order to recover the distance between two points in the new coordinate system, one needs to know the \scare{generalized fundamental metrical form}. In Einstein's notation (where summation over repeated indices is implied), it reads:

\begin{equation}\label{eq:lineelement}
ds^2=\gmn dx_\mu dx_\nu\,.
\end{equation}
%
In Reichenbach's interpretation \cref{eq:lineelement} leaves the \scare{topological} function of numbering to the coordinate system, and assigns the \scare{metrical} function of measuring spatial lengths and time intervals to the metrical coefficients \gmn \rzlp{**}{**}. The \gmn are numbers by which coordinate differences $dx_\nu$ have to be multiplied so that, over larger region of \st, identical rods measure $ds=+ 1$ in every position and in every orientation, identical clocks $ds=-1$, and light rays $ds =0$. For example, a unit rod laid along the $x_1$-axis ($dx_2=0,dx_3=0,dx_4=0$) will measure $-1=\sqrt{g_{11}}dx_1$; a unit clock at rest will measure $1=\sqrt{g_{44}}dx_4$. By transporting our \rac, we can determine $g_{11}= ds^2/dx_1^2$, $g_{44}= ds^2/dx_4^2$ and in general all values of the \gmn in a given coordinate system. 

This measurement procedure is meaningful under the condition that the ratio $n_1/n_2$ is fixed once and for all, so that identical rods and clocks always measure the same $ds^2$ wherever they are placed. This empirical fact. However, which is a convetion which to be rigid is  The geometry of space-time is \Mink if it is possible so that $\gmn$ are constant, the diagonal $\gmnbar=1,1,1,-1$. However, in the general case, this is not possible, as $\gmn$ can be functions of the coordinates, $\gmn \neq \gmnbar$, and space-time is non-\Mink{}ian over larger regions. This indeed, the case in the case of the presence of a real gravitational field. The \gmn are the potentials of the gravitational field. The statement of Reichenbach's famous conventionalism is in this formalism. In principle, Reichenbach pointed out, one can always reintroduce such a difference, by a universal force, that rod is not a unit rod and clock is not unit clock, by they shortened and by a field $\gamma\mn$.

In other terms one can set $\gmn=\gmnbar+\gamma_{\mu \nu}$, where $\gmnbar$ are the normal orthogonal values of the $\gmn$, and refer to the $\gmnbar$ as the \Mink geometry measured by \rac and light rays, and only to the $\gamma_{\mu \nu}$ as the some gravitational potential field determined by the path of free-falling test particles \rzlp{**}{237}. If this separation into the geometry and gravitational field is introduced, then \rac and light rays measure the $\gmnbar$, but free-falling particles follow the geodesics of the $\gamma_{\mu \nu}$ \rzlp{**}{237}. Free-falling particles will determine a \scare{measurable but distorted} geometry, which would differ from the \scare{true but hidden} \Mink geometry. However, since the $\gmnbar$ are in principle inaccessible to measurement, the splitting of $\gmnbar$ and $\gamma_{\mu \nu}$ is \q{hardly appropriate} \rzlp{**}{237}. In \gr, the gravitational field affects equally rods and clocks, light rays, and the behavior of test particles; as a consequence, all these instruments agree on the \emph{same} geometry, that is, on the same Riemannian geometry with the same curvature.

Once again this agreement is the reason \gr is to have established a new link between physical geometry and gravitation. Traditionally, \spti geometry has been associated with the behavior of ideal measuring instruments rods (geometry) and clocks (chronometry) and \scare{free} test-particles (inertial structure). In \pgrc theories, including \sr, non-accelerating \rac, and light rays determine the geometry of space and time (that is flat \Mink geometry of special relativity) $\gmnbar$, whereas the motion of test particles determine field properties. For example, charged test particles deviate from their straight \wl in \Mink \st under the influence of the electromagnetic field. The equivalence principle makes this separation impossible in the case of the electromagnetic field. Gravitation is a \emph{universal force} that affects all bodies in the same manner. \q{This is the physical significance of the equality of gravitational and inertial mass} \rzlp{**}{257}: \q{If gravitational and inertial mass were not equal, we would not be able to look upon the paths of freely falling mass points as (four-dimensional) geodesics \textelp{}, \emph{since different geometries would result for the various materials of the mass points}} \rzlp{**}{257}. It be bacuse that are equal that that  \rac and light rays and free-falling particles define a \emph{single geometry} \rzlp{**}{256}. The distinction between the fixed geometrical background and the changing gravitatianl field disappears.

As Reichenbach points out, because of this non-separability, \q{it has occasionally been said that this conception deprives gravitation of its physical character and \emph{that gravitation, therefore, becomes geometry}} \rzlp{**}{256}. However, this conclusion, in Reichenbach's view, was not justified. Two aspects have to be distinguished:

\begin{itemize}
\item The universal effect of gravitation on all kinds of measuring instruments defines a \emph{single geometry}. \q{In this respect we may say that gravitation is \emph{geometrized}} \rzlp{**}{256}. We do not speak of deformation of our measuring instruments \q{produced by the gravitational field}, but we regard \q{the measuring instruments as \scare{free from deforming forces} in spite of the gravitational effects} \rzlp{**}{256}. what theu disagree, from \Mink space-time.

\item Even if \q{we do not introduce a force to explain the \emph{deviation} of a measuring instrument from some normal geometry}, we must still invoke a force as a \emph{cause} for the fact that \q{there is a general correspondence of all measuring instruments} \rzlp{**}{256}. One may still consider the gravitational field as the \emph{cause} of the fact that all measuring instruments happen to \textit{agree} on the same geometry.
\end{itemize}

Thus, Reichenbach points out that there two different philosophical issues at stake that should not be confused:

\begin{itemize}
\item Riemann, Helmholtz, and Poincar\'e introduced the problem of the \emph{coordinative definition} in the philosophy of geometry \rzlp{**}{257}. \Gr continued in this tradition. Once one defines $ds\pm =1$, $ds=0$ in terms of \rac and light rays, the geometry of \st can be ascertained empirically; it is a branch of physics that can be true or false.

%Conceptual definition of $ds\pm =1$, $ds=0$ in integrated by by light rays, there is a coordinative definition that \rac.

\item Einstein introduced the problem of a \emph{scientific explanation} of physical geometry, which finds its mathematical solution in the field equations. The gravitational field has an effect on \rac and light rays, that is comparable to that of any other field of force, if not for the fact that it is a universal effect \rzlp{**}{256}.

\end{itemize}
%
The issue of \s{coordination} is well-known to Reichenbach's scholars, the second one has been hardly emphasized in the literature. The conventionalist move is a stepping stone. Once, again it serve to breake the crypto-kantian prejudice that natural geoemty is Euclidean. However, once the is broken Reichenbach once again to reastiblich of geomery. Indeed, the removal of serve that only one convetional choice is legitimate, and the geometry is empirical task. 

%Still However, it has not been appreciated that 

%Reichenbach insisted on the importance of the problem of \scare{explanation} on several occasions, including in his writings on \sr. 

%Lorentz provided a \emph{dynamical explanation} for the fact that \rac do not behave according to Newtonian \spti, but they are distorted by their motion through the ether. By contrast, Einstein-\Mink provided a \emph{geometrical illustration} of the same behavior by claiming that \rac behave as they do because \spti is \Mink{}an.  In \gr, an analogous distinction can be made. However, no explation is proved, of wy theu do conferom to this geomety and not another. In \gr, an analogous distinction can be made. 

Rods and clocks, light rays, and free-falling particles all agree on the same non-Euclidean non-flat \spti geometry. Two types of explanations of this empirical fact are possible: \begin{inparaenum}[(1)] \item a \emph{dynamical explanation}: there is force, gravitation, that distorts all our measuring instruments, \rac, light rays, and free-falling particles in the same way \item a \emph{geometrical explanation}: free-falling particles light rays, \rac behave as they do the geometry of \st is non-Euclidean, and \end{inparaenum}. Like in the case of \sr, Reichenbach considered the first dynamical explanation unsuitable since it introduced an otherwise inaccessible background geometry. However, he considered the geometrical explanation, simply the renunciation of any explanation. It is simply the restament of the agrement between, light particle  and matter goemtry, happen to agree.

%It simply the repression of a physical of fact in geometrical terms, that gravitational field \textit{does not} affect the behavior of \rac. 

Thus, Reichenbach insists that also in \gr, it was necessary to provide a dynamical explanation of the observed behavior of \rac, although a dynamical explanation of a new kind. As we have seen, traditionally, we introduce a dynamical explanation to account for the fact that \rac, light rays, and test particles \emph{diverge} from an alleged standard behavior defined by a flat background geometry. However, according to Reichenbach, one needs to explain why \rac, light rays, and test particles \emph{converge} towards a \scare{single geometry}, which is general non-flat and depends on the matter distribution. In order to characterize this kind of explanation, Reichenbach resorts once again to Weyl's notion of \scare{\emph{adjustment}} (\german{Einstellung}) to contrast to that of \emph{deviation} \origg{Abweichung}: \q{The word \scare{adjustment,}, which was first used by Weyl in this connection, characterizes the problem very well} \rzlp{**}{201}. \Gr requires the introduction of a similar explanation. It cannot be an accident that, say, around the sun, measuring instruments of all kinds, light rays, free falling ecc. all agree on the same non-flat \spti geometry.

%Weyl had used the expressions to account for the non-trivial Riemannian behavior or \rac. It cannot be an accident that two measuring rods that have the same length at one place always have the same length when brought to a different place along different paths. \q{It must be explained as an adjustment to the field in which the measuring rods are embedded like test-bodies} \rzlp{**}{201}. \Gr requires the introduction of a similar explanation. It cannot be an accident that, say, around the sun, measuring instruments of all kinds, light rays, free falling ecc. all agree on the same non-flat \spti geometry. The geometrical is only statemnet of this fact, that that this fact must be explaiedn: In Reichenbach's view, \q{the word \scare{adjustment} \textelp{} poses a problem rather than supplies a solution}. As Reichenbach quite surprisingly pointed out, \q{\textins{t}he answer can, of course, be given only by a detailed theory of matter, which has not yet been elaborated} \rzlp{**}{201}.

The combination of gravitation and geometry, therefore, does not force us to forgo \emph{dynamical explanations} but teaches us that this kind of explanation is applicable even to those cases in which it was customary to resort to a \emph{geometrical explanation}.  Indeed, according to Einstein's theory, \gr teaches us that we may consider the \q{effect of gravitational fields on measuring instruments to be of the same type as all known effects of forces} \rzlp{**}{257}. What is characteristic of gravitation is that this force has an effect on the measuring instruments also used in geometry, \rac and light rays, that is on those measuring instruments that in previous theories could be insulated from the effects of physical fields \rzlp{**}{13}. The universal influence of the gravitational field on all our measuring instruments, however, does not imply that it is \q{\emph{the theory of gravitation that becomes geometry}}; on the contrary, that it implies that \q{\emph{it is geometry that becomes an expression of the gravitational field}} \rzlp{**}{256}. 

According to Reichenbach, in \gr, too, only a theory of matter can explain this peculiar behavior of the space-time measuring instruments. In the presence of a real gravitational field it is impossible to arrange rods and clocks in a rectangular grid, just like it is impossible to \scare{develop} a flat piece of paper around a sphere. That the geometry of space-time is a flat network of \rac, cannot explain the behavior of bodies—that explains why gravitational fields follow the geodesic path, and not \q{We know that a more detailed investigation would reveal the presence of molecular force-fields, which affect the molecules on the surface of the sphere and thus force it into a definite}{wir wissen, daß eine genauere Betrachtung das Vorhandensein molekularer Kraftfelder lehren W\"{u}rde, die die an der Oberfl\"ache der Kugel liegenden Molek\"{u}le angreifen und in bestimmte Bahnen zwingen} \rzl{295}{258} congruence relationship. Once again, only a dynamical expansion could complete the theory. That the Earth is spherical, however, clearly shows that we should explain how the surface acts to resist or react. This would be the proper explanation—in the case of space or space-time, only a theory of matter can provide it.

